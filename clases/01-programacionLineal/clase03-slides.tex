\documentclass[aspectratio=169,12pt,spanish]{beamer}
\usepackage[T1]{fontenc}
\usepackage[spanish]{babel}

\usepackage{wrapfig}

%\usepackage{multicol}
%\usepackage{mathtools}

\usepackage[normalem]{ulem}

\usepackage{pgf,tikz}
\usetikzlibrary{matrix}
\usetikzlibrary{arrows}

%\usepackage{wrapfig}
\mode<presentation>
\usefonttheme{professionalfonts}
\usetheme{Darmstadt}
\usecolortheme{orchid}
\useoutertheme{default}
\setbeamertemplate{headline}{}

\newcounter{savedenum}
\newcommand*{\saveenum}{\setcounter{savedenum}{\theenumi}}
\newcommand*{\resume}{\setcounter{enumi}{\thesavedenum}}

\renewcommand{\baselinestretch}{1.1}

%gets rid of bottom navigation bars
\setbeamertemplate{footline}[page number]

%gets rid of navigation symbols
\setbeamertemplate{navigation symbols}{}

%\frameframe{none} % No default frame

%\setlength{\framewidth}{8.7in} \setlength{\frameheight}{7.2in}

\parindent 0pt
\setlength{\parskip} {1ex plus 0.5ex minus 0.2ex}


\usepackage[bbgreekl]{mathbbol}
\usepackage{amssymb, amsthm, amsmath}
\usepackage{bm}

\newtheorem{ejercicio}{Ejercicio}

\DeclareSymbolFontAlphabet{\mathbb}{AMSb}
\DeclareSymbolFontAlphabet{\mathbbl}{bbold}

\usepackage{multicol}
\usepackage{colortbl}
\usepackage{lmodern}
\usepackage{tabularx}
\usepackage{multirow}
\usepackage{stmaryrd}
\usepackage{color}
\usepackage{graphicx}
\usepackage{hyperref}

\graphicspath{ {../../images} }
\usepackage{listings}
\lstset{
  basicstyle=\ttfamily,
  columns=fullflexible,
}

\usepackage{url}
\usepackage{multicol}
\usepackage{dsfont}

% Bold symbols for vectors and matrices
\newcommand{\xstar}{\bm{x}^{\star}}
\newcommand{\alphab}{\bm{\alpha}}
\newcommand{\ab}{\bm{a}}
\newcommand{\bb}{\bm{b}}
\newcommand{\cb}{\bm{c}}
\newcommand{\db}{\bm{d}}
\newcommand{\eb}{\bm{e}}
\newcommand{\gb}{\bm{g}}
\newcommand{\mb}{\bm{m}}
\newcommand{\pb}{\bm{p}}
\newcommand{\qb}{\bm{q}}
\newcommand{\rb}{\bm{r}}
\newcommand{\ssb}{\bm{s}}
\newcommand{\ub}{\bm{u}}
\newcommand{\vb}{\bm{v}}
\newcommand{\wb}{\bm{w}}
\newcommand{\xb}{\bm{x}}
\newcommand{\yb}{\bm{y}}
\newcommand{\zb}{\bm{z}}

\newcommand{\Ab}{\bm{A}}
\newcommand{\Bb}{\bm{B}}
\newcommand{\Cb}{\bm{C}}
\newcommand{\Db}{\bm{D}}
\newcommand{\Eb}{\bm{E}}
\newcommand{\Fb}{\bm{F}}
\newcommand{\Gb}{\bm{G}}
\newcommand{\Hb}{\bm{H}}
\newcommand{\Ib}{\bm{I}}
\newcommand{\Id}{\bm{I}}
\newcommand{\Kb}{\bm{K}}
\newcommand{\Lb}{\bm{L}}
\newcommand{\Mb}{\bm{M}}
\newcommand{\Pb}{\bm{P}}
\newcommand{\Qb}{\bm{Q}}
\newcommand{\Rb}{\bm{R}}
\newcommand{\Sb}{\bm{S}}
\newcommand{\Tb}{\bm{T}}
\newcommand{\Ub}{\bm{U}}
\newcommand{\Vb}{\bm{V}}
\newcommand{\Wb}{\bm{W}}
\newcommand{\Xb}{\bm{X}}
\newcommand{\Yb}{\bm{Y}}
\newcommand{\Zb}{\bm{Z}}
\newcommand{\Lambdab}{\bm{\Lambda}}
\newcommand{\cero}{\bm{0}}

% Rings and fields
\newcommand{\A}{\mathbb{A}}
\newcommand{\Z}{\mathbb{Z}}
\newcommand{\Q}{\mathbb{Q}}
\newcommand{\C}{\mathbb{C}}
\newcommand{\R}{\mathbb{R}}
\newcommand{\K}{\mathbb{K}}
\newcommand{\N}{\mathbb{N}}

\newcommand{\borel}{{\mathcal B}}
\newcommand{\pmom}{{\rho_{\text{mom}}}}
\newcommand{\MX}{{\mathcal{M}(X)}}


% Inner product
\newcommand{\innerl}[2]{\langle #1, #2 \rangle}
\newcommand{\inner}[2]{#1 \boldsymbol{\cdot} #2}
\newcommand{\innerTrace}[2]{#1 \bullet #2}

% Symmetric and positive definite matrices
\newcommand{\Splusplusn}{{\mathcal S_{++}^n}}
\newcommand{\Splusn}{{\mathcal S_+^n}}
\newcommand{\Splus}{{\mathcal S_+}}
\newcommand{\Sym}{{\mathcal S}}
\newcommand{\Symn}{{\mathcal S^n}}

% Cones
\newcommand\CC{\mathcal{C}}
\DeclareMathOperator{\cone}{cono}
\DeclareMathOperator{\conv}{conv}
\DeclareMathOperator{\supp}{supp}


% Spectrahedron
\newcommand{\eLL}{{\mathcal L}}

% Matrices and vectors over R or C
\newcommand{\Rnn}{\R^{n\times n}}
\newcommand{\Cnn}{\C^{n\times n}}
\newcommand{\Rn}{\R^{n}}
\newcommand{\Rm}{\R^{m}}


% Math operators
\DeclareMathOperator{\Tr}{Tr}
\DeclareMathOperator{\tr}{Tr}
\DeclareMathOperator{\interior}{int}
\DeclareMathOperator{\rank}{rank}
\DeclareMathOperator{\diag}{diag}

\newcommand\one{\mathds{1}} 

\pagestyle{empty}

\begin{document}

%------------------------------------------------------------------

\begin{frame}

 \begin{center}

\Large\textbf{Optimización Semidefinida} \\
\large\textbf{Clase 03 - El método simlex}
%\vspace{0.5cm}

% \textit{Santiago Laplagne} \\
%slaplagn@dm.uba.ar \\


%\vspace{0.5cm}
%{\small Trabajo en progreso en conjunto con \emph{Jose Capco} (Universit\"at Innsbruck) y \emph{Claus Scheiderer} %(Universit\"at Konstanz).} \\

\vspace{1cm}
 Segundo Cuatrimestre 2021
 \\
 {\small Facultad de Ciencias Exactas y Naturales, UBA}
 \end{center}

\end{frame}



%------------------------------------------------------------------

\begin{frame}
\frametitle{Conceptos generales de optimización}

\begin{itemize}
\item Muchos métodos de optimización se basan en un principio simple que consiste en comenzar en una solución factible cualquiera e intentar moverse a otra solución factible cercana de forma que se reduzca el costo de la función a minimizar. Si no existe ninguna solución cercana que permita mejorar la función, hemos alcanzado un mínimo local.
\item En general, un mínimo local no tiene por qué ser un mínimo global, una función podría tener varios mínimos locales, y el mínimo global ser solo uno de ellos.
\item En programación lineal un mínimo local es también global, dado que estamos minimizando una función convexa sobre un conjunto convexo.
\end{itemize}

\end{frame}

%------------------------------------------------------------------

\begin{frame}
\frametitle{Propiedad fundamental de programación lineal}

El método Simplex es uno de los métodos más usados en la práctica para resolver problemas de programaci\'on lineal y se basa en aprovechar una propiedad adicional de la programación lineal:
\begin{block}{}
Si existe un mínimo, éste se alcanza en un vértice del poliedro de puntos factibles. 
\end{block}{}

Veremos que 
\begin{itemize}
\item si estamos parados en un vértice, alcanza verificar si la función objetivo decrece al movernos a alguno de los vértices vecinos. 
\item si no decrece en ninguno de los vértices vecinos, hemos encontrado un mínimo local y por lo tanto global. 
\end{itemize}

\end{frame}

%------------------------------------------------------------------

\begin{frame}
\frametitle{Esquema del método simplex}

Obtenemos el siguiente algoritmo para minimizar una funcional $\inner{\cb}{\xb}$ sobre un poliedro $P$:
\begin{enumerate}
\item Elegir un vértice $\xb$ del poliedro $P$.
\item Para cada uno de los vértices vecinos $\{\yb_1, \dots, \yb_s\}$ de $\xb$ verificar si la función objetivo mejora en esos puntos.
\item Si no mejora en ninguno de los puntos, $\xb$ es un óptimo global y finalizamos el procedimiento.
\item Si encontramos $\yb_j$ tal que $\inner{\cb}{\yb_j} < \inner{\cb}{\xb}$, tomamos $\xb = \yb_j$ y volvemos a comenzar.
\end{enumerate}

\end{frame}

%------------------------------------------------------------------

\begin{frame}
\frametitle{Problema en forma estándar}

Para todo el desarrollo del método simplex, vamos a considerar el problema en forma estándar
\begin{alignat*}{2}
  & \text{minimizar: } & & \inner{\cb}{\xb} \\
   & \text{sujeto a: } & \quad & \Ab\xb = \bb, \\
   & & & \xb \ge 0.
\end{alignat*}

con $\Ab \in \R^{m \times n}$ matriz de rango $m \le n$.
\end{frame}

%------------------------------------------------------------------

\begin{frame}
\frametitle{Soluciones adyacentes}

Dos soluciones básicas de un conjunto de restricciones lineales en $\R^n$ se dicen \emph{adyacentes} si existen $n-1$ restricciones linealmente independientes que están activas en ambas soluciones. 

Si las dos soluciones son factibles, llamamos \emph{arista} del conjunto factible al segmento que las une.

\begin{ejercicio}
Sea $P = \{\xb \in \R^n \mid \inner{\ab_i}{\xb} \ge b_i,  1 \le i \le m\}$ un poliedro y $\ub$, $\vb$ dos soluciones básicas factibles adyacentes, con $\inner{\ab_i}{\ub} = \inner{\ab_i}{\vb} = b_i$ para $1 \le i \le n-1$ y $\{\ab_1, \dots, \ab_{n-1}\}$ vectores linealmente independientes.
Probar que el segmento $L = \{\lambda \ub + (1-\lambda)\vb \mid 0 \le \lambda \le 1\}$ que une $\ub$ y $\vb$ verifica
$$
L = \{\zb \in P \mid \inner{\ab_i}{\zb} = b_i, 1\le i \le n-1\}.
$$
\end{ejercicio}


\end{frame}

%------------------------------------------------------------------

\begin{frame}
\frametitle{Direcciones básicas}

\begin{itemize}
\item Para movernos de una solución dada a una adyacente, podríamos reemplazar alguna de las variables básicas por una variable no-básica y calcular la nueva solución básica. 
\item Sin embargo, vamos a ver que al elegir cuál es la variable no-básica que vamos a convertir en básica, esto ya determina la dirección en la que debemos movernos para desplazarnos a una solución adyacente. \pause
\item En efecto, al permitir una nueva variable no-nula, tendremos un sistema de $m+1$ ecuaciones y $m$ incógnitas, cuyas soluciones forman una recta afín.
\end{itemize}

\end{frame}

%------------------------------------------------------------------

\begin{frame}
\frametitle{Coordenadas de una solución básica}


Sea $\xb$ una solución básica factible de un problema en forma estándar y sean $B = \{B(1), \dots, B(m)\}$ los índices de las variables básicas. 

Sea $\Bb = (\Ab_{B(1)} \cdots \Ab_{B(m)}) \in \R^{m \times m}$ la correspondiente matriz base. 

En particular $x_i = 0$ para cada variable no-básica y  podemos calcular las coordenadas correspondientes a variables básicas por la fórmula
$$
\xb_{B} = \Bb^{-1}\bb.
$$


\end{frame}

%------------------------------------------------------------------

\begin{frame}
\frametitle{Direcciones básicas}

Para movernos a una solución adyacente en la que una variable no-básica $x_j$ pase a ser variable básica, debemos desplazarnos a un nuevo vector $$\xb + \theta \db$$ 
con $\theta > 0$ y $\db = (d_1, \dots, d_n)$ con $d_j =1$ y $d_i = 0$ para todos los índices $i$ distintos de $j$ de variables no-básicas. 

El vector $\xb_B$ de variables básicas va a cambiar a $\xb_B + \theta \db_B$ con $\db_B = (d_{B(1)}, \dots, d_{B(m)})$.


\end{frame}

%------------------------------------------------------------------

\begin{frame}
\frametitle{Direcciones básicas}

Como solo nos interesan las soluciones básicas factibles, debe cumplirse
$$
\Ab(\xb + \theta \db) = \bb,
$$
y como $\xb$ también es factible, $\Ab \xb = \bb$. Por lo tanto debe cumplirse $\Ab \db = \cero$ y esto nos permite calcular $\db_B$. En efecto,
$$
\cero = \Ab \db = \sum_{i=1}^n \Ab_i d_i = \sum_{i=1}^m \Ab_{B(i)} d_{B(i)} + \Ab_j = \Bb \db_B + \Ab_j,$$
donde $\Ab_i$ es la $i$-ésima columna de $\Ab$.

Como la matriz $\Bb$ es inversible, obtenemos
$$
\db_B = -\Bb^{-1} \Ab_j.
$$

\end{frame}

%------------------------------------------------------------------

\begin{frame}
\frametitle{Direcciones básicas}

\begin{itemize}
\item Llamamos \emph{$j$-ésima dirección básica} al vector $\db$ que acabamos de construir. 
\item Por construcción, las restricciones de igualdad se van a mantener si nos movemos en esta dirección. 
\item Para las condiciones de no-negatividad de las variables, recordemos que la variable no-básica $x_j$ aumenta y las demás variables no-básicas se mantienen en 0, por lo tanto conservan la no-negatividad. 
\item Para las variables básicas, como estamos suponiendo que todas las soluciones son no-degeneradas, las variables básicas son todas positivas y podemos tomar $\theta$ suficientemente pequeño para que se sigan siendo positivas.
\end{itemize}

\end{frame}

%------------------------------------------------------------------

\begin{frame}
\frametitle{Costo reducido}

Antes de calcular el valor de $\theta$ apropiado para movernos a una solución adyacente, calculamos cómo varía el costo de la función a optimizar al desplazarnos en la dirección de la $j$-ésima dirección básica.

Si $\db$ es la $j$-ésima dirección básica, la razón $\inner{\cb}{\db}$ de cambio del costo en la dirección $\db$ está dada por
$$
\inner{\cb}{\db} = \inner{\cb_B}{\db_B} + c_j,
$$
donde $\cb_B = (c_{B(1)}, \dots, c_{B(m)})$. 

\end{frame}

%------------------------------------------------------------------

\begin{frame}
\frametitle{Costo reducido}

Utilizando que $\db_B = -\Bb^{-1}\Ab_{j}$, hacemos la siguiente definición.

\begin{definition}
Sea $\xb$ una solución básica asociada a una matriz base  $\Bb$, y sea $\cb_B$ el vector de costos de las variables básicas. Para cada $j$, definimos el \emph{costo reducido} $\bar c_j$ de la variable $x_j$ por la fórmula
$$
\bar c_j = c_j - \inner{\cb_B}{(\Bb^{-1}\Ab_{j})}.
$$
\end{definition}

El costo reducido nos dice cuánto varía la función de costo al movernos una unidad en la dirección de la $j$-ésima dirección básica. El término $c_j$ indica el aumento del costo por unidad de la variable $x_j$ y el último término es el costo de modificar las demás variables para compensar el cambio en la variable $x_j$.


\end{frame}

%------------------------------------------------------------------

\begin{frame}
\frametitle{Ejemplo}

Consideremos el problema de programación lineal
\begin{alignat*}{2}
  & \text{minimizar: } & & \inner{\cb}{\xb} \\
   & \text{sujeto a: } & \quad & \Ab\xb = \bb \\
   & & & \xb \ge 0,
\end{alignat*}
con
$$\cb = (2, 0, 0, 0) \in \R^4, \quad 
\Ab = \begin{pmatrix}
1 & 1 & 1 & 1 \\
2 & 0 & 3 & 4
\end{pmatrix} \in \R^{2 \times 4}
\quad \text{ y } \quad \bb = \begin{pmatrix} 2 \\ 2 \end{pmatrix} \in \R^2.
$$


\end{frame}

%------------------------------------------------------------------

\begin{frame}
\frametitle{Ejemplo (cont.)}

Las primeras dos columnas de $\Ab$ son $\Ab_1 = (1,2)$ y $\Ab_2 = (1, 0)$. Como son linealmente independientes podemos elegir $x_1$ y $x_2$ como variables básicas. La matriz base correspondiente es
$$
\Bb = \begin{pmatrix} 1 & 1 \\ 2 & 0 \end{pmatrix}.
$$

Tomamos $x_3 = x_4 = 0$, y calculamos los valores de $x_1$ y $x_2$: $x_1 = 1$ y $x_2 = 1$. Luego $\xb = (1, 1, 0, 0)$ es una solución básica factible no-degenerada.

\end{frame}

%------------------------------------------------------------------

\begin{frame}
\frametitle{Ejemplo (cont.)}

Construimos una dirección básica $\db$ correspondiente a aumentar el valor de la variable $x_3$. Tenemos $d_3 = 1$ y $d_4 = 0$. Las coordenedas de $\db$ correspondientes a las variables básicas las obtenemos por la fórmula
$$
\begin{pmatrix} d_1 \\ d_2 \end{pmatrix} =
\begin{pmatrix} d_{B(1)} \\ d_{B(2)} \end{pmatrix} =
\db_B = -\Bb^{-1} \Ab_3 =
-\begin{pmatrix} 0 & \frac{1}{2} \\ 1 & -\frac{1}{2} \end{pmatrix}
\begin{pmatrix} 1 \\ 3 \end{pmatrix} =
\begin{pmatrix} -\frac{3}{2} \\ \frac{1}{2} \end{pmatrix}.
$$

Por lo tanto $\db = \left(-\frac{3}{2}, \frac{1}{2}, 1, 0 \right)$ y el costo de movernos en esta dirección es
$$\bar c_3 = \inner{\cb}{\db} = -\frac{3}{2} c_1 + \frac{1}{2}c_2 + c_3 = -3.$$


\end{frame}

%------------------------------------------------------------------

\begin{frame}
\frametitle{Costos reducidos de las variables básicas}

Calculamos el costo reducido de las variables básicas aplicando la definición. Consideramos $x_{B(i)}$.

Como $\Bb^{-1} \Bb = \Ib$, 
$$\Bb^{-1} \Ab_{B(i)} = \eb_i,$$
el $i$-ésimo vector canónico. Por lo tanto, 
$$
\bar c_{B(i)} = c_{B(i)} - \inner{\cb_B}{\Bb^{-1} \Ab_{B(i)}} = c_{B(i)} - \inner{\cb_B}{\eb_i} = c_{B(i)} - c_{B(i)} = 0,
$$
y comprobamos que el costo reducido de las variables básicas es 0.


\end{frame}



%------------------------------------------------------------------

\begin{frame}
\frametitle{Costos reducidos y soluciones óptimas}

El siguiente resultado confirma la idea intuitiva que tenemos de los costos reducidos y es la clave del método simplex.

\vspace{0.5cm}

\begin{theorem}
Sea $\xb$ una solución básica factible asociada a una matriz base $\Bb$, y sea $\bar \cb$ el vector de costos reducidos.
\begin{enumerate}
\item Si $\bar \cb \ge 0$, entonces $\xb$ es una solución óptima.
\item Si $\xb$ es una solución óptima no-degenerada, entonces $\bar \cb \ge 0$.
\end{enumerate}
\end{theorem}


\end{frame}

%------------------------------------------------------------------

\begin{frame}
\frametitle{Desplazamiento máximo}

Por último, si encontramos una dirección básica en la cual movernos para reducir el costo, debemos calcular el mayor valor de $\theta$ que podemos tomar, lo que equivale a determinar
$$
\theta^\star = \max\{\theta \ge 0 \mid \xb + \theta \db \in P\}.
$$

Vamos a deducir una fórmula para $\theta^\star$. Como $\Ab \db = \cero$, tenemos que
$$
\Ab (\xb + \theta \db) = \Ab \xb = \bb
$$
para todo $\theta$, por lo tanto las restricciones de igual se cumplen siempre. 

El punto $\xb + \theta \db$ solo puede salirse del poliedro si alguna de las coordenadas se vuelve negativa.
\end{frame}

%------------------------------------------------------------------

\begin{frame}
\frametitle{Desplazamiento máximo}

Distinguimos dos casos:

\begin{enumerate}
\item Si $\db \ge 0$ (es decir, $d_i \ge 0$ para todo $1 \le i \le n$), entonces $\xb + \theta \db \ge 0$ para todo $\theta \ge 0$. El vector $\xb + \theta \db$ nunca se vuelve no-factible, y tomamos $\theta^\star = +\infty$. El costo óptimo es $-\infty$.

\item Si $d_i < 0$ para algunos $i$, la restricción $x_i + \theta d_i > 0$ nos da la restricción
$$
\theta \le -\frac{x_i}{d_i}.
$$
Estas restricciones se deben cumplir para todo $i$ tal que $d_i < 0$. Por lo tanto el mayor valor de $\theta$ que podemos tomar es
$$
\theta^\star = \min_{\{i \mid d_i < 0 \}}\left(-\frac{x_i}{d_i}\right).
$$
\end{enumerate}

\end{frame}

%------------------------------------------------------------------

\begin{frame}
\frametitle{Desplazamiento máximo}


Recordemos que para las variables $x_i$ no-básicas, 
\begin{itemize}
\item o bien $d_i = 1$ si es la variable que pasa a ser variable básica, 
\item o bien $d_i = 0$.
\end{itemize}

En ambos casos $d_i \ge 0$, por lo tanto podemos restringirnos a mirar las variables básicas en la fórmula anterior. Obtenemos
\begin{equation}
\label{eq:thetastar}
\theta^\star = \min_{\{i = 1, \dots, m \mid d_{B(i)} < 0 \}}\left(-\frac{x_{B(i)}}{d_{B(i)}}\right).
\end{equation}

Más aún, como supusimos que todas las soluciones son no-degeneradas, $x_i > 0$ para todas las variables básicas, y por lo tanto obtenemos siempre $\theta^\star > 0$.

\end{frame}

%------------------------------------------------------------------

\begin{frame}
\frametitle{Ejemplo (cont.)}

Continuamos el ejemplo. Obtuvimos 
$$\xb = (1, 1, 0, 0), \quad \db = (-3/2, 1/2, 1, 0)\quad \text{ y } \quad \bar c_3 = -3.$$ 
Como $\bar c_3$ es negativo, podemos disminuir el costo de la función objetivo desplazándonos en esta dirección. Es decir, consideramos los vectores
$$
\xb + \theta \db,
$$
con $\theta > 0$. 

La única coordenada que decrece al desplazarnos es $x_1$, por lo que si tomamos cualquier valor de $\theta > 0$ tal que la primera coordenada se mantenga no-negativa, estaremos dentro del poliedro. 

\end{frame}

%------------------------------------------------------------------

\begin{frame}
\frametitle{Ejemplo (cont.)}

El máximo valor de $\theta$ que podemos tomar es
$$
\theta^\star = -\frac{x_1}{d_1} = \frac{2}{3},
$$
y nos desplazamos a
$$
\yb = \xb + \frac{2}{3} \db = \left(0, \frac{4}{3}, \frac{2}{3}, 0\right).
$$

Las columnas correspondientes a las coordenadas no-nulas son $\Ab_2 = (1, 0)$ y $\Ab_3 = (1,3)$, y son linealmente independientes. Por lo tanto $\yb$ es una nueva solución básica factible. Como queríamos minimizar el valor de $x_1$, y tenemos la restricción $x_1 \ge 0$, hemos encontrado el costo óptimo.

\end{frame}

%------------------------------------------------------------------

\begin{frame}
\frametitle{Ejemplo (cont.)}

Para verificación, calculamos el costo reducido $\bar c_4$ por la fórmula
$$
\bar c_4 = c_4 - \inner{\cb_B}{(\Bb^{-1} \Ab_4)} = 1 - \begin{pmatrix} 1 & 0 \end{pmatrix} \begin{pmatrix} 1 & 1 \\ 0 & 3 \end{pmatrix}^{-1}
\begin{pmatrix} 1 \\ 4 \end{pmatrix} = \frac{4}{3}.
$$

Vemos que movernos en la dirección básica correspondiente va a incrementar el valor de la función objetivo y por lo tanto no debemos desplazarnos en esa dirección.

\end{frame}

%------------------------------------------------------------------

\begin{frame}
\frametitle{Nueva solución básica factible}

Una vez que determinamos $\theta^\star$, suponiendo que es finito, nos movemos a una nueva solución factible
$$
\yb = \xb + \theta^\star \db.
$$

Por la construcción que hicimos de $\db$ y $\theta^\star$, si $\ell$ es el índice que minimiza la fórmula (\ref{eq:thetastar}) de $\theta^\star$ (es decir, $\theta^\star = -x_{B(\ell)} / d_{B(\ell)}$), se cumple
$$
x_{B(\ell)} = 0 \quad \text{ y } \quad x_j = \theta^\star.
$$

Obtenemos que $x_j$ reemplaza a $x_{B(\ell)}$ en el conjunto de variables básicas. El nuevo conjunto de índices de las variables básicas es $\bar B = \{\bar B(1), \dots, \bar B(m)\}$, con
$$
\bar B(i) = \begin{cases}
B(i), & i \neq \ell,\\
j, &i = \ell.
\end{cases}
$$

\end{frame}

%------------------------------------------------------------------

\begin{frame}
\frametitle{Nueva solución básica factible}

Llamamos $\bar \Bb$ a la nueva matriz base, formada por las columnas de $\Ab$ correspondientes a las nuevas variables básicas.

\begin{theorem}
\begin{enumerate}
\item Las columnas $\Ab_{B(i)}$, $i \neq \ell$, y $\Ab_j$ son linealmente independientes y, por lo tanto, $\bar \Bb$ es una matriz base.
\item El vector $\yb = \xb + \theta^\star \db$ es una solución básica factible con matriz base asociada $\bar \Bb$.
\end{enumerate}
\end{theorem}

\textbf{Demostración.} Ver \cite[Teorema 3.2]{Bertsimas1997}.

\end{frame}

%------------------------------------------------------------------

\begin{frame}
\frametitle{Algortimo del método simplex}

Mediante el procedimiento descripto, pasamos de una solución básica factible a otra solución básica factible con menor costo.
Obtenemos la siguiente iteración del método simplex.

\begin{enumerate}
\item Comenzamos con una solución básica factible $\xb$, correspondiente a variables básicas con índices $I = \{B(1), \dots, B(m)\}$.
\item Tomamos la submatriz $\Bb$ de $\Ab$ formada por las columnas correspondientes a las variables básicas, y calculamos los costos reducidos
    $$\bar c_j = c_j - \inner{\cb_B}{(\Bb^{-1} \Ab_j)}, \quad j \not\in I.$$
Si todos los costos son no-negativos, $\xb$ es una solución óptima y terminamos. Si no, elegimos algún $j$ tal que $\bar c_j < 0$.
\saveenum
\end{enumerate}

\end{frame}

%------------------------------------------------------------------

\begin{frame}
\frametitle{Algortimo del método simplex}

\begin{enumerate}
\resume
\item Calculamos $\ub = -\db_B = \Bb^{-1} \Ab_j$. Si $\ub$ no tiene ninguna componente positiva, tomamos $\theta^\star = \infty$, el costo óptimo es $-\infty$ y el algoritmo termina.
\item Si alguna componente de $\ub$ es positiva, tomamos
$$
\theta^\star = \min_{\{i = 1, \dots, m \mid u_i > 0\}} \frac{x_{B(i)}}{u_i}.
$$
\item Sea $\ell$ un índice para el cuál se alcanza el mínimo. Construimos una nueva base reemplazando $B(\ell)$ por $j$. La nueva variable básica $\yb$ queda definida por
    $$
    \begin{cases}
    y_j = \theta^\star, \\
    y_{B(i)} = x_{B(i)} - \theta^\star u_i, & 1 \le i \le m, i \neq l \\
    y_i = 0 & \text{ para los índices de variables no-básicas.}
    \end{cases}
    $$
\end{enumerate}


\end{frame}

%------------------------------------------------------------------

\begin{frame}
\frametitle{Puntos extremales y soluciones óptimas}

En base a todo lo que vimos es fácil verificar que el método simplex descripto resuelve correctamente el problema de programación lineal.

Completando el método para considerar soluciones degeneradas, podemos demostrar la siguiente propiedad de los problemas de programación lineal.

\vspace{0.5cm}

\begin{theorem}
Consideremos el problema de programación lineal de minimizar una funcional $\inner{\cb}{\xb}$ sobre un poliedro $P$. Si $P$ tiene al menos un punto extremal entonces o bien el costo óptimo es $-\infty$ o bien existe un punto extremal que es óptimo.
\end{theorem}


\end{frame}

%------------------------------------------------------------------

\begin{frame}
\frametitle{Existencia de puntos extremales}

En general los poliedros pueden no tener puntos extremales. Por ejemplo, un semiplano en $\R^2$ no tiene puntos extremales.
Una condición muy simple para determinar si un poliedro tiene puntos extremales es la existencia o no de líneas rectas incluidas en el poliedro.

Decimos que un poliedro $P \subset \R^n$ contiene una recta si existe un vector $\xb \in P$ y una dirección $\db \in \R^n$ tales que $\xb + \lambda \db \in P$ para todo $\lambda \in \R$.


\end{frame}

%------------------------------------------------------------------

\begin{frame}
\frametitle{Existencia de puntos extremales}

\begin{theorem}
Dado un poliedro $P = \{\xb \in \R^n \mid \inner{\ab_i}{\xb} \ge b_i, i = 1, \dots, m\}$, las siguientes condiciones son equivalentes.

\begin{enumerate}
\item El poliedro $P$ contiene al menos un punto extremal.
\item El poliedro $P$ no contiene ninguna recta.
\item Existen $n$ vectores linealmente independientes entre los vectores $\ab_1, \ab_2, \dots, \ab_m$.
\end{enumerate}
\end{theorem}

\textbf{Demostración.} \cite[Teorema 2.6]{Bertsimas1997}.

En particular, cualquier poliedro acotado tiene puntos extremales y cualquier poliedro en forma estándar tiene puntos extremales, debido a que el ortante positivo $\{x_i \ge 0, 1 \le i \le n\}$ no contiene ninguna recta.

\end{frame}

%------------------------------------------------------------------

\begin{frame}
\frametitle{Referencias}

\bibliographystyle{apalike}
\bibliography{../../latex/optimizacion}

\end{frame}

\end{document}

%------------------------------------------------------------------

\begin{frame}
\frametitle{Existencia de solución óptima}


Ahora que ya vimos condiciones para que un poliedro tenga puntos extremales, podemos preguntarnos en qué casos un problema de programación lineal tiene soluciones óptimas y verificar que se alcanzan en los puntos extremales, como vimos intuitivamente en los ejemplos.

\begin{theorem}
Consideremos el problema de programación lineal de minimizar una funcional $\inner{\cb}{\xb}$ sobre un poliedro $P$. Si $P$ tiene al menos un punto extremal y el problema tiene solución óptima, entonces existe una solución óptima que es un punto extremal.
\end{theorem}

\textbf{Demostración.}



\end{frame}

%------------------------------------------------------------------

\begin{frame}
\frametitle{Existencia de punto extremal óptimo}

Más aún, si el costo óptimo de la función a optimizar es finito, entonces siempre existe solución óptima.

\begin{theorem}
Consideremos el problema de programación lineal de minimizar una funcional $\inner{\cb}{\xb}$ sobre un poliedro $P$. Si $P$ tiene al menos un punto extremal entonces o bien el costo óptimo es $-\infty$ o existe un punto extremal que es óptimo.
\end{theorem}

\textbf{Demostración.} Ver \cite[Teorema 2.8]{Bertsimas1997}.


\end{frame}

%------------------------------------------------------------------

\begin{frame}
\frametitle{}

\end{frame}

%------------------------------------------------------------------

\begin{frame}
\frametitle{}

\end{frame}

%------------------------------------------------------------------

\begin{frame}
\frametitle{}

\end{frame}

%------------------------------------------------------------------

\begin{frame}
\frametitle{}

\end{frame}

%------------------------------------------------------------------

\begin{frame}
\frametitle{}

\end{frame}

