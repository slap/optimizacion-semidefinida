\documentclass[aspectratio=169,12pt,spanish]{beamer}
\usepackage[T1]{fontenc}
\usepackage[spanish]{babel}

\usepackage{wrapfig}

%\usepackage{multicol}
%\usepackage{mathtools}

\usepackage[normalem]{ulem}

\usepackage{pgf,tikz}
\usetikzlibrary{matrix}
\usetikzlibrary{arrows}

%\usepackage{wrapfig}
\mode<presentation>
\usefonttheme{professionalfonts}
\usetheme{Darmstadt}
\usecolortheme{orchid}
\useoutertheme{default}
\setbeamertemplate{headline}{}

\newcounter{savedenum}
\newcommand*{\saveenum}{\setcounter{savedenum}{\theenumi}}
\newcommand*{\resume}{\setcounter{enumi}{\thesavedenum}}

\renewcommand{\baselinestretch}{1.1}

%gets rid of bottom navigation bars
\setbeamertemplate{footline}[page number]

%gets rid of navigation symbols
\setbeamertemplate{navigation symbols}{}

%\frameframe{none} % No default frame

%\setlength{\framewidth}{8.7in} \setlength{\frameheight}{7.2in}

\parindent 0pt
\setlength{\parskip} {1ex plus 0.5ex minus 0.2ex}


\usepackage[bbgreekl]{mathbbol}
\usepackage{amssymb, amsthm, amsmath}
\usepackage{bm}

\newtheorem{ejercicio}{Ejercicio}
\newtheorem{proposition}[theorem]{Proposición}

\DeclareSymbolFontAlphabet{\mathbb}{AMSb}
\DeclareSymbolFontAlphabet{\mathbbl}{bbold}

\usepackage{multicol}
\usepackage{colortbl}
\usepackage{lmodern}
\usepackage{tabularx}
\usepackage{multirow}
\usepackage{stmaryrd}
\usepackage{color}
\usepackage{graphicx}
\usepackage{hyperref}

\graphicspath{ {../../images} }
\usepackage{listings}
\lstset{
  basicstyle=\ttfamily,
  columns=fullflexible,
}

\usepackage{url}
\usepackage{multicol}
\usepackage{dsfont}

% Bold symbols for vectors and matrices
\newcommand{\xstar}{\bm{x}^{\star}}
\newcommand{\alphab}{\bm{\alpha}}
\newcommand{\ab}{\bm{a}}
\newcommand{\bb}{\bm{b}}
\newcommand{\cb}{\bm{c}}
\newcommand{\db}{\bm{d}}
\newcommand{\eb}{\bm{e}}
\newcommand{\gb}{\bm{g}}
\newcommand{\mb}{\bm{m}}
\newcommand{\pb}{\bm{p}}
\newcommand{\qb}{\bm{q}}
\newcommand{\rb}{\bm{r}}
\newcommand{\ssb}{\bm{s}}
\newcommand{\ub}{\bm{u}}
\newcommand{\vb}{\bm{v}}
\newcommand{\wb}{\bm{w}}
\newcommand{\xb}{\bm{x}}
\newcommand{\yb}{\bm{y}}
\newcommand{\zb}{\bm{z}}

\newcommand{\Ab}{\bm{A}}
\newcommand{\Bb}{\bm{B}}
\newcommand{\Cb}{\bm{C}}
\newcommand{\Db}{\bm{D}}
\newcommand{\Eb}{\bm{E}}
\newcommand{\Fb}{\bm{F}}
\newcommand{\Gb}{\bm{G}}
\newcommand{\Hb}{\bm{H}}
\newcommand{\Ib}{\bm{I}}
\newcommand{\Id}{\bm{I}}
\newcommand{\Kb}{\bm{K}}
\newcommand{\Lb}{\bm{L}}
\newcommand{\Mb}{\bm{M}}
\newcommand{\Pb}{\bm{P}}
\newcommand{\Qb}{\bm{Q}}
\newcommand{\Rb}{\bm{R}}
\newcommand{\Sb}{\bm{S}}
\newcommand{\Tb}{\bm{T}}
\newcommand{\Ub}{\bm{U}}
\newcommand{\Vb}{\bm{V}}
\newcommand{\Wb}{\bm{W}}
\newcommand{\Xb}{\bm{X}}
\newcommand{\Yb}{\bm{Y}}
\newcommand{\Zb}{\bm{Z}}
\newcommand{\Lambdab}{\bm{\Lambda}}
\newcommand{\cero}{\bm{0}}

% Rings and fields
\newcommand{\A}{\mathbb{A}}
\newcommand{\Z}{\mathbb{Z}}
\newcommand{\Q}{\mathbb{Q}}
\newcommand{\C}{\mathbb{C}}
\newcommand{\R}{\mathbb{R}}
\newcommand{\K}{\mathbb{K}}
\newcommand{\N}{\mathbb{N}}

\newcommand{\borel}{{\mathcal B}}
\newcommand{\pmom}{{\rho_{\text{mom}}}}
\newcommand{\MX}{{\mathcal{M}(X)}}


% Inner product
\newcommand{\innerl}[2]{\langle #1, #2 \rangle}
\newcommand{\inner}[2]{#1 \boldsymbol{\cdot} #2}
\newcommand{\innerTrace}[2]{#1 \bullet #2}

% Symmetric and positive definite matrices
\newcommand{\Splusplusn}{{\mathcal S_{++}^n}}
\newcommand{\Splusn}{{\mathcal S_+^n}}
\newcommand{\Splus}{{\mathcal S_+}}
\newcommand{\Sym}{{\mathcal S}}
\newcommand{\Symn}{{\mathcal S^n}}

% Cones
\newcommand\CC{\mathcal{C}}
\DeclareMathOperator{\cone}{cono}
\DeclareMathOperator{\conv}{conv}
\DeclareMathOperator{\supp}{supp}


% Spectrahedron
\newcommand{\eLL}{{\mathcal L}}

% Matrices and vectors over R or C
\newcommand{\Rnn}{\R^{n\times n}}
\newcommand{\Cnn}{\C^{n\times n}}
\newcommand{\Rn}{\R^{n}}
\newcommand{\Rm}{\R^{m}}


% Math operators
\DeclareMathOperator{\Tr}{Tr}
\DeclareMathOperator{\tr}{Tr}
\DeclareMathOperator{\interior}{int}
\DeclareMathOperator{\rank}{rank}
\DeclareMathOperator{\diag}{diag}

\newcommand\one{\mathds{1}} 

\pagestyle{empty}

\begin{document}

%------------------------------------------------------------------

\begin{frame}

 \begin{center}

\Large\textbf{Optimización Semidefinida} \\
\large\textbf{Clase 04 - Matrices definidas positivas}
%\vspace{0.5cm}

% \textit{Santiago Laplagne} \\
%slaplagn@dm.uba.ar \\


%\vspace{0.5cm}
%{\small Trabajo en progreso en conjunto con \emph{Jose Capco} (Universit\"at Innsbruck) y \emph{Claus Scheiderer} %(Universit\"at Konstanz).} \\

\vspace{1cm}
 Segundo Cuatrimestre 2021
 \\
 {\small Facultad de Ciencias Exactas y Naturales, UBA}
 \end{center}

\end{frame}



%------------------------------------------------------------------

\begin{frame}
\frametitle{Motivación}

Consideremos una sucesión de vectores definida recursivamente por
$$
\xb[k + 1] = \Ab \xb[k], \quad \xb[0] = \xb_0
$$

El sistema tiende a 0 para todo vector inicial si y solo todos los autovalores de $\Ab$ tienen módulo menor que 1.

Dada una matriz $\Ab$ con autovalores $\lambda_1, \dots, \lambda_n$, los autovalores de $\Ab - \alpha I$ son $\lambda_1 - \alpha, \dots, \lambda_n - \alpha$.

Si $\Ab$ es simétrica, todos sus autovalores son reales, y podemos calcular su radio espectral calculando el mayor y menor autovalor.

En este caso, podemos plantear el problema de hallar el menor autovalor como un problema de optimización:
$$
\min \{\lambda : \lambda \text{ autovalor de $\Ab$}\} = \sup_{\alpha \in \R} \{\alpha : \Ab - \alpha I \succeq 0\}.
$$

\end{frame}

%------------------------------------------------------------------

\begin{frame}
\frametitle{Matrices simétricas}

Notamos $\Sym^n$ al espacio de matrices simétricas
$$
\Sym^n = \{A \in \R^{n \times n} | a_{ij} = a_{ji}, \forall 1 \le i, j \le n\}.
$$
Es un subespacio vectorial de $\R^{n \times n}$ de dimensión $\frac{n(n+1)}{2}$.

\begin{proposition}
Si $\Ab$ es simétrica, entonces
\begin{enumerate}
\item $\xb^* \Ab \xb \in \R$ para todo $\xb \in \C^n$,
\item \label{item:realav} todos los autovalores de $\Ab$ son reales,
\item autovectores correspondientes a autovalores distintos son perpendiculares,
\item $\Sb^T \Ab \Sb$ es simétrica para toda $\Sb \in \Rnn$.
\end{enumerate}
\end{proposition}


\end{frame}

%------------------------------------------------------------------

\begin{frame}
\frametitle{Teorema espectral para matrices simétricas}

\begin{theorem}
Cualquier matriz simétrica $\Ab \in \Symn$ admite una descomposición
$$
\Ab = \Ub \Db \Ub^T,
$$
con $\Ub \in \Rnn$ matriz ortogonal y $\Db \in \Rnn$ matriz diagonal, con los autovalores $\{\lambda_1, \dots, \lambda_n\} \subset \R$ de $\Ab$ en la diagonal. 
%Equivalentemente,
%$$
%\Ab = \sum_{i=1}^n \lambda_i u_i u_i^T,
%$$
%con $u_i \in \R_n$, $1 \le i \le n$, los autovectores de $\Ab$, que se corresponden con las columnas de $\Pb$.
\end{theorem}

\textbf{Idea de la demostración.} Para cualquier matriz $\Xb \in \Rnn$ podemos calcular la descomposición de Schur
$$
\Xb = \Ub \Tb \Ub^T,
$$
con $\Ub$ matriz ortogonal y $\Tb$ triangular superior. Si $\Xb$ es simétrica, $\Tb$ resulta diagonal.
Ver \cite[Teorema 4.1.5]{Horn1985}.


\end{frame}

%------------------------------------------------------------------

\begin{frame}
\frametitle{Congruencia de matrices}

En el caso de matrices simétricas existe una diagonalización más simple que la diagonalización por matrices semejantes, que es especialmente útil en problemas de programación semidefinida.

\begin{definition}
Dos matrices $\Ab, \Bb \in \Rnn$ son \emph{congruentes} si existe una matrix $\Sb \in \Rnn$ inversible tal que
$$
\Ab = \Sb \Bb \Sb^T.
$$
\end{definition}

La relación de congruencia es una relación de equivalencia.

\end{frame}

%------------------------------------------------------------------

\begin{frame}
\frametitle{Signatura de una matriz}

Dada una matriz simétrica $\Ab \in \Symn$ definimos la \emph{signatura} de $\Ab$ como la terna 
$$(n_+, n_-, n_0),$$
donde 
\begin{itemize}
\item $n_+$ la cantidad de autovalores positivos de $\Ab$, 
\item $n_-$ la cantidad de autovalores negativos, 
\item $n_0$ es la cantidad de autovalores nulos.
\end{itemize}

Si $\Ab \in \Symn$, podemos escribir $\Ab = \Ub \Lambdab \Ub^T$ con $\Lambdab =  \diag(\lambda_1, \dots, \lambda_n)$, la matriz con los autovalores de $\Ab$ en la diagonal, y $\Ub$ unitaria. Suponemos que los autovalores positivos son los primeros, luego los autovalores negativos y finalmente los autovalores cero.


\end{frame}

%------------------------------------------------------------------

\begin{frame}
\frametitle{Ley de inercia}

Definiendo la matriz diagonal no-singular real
$$\Db = \diag(\sqrt{\lambda_1}, \dots, \sqrt{\lambda_{n_+}}, \sqrt{-\lambda_{n_{+}+1}}, \dots, \sqrt{-\lambda_{n_{+}+n_{-}}}, 1, \dots, 1),$$
obtenemos
$$
\Lambdab = \Db \begin{pmatrix}
1 &        &   &    &        &    &   &        &   \\
  & \ddots &   &    &        &    &   &        &   \\
  &        & 1 &    &        &    &   &        &   \\
  &        &   & -1 &        &    &   &        &   \\
  &        &   &    & \ddots &    &   &        &   \\
  &        &   &    &        & -1 &   &        &   \\
  &        &   &    &        &    & 0 &        &   \\
  &        &   &    &        &    &   & \ddots &   \\
  &        &   &    &        &    &   &        & 0 \\
\end{pmatrix}
\Db.$$
  
\end{frame}

%------------------------------------------------------------------

\begin{frame}
\frametitle{Ley de inercia}

Podemos escribir a la matriz $\Ab$ como
$$
\Ab = \Ub \Lambdab \Ub^T = \Sb
\begin{pmatrix}
1 &        &   &    &        &    &   &        &   \\
  & \ddots &   &    &        &    &   &        &   \\
  &        & 1 &    &        &    &   &        &   \\
  &        &   & -1 &        &    &   &        &   \\
  &        &   &    & \ddots &    &   &        &   \\
  &        &   &    &        & -1 &   &        &   \\
  &        &   &    &        &    & 0 &        &   \\
  &        &   &    &        &    &   & \ddots &   \\
  &        &   &    &        &    &   &        & 0 \\
\end{pmatrix}
\Sb^T = \Sb \Ib(\Ab) \Sb^T,$$
donde $\Sb = \Ub \Db$ es una matriz no-singular e $\Ib(\Ab)$ es la matriz de inercia de $\Ab$.

\end{frame}

%------------------------------------------------------------------

\begin{frame}
\frametitle{Ley de inercia}

Utilizando esta escritura, la ley de inercia nos permite determinar fácilmente si dos matrices simétricas son congruentes.

\begin{theorem}
Dos matrices simétricas $\Ab, \Bb \in \Symn$ son congruentes si y solo si tienen la misma signatura.
\end{theorem}

\textbf{Demostración.}
Si dos matrices $\Ab, \Bb$ tienen la misma signatura, utilizando la construcción anterior, obtenemos que $\Ab$ y $\Bb$ son congruentes a la misma matriz de inercia. Como la relación de congruencia es una relación de equivalencia, las matrices $\Ab$ y $\Bb$ son congruentes.
\end{frame}

%------------------------------------------------------------------

\begin{frame}
\frametitle{Ley de inercia (continacuón de la demostración)}

Para la otra implicación, suponemos $\Ab$ y $\Bb$ congruentes, con $\Ab = \Sb \Bb \Sb^T$, para alguna matriz no-singular $\Sb \in \Rnn$. Como las matrices congruentes tienen el mismo rango, $n_0(\Ab) = n_0(\Bb)$ y alcanza ver que $n_+(\Ab) = n_+(\Bb)$. Notamos $k = n_+(\Ab)$.

Sean $\vb_1, \dots, \vb_k$ autovectores ortonormales de $\Ab$ correspondientes a los autovalores positivos $\lambda_1(\Ab), \dots, \lambda_{k}$ y sea $S_+(\Ab) = \langle \vb_1, \dots, \vb_{k} \rangle$.

La dimensión de $S_+(\Ab)$ es $k$, y si
$$\xb = \alpha_1 \vb_1 + \dots + \alpha_k \vb_{k} \neq 0,$$
entonces
$$\xb^T \Ab \xb = \lambda_1(\Ab) |\alpha_1|^2 + \dots + \lambda_{k}(\Ab) |\alpha_k|^2 > 0.$$

\end{frame}

%------------------------------------------------------------------

\begin{frame}
\frametitle{Ley de inercia (continación de la demostración)}

Por lo tanto,
$$
\xb^T (\Sb \Bb \Sb^T) \xb = (\Sb^T\xb)^T \Bb (\Sb^T \xb) > 0,
$$
luego $\yb^T \Bb \yb > 0$ para todo vector no nulo $\yb \in \langle \Sb^T \vb_1, \dots, \Sb^T \vb_k \rangle$, que es un espacio de dimensión $k$. Por lo tanto, $n_+(\Bb) \ge k = n_+(\Ab)$ y realizando el mismo razonamiento intercambiando $\Ab$ por $\Bb$, obtenemos $n_+(\Ab) = n_+(\Bb)$.

\vspace{1cm}

En particular, cualquier matriz simétrica es congruente a una matriz diagonal con valores $0$, $+1$ o $-1$ en la diagonal, y la cantidad de cada uno de estos valores en la diagonal es invariante, no depende de la matriz $\Sb$.

\end{frame}

%------------------------------------------------------------------

\begin{frame}
\frametitle{Diagonalización por congruencia}

Dada una matriz simétrica $\Ab \in \Symn$ podemos obtener una matriz diagonal $\Db$ congruente a $\Ab$ por eliminación gaussiana (realizando simultáneamente eliminación en filas y columnas). 

Por lo tanto, podemos calcular eficientemente tanto la diagonalización por congruencia como la signatura de la matriz.

\begin{example}
Calcular una matriz $\Db$ diagonal que sea congruente a 
$$
\Ab = 
\begin{pmatrix}
3 & 0 & 6 \\
0 & 0 & 1 \\
6 & 1 & 1 
\end{pmatrix}.
$$
Calcular los autovalores de $\Ab$ y verificar la ley de inercia.
\end{example}

\end{frame}

%------------------------------------------------------------------

\begin{frame}
\frametitle{Matrices definidas positivas}

Una matriz $\Ab \in \R^{n \times n}$ se llama \emph{semidefinida positiva} si es simétrica y
$$
\xb^T \Ab \xb  \ge 0
$$
para todo $\xb \in \R^n$. Si además $\xb^T \Ab \xb = 0$ solo para $\xb = 0$, decimos que la matriz es \emph{definida positiva}.

Notamos $\Ab \succ 0$ a las matrices definidas positivas y $\Ab \succeq 0$ a las matrices semidefinidas positivas.

Si $\Ab = \Mb^T \Mb$ para $\Mb \in \R^{m \times n}$ cualquiera, entonces
$$
\xb^T \Ab \xb = \xb^T \Mb^T \Mb \xb = \| \Mb \xb \|^2 \ge 0,
$$
lo que nos da un amplio stock de matrices positivas semidefinidas.

\end{frame}

%------------------------------------------------------------------

\begin{frame}
\frametitle{Caracterización de matrices semidefinidas positivas}

Dado un conjunto $S \subset \{1, \dots, n\}$, definimos el menores principales asociado a $S$ como el determinante de la submatriz cuadrada  $A_{S,S}$ formada por las filas y columnas de $A$ con índices en $S$.

\begin{theorem}
  Si $A \in \Rnn$ es un matriz simétrica, las siguientes propiedades son equivalentes:
  \begin{enumerate}
  \item $\Ab$ es semidefinida positiva ($\xb^T \Ab \xb \ge 0$ para todo $\xb \in \R^n$),
  \item todos los autovalores de $\Ab$ son no-negativos,
  \item $\Ab = \Mb^T \Mb$ para alguna matriz $\Mb \in \R^{n \times n}$,
  \item \emph{todos} los menores principales de $\Ab$ son no-negativos.
  \end{enumerate}
\end{theorem}

\end{frame}

%------------------------------------------------------------------

\begin{frame}
\frametitle{Caracterización de matrices definidas positivas}

Para matrices definidas positivas, obtenemos equivalencias similares.

\begin{theorem}
  Si $\Ab \in \Rnn$ es un matriz simétrica, las siguientes propiedades son equivalentes.
  \begin{enumerate}
  \item $\Ab$ es definida positiva ($\xb^T \Ab \xb > 0$ para todo $\xb \in \R^n$),
  \item todos los autovalores de $\Ab$ son positivos,
  \item \label{MTM} $\Ab = \Mb^T \Mb$ para alguna matriz $\Mb \in \R^{n \times n}$ no singular,
  \item \label{menoresprincipales} todos los menores principales de $\Ab$ son positivos,
  \item el polinomio característico de $\Ab$ tiene signos alternados: si $\chi_{\Ab}(t) = t^n + a_{n-1} t^{n-1} + a_1 t + a_0$, entonces $a_i a_{i+1} < 0$ para todo $0 \le i \le {n-1}$ (definiendo $a_n = 1$).
  \end{enumerate}
\end{theorem}

\end{frame}

%------------------------------------------------------------------

\begin{frame}
\frametitle{Menores principales de cabeza}

Si $S = \{1, 2, \dots, k\}$, decimos que el menor asociado es un \emph{menor principal de cabeza}.

En el caso de matrices definidas positivas, podemos restringir la condición (\ref{menoresprincipales}) a considerar solo los menores principales de cabeza.

\textbf{Idea de la demostración.} Usamos que para una matriz simétrica $\Ab$, los autovalores de 
$$
\hat \Ab = \begin{pmatrix} 
\Ab & \yb \\
\yb^T & a 
\end{pmatrix},$$
con $\yb \in \R^n$ y $a \in \R$ están entrecruzados con los autovalores de $\Ab$ (ver \cite[Teorema 4.3.8]{Horn1985}).


\end{frame}

%------------------------------------------------------------------

\begin{frame}
\frametitle{Factorización de Cholesky}

En ambos casos, la condición (\ref{MTM}) podemos restringirla a matrices triangulares inferiores. Recordemos que cualquier matriz $\Cb \in \Rnn$ admite una descomposición $\Qb\Rb$ y puede escribirse como $\Cb = \Qb \Rb$ con $\Qb$ unitaria y $\Rb$ triangular superior del mismo rango que $\Cb$. Luego
$$
\Ab = \Cb^T \Cb = (\Qb \Rb)^T \Qb \Rb = \Rb^T \Qb^T \Qb \Rb = \Rb^T \Rb
$$
y tomando $\Lb = \Rb^T$, obtenemos $\Ab = \Lb \Lb^T$.

\end{frame}

%------------------------------------------------------------------

\begin{frame}
\frametitle{Factorización de Cholesky}


Si $\Cb$ es no singular, podemos elegir $\Rb$ con todos los valores en la diagonal positivos (y la descomposición de esta forma es única).
Esto prueba el siguiente corolario, que nos da la \emph{descomposición de Cholesky} de una matriz definida positiva.

\begin{corollary}
Una matriz $\Ab$ es definida positiva si y solo si existe una matriz trinagular inferior $\Lb \in \Rnn$ con valores positivos en la diagonal tal que $\Ab = \Lb \Lb^T$.
\end{corollary}



\end{frame}

%------------------------------------------------------------------

\begin{frame}
\frametitle{Conos}


Un subconjunto $\CC$ de un espacio vectorial real $V$ es un \emph{cono convexo} si es cerrado por combinaciones lineales positivas. Es decir,
\begin{enumerate}
\item Si $\xb \in \CC$ y $a \ge 0$, entonces $a\xb \in \CC$.
\item Si $\xb, \yb \in \CC$, entonces $\xb + \yb \in \CC$.
\end{enumerate}

En este apunte siempre que hablemos de \emph{conos} vamos a referirnos a conos convexos.

\end{frame}

%------------------------------------------------------------------

\begin{frame}
\frametitle{Conos}

Decimos que un cono es \emph{puntiagudo} si
$$
\xb  \in \CC \text{ y } -\xb \in \CC \Rightarrow \xb = 0.
$$

El \emph{cono dual} de un cono $\CC$ es
$$
\CC^* = \{\yb \in \R^n \mid \xb^T \yb \ge 0 \enspace \forall \xb \in \CC\}.
$$

\end{frame}

%------------------------------------------------------------------

\begin{frame}
\frametitle{Orden parcial}

Dado un cono puntiagudo $\CC$ en $\R^n$, podemos definir un orden parcial en $\R^n$ por
$$
\xb \succeq \yb \iff \xb - \yb \in \CC
$$
para $\xb, \yb \in \R^n$. Este orden satisface las siguientes propiedades:
\begin{itemize}
\item \textbf{reflexividad:} $\forall \xb \in \R^n: \xb \succeq \xb$
\item \textbf{antisimetría:} $\forall \xb, \yb \in \R^n: \xb \succeq \yb, \yb \succeq \xb \Rightarrow \xb = \yb$
\item \textbf{transitividad:} $\forall \xb, \yb, \zb \in \R^n: \xb \succeq \yb, \yb \succeq \zb \Rightarrow \xb \succeq \zb$
\item \textbf{homogeneidad:} $\forall \xb, \yb \in \R^n, \forall \alpha \in \R_{\ge 0}: \xb \succeq \yb \Rightarrow \alpha \xb \succeq \alpha \yb$
\item \textbf{aditividad:} $\forall \xb, \yb, \xb', \yb' \in \R^n: \xb \succeq \yb, \xb' \succeq \yb' \Rightarrow \xb + \xb' \succeq \yb + \yb'$
\end{itemize}

\end{frame}

%------------------------------------------------------------------

\begin{frame}
\frametitle{Desigualdad estricta}

En general, nos van a interesar conos convexos de dimensión completa, es decir conos con interior no vacío. En ese caso, podemos definir la desigualdad estricta
$$
\xb \succ \yb \iff \xb - \yb \in \interior \CC.
$$

\end{frame}



%------------------------------------------------------------------

\begin{frame}
\frametitle{Separación}

Tenemos el siguiente resultado de separación.

\begin{lemma} Sean $\CC \subset \R^n$ un cono convexo cerrado y $\xb \in \R^n \smallsetminus \CC$ un punto fuera de $\CC$. Existe un hiperplano que separa a $\{\xb\}$ de $\CC$. Más aún, existe un vector no nulo $\cb \in \R^n$ tal que
$$
\forall \yb \in \R^n : \inner{\cb}{\yb} \ge 0 > \inner{\cb}{\xb}.
$$
\end{lemma}

\end{frame}

%------------------------------------------------------------------

\begin{frame}
\frametitle{Ejemplos}

\noindent\textbf{Cono generado por un conjunto de puntos.}
El cono generado por un conjunto de puntos $A = \{x_1, \dots, x_N\} \subset \R^n$ es el menor cono que contiene a $A$. Es el cono
$$
\cone A = \left\{ \sum_{i=1}^N \alpha_i x_i \mid \alpha_1, \dots, \alpha_N \in \R_{\ge 0}\right\}.
$$

\noindent\textbf{El ortante no-negativo.}
El ortante no-negativo que utilizamos en programación lineal, definido por
$$
\R^n_{\ge 0} = \{(x_1, \dots, x_n) \in \R^n \mid x_1, \dots, x_n \ge 0\}
$$
es un cono puntiagudo convexo cerrado de dimensión completa.

\end{frame}

%------------------------------------------------------------------

\begin{frame}
\frametitle{Ejemplo: el cono de matrices semidefinidas positivas}

\begin{proposition}
El conjunto $\Splusn$ de matrices semidefinidas positivas es un cono convexo.
\end{proposition}

\textbf{Demostración.}
Podemos escribir a $\Splusn$ como intersección de infinitos semiespacios:
$$
\Splusn = \{\Ab \in \Symn \mid \xb^T \Ab \xb \ge 0 \enspace \forall \xb \in \R^n\} = \bigcap_{\xb\in\R^n} \{\Ab \in \Symn \mid \xb^T \Ab \xb \ge 0\},
$$
y por lo tanto es un conjunto convexo cerrado.
Para ver que forman un cono, si $\Ab, \Bb \succeq 0$ y $a \ge 0$, es claro que $a \Ab \succeq 0$ y también que
$$\xb^T (\Ab + \Bb) \xb = \xb^T \Ab \xb + \xb^T \Bb \xb \ge 0$$
 para todo $\xb \in \R^n$.

\end{frame}

%------------------------------------------------------------------

\begin{frame}
\frametitle{Desplazamiento máximo}


Recordemos que para las variables $x_i$ no-básicas,
\begin{itemize}
\item o bien $d_i = 1$ si es la variable que pasa a ser variable básica,
\item o bien $d_i = 0$.
\end{itemize}

En ambos casos $d_i \ge 0$, por lo tanto podemos restringirnos a mirar las variables básicas en la fórmula anterior. Obtenemos
\begin{equation}
\label{eq:thetastar}
\theta^\star = \min_{\{i = 1, \dots, m \mid d_{B(i)} < 0 \}}\left(-\frac{x_{B(i)}}{d_{B(i)}}\right).
\end{equation}

Más aún, como supusimos que todas las soluciones son no-degeneradas, $x_i > 0$ para todas las variables básicas, y por lo tanto obtenemos siempre $\theta^\star > 0$.

\end{frame}

%------------------------------------------------------------------

\begin{frame}
\frametitle{Ejemplo (cont.)}

Continuamos el ejemplo. Obtuvimos
$$\xb = (1, 1, 0, 0), \quad \db = (-3/2, 1/2, 1, 0)\quad \text{ y } \quad \bar c_3 = -3.$$
Como $\bar c_3$ es negativo, podemos disminuir el costo de la función objetivo desplazándonos en esta dirección. Es decir, consideramos los vectores
$$
\xb + \theta \db,
$$
con $\theta > 0$.

La única coordenada que decrece al desplazarnos es $x_1$, por lo que si tomamos cualquier valor de $\theta > 0$ tal que la primera coordenada se mantenga no-negativa, estaremos dentro del poliedro.

\end{frame}

%------------------------------------------------------------------

\begin{frame}
\frametitle{Ejemplo (cont.)}

El máximo valor de $\theta$ que podemos tomar es
$$
\theta^\star = -\frac{x_1}{d_1} = \frac{2}{3},
$$
y nos desplazamos a
$$
\yb = \xb + \frac{2}{3} \db = \left(0, \frac{4}{3}, \frac{2}{3}, 0\right).
$$

Las columnas correspondientes a las coordenadas no-nulas son $\Ab_2 = (1, 0)$ y $\Ab_3 = (1,3)$, y son linealmente independientes. Por lo tanto $\yb$ es una nueva solución básica factible. Como queríamos minimizar el valor de $x_1$, y tenemos la restricción $x_1 \ge 0$, hemos encontrado el costo óptimo.

\end{frame}

%------------------------------------------------------------------

\begin{frame}
\frametitle{Ejemplo (cont.)}

Para verificación, calculamos el costo reducido $\bar c_4$ por la fórmula
$$
\bar c_4 = c_4 - \inner{\cb_B}{(\Bb^{-1} \Ab_4)} = 1 - \begin{pmatrix} 1 & 0 \end{pmatrix} \begin{pmatrix} 1 & 1 \\ 0 & 3 \end{pmatrix}^{-1}
\begin{pmatrix} 1 \\ 4 \end{pmatrix} = \frac{4}{3}.
$$

Vemos que movernos en la dirección básica correspondiente va a incrementar el valor de la función objetivo y por lo tanto no debemos desplazarnos en esa dirección.

\end{frame}

%------------------------------------------------------------------

\begin{frame}
\frametitle{Nueva solución básica factible}

Una vez que determinamos $\theta^\star$, suponiendo que es finito, nos movemos a una nueva solución factible
$$
\yb = \xb + \theta^\star \db.
$$

Por la construcción que hicimos de $\db$ y $\theta^\star$, si $\ell$ es el índice que minimiza la fórmula (\ref{eq:thetastar}) de $\theta^\star$ (es decir, $\theta^\star = -x_{B(\ell)} / d_{B(\ell)}$), se cumple
$$
x_{B(\ell)} = 0 \quad \text{ y } \quad x_j = \theta^\star.
$$

Obtenemos que $x_j$ reemplaza a $x_{B(\ell)}$ en el conjunto de variables básicas. El nuevo conjunto de índices de las variables básicas es $\bar B = \{\bar B(1), \dots, \bar B(m)\}$, con
$$
\bar B(i) = \begin{cases}
B(i), & i \neq \ell,\\
j, &i = \ell.
\end{cases}
$$

\end{frame}

%------------------------------------------------------------------

\begin{frame}
\frametitle{Nueva solución básica factible}

Llamamos $\bar \Bb$ a la nueva matriz base, formada por las columnas de $\Ab$ correspondientes a las nuevas variables básicas.

\begin{theorem}
\begin{enumerate}
\item Las columnas $\Ab_{B(i)}$, $i \neq \ell$, y $\Ab_j$ son linealmente independientes y, por lo tanto, $\bar \Bb$ es una matriz base.
\item El vector $\yb = \xb + \theta^\star \db$ es una solución básica factible con matriz base asociada $\bar \Bb$.
\end{enumerate}
\end{theorem}

\textbf{Demostración.} Ver \cite[Teorema 3.2]{Bertsimas1997}.

\end{frame}

%------------------------------------------------------------------

\begin{frame}
\frametitle{Algortimo del método simplex}

Mediante el procedimiento descripto, pasamos de una solución básica factible a otra solución básica factible con menor costo.
Obtenemos la siguiente iteración del método simplex.

\begin{enumerate}
\item Comenzamos con una solución básica factible $\xb$, correspondiente a variables básicas con índices $I = \{B(1), \dots, B(m)\}$.
\item Tomamos la submatriz $\Bb$ de $\Ab$ formada por las columnas correspondientes a las variables básicas, y calculamos los costos reducidos
    $$\bar c_j = c_j - \inner{\cb_B}{(\Bb^{-1} \Ab_j)}, \quad j \not\in I.$$
Si todos los costos son no-negativos, $\xb$ es una solución óptima y terminamos. Si no, elegimos algún $j$ tal que $\bar c_j < 0$.
\saveenum
\end{enumerate}

\end{frame}

%------------------------------------------------------------------

\begin{frame}
\frametitle{Algortimo del método simplex}

\begin{enumerate}
\resume
\item Calculamos $\ub = -\db_B = \Bb^{-1} \Ab_j$. Si $\ub$ no tiene ninguna componente positiva, tomamos $\theta^\star = \infty$, el costo óptimo es $-\infty$ y el algoritmo termina.
\item Si alguna componente de $\ub$ es positiva, tomamos
$$
\theta^\star = \min_{\{i = 1, \dots, m \mid u_i > 0\}} \frac{x_{B(i)}}{u_i}.
$$
\item Sea $\ell$ un índice para el cuál se alcanza el mínimo. Construimos una nueva base reemplazando $B(\ell)$ por $j$. La nueva variable básica $\yb$ queda definida por
    $$
    \begin{cases}
    y_j = \theta^\star, \\
    y_{B(i)} = x_{B(i)} - \theta^\star u_i, & 1 \le i \le m, i \neq l \\
    y_i = 0 & \text{ para los índices de variables no-básicas.}
    \end{cases}
    $$
\end{enumerate}


\end{frame}

%------------------------------------------------------------------

\begin{frame}
\frametitle{Puntos extremales y soluciones óptimas}

En base a todo lo que vimos es fácil verificar que el método simplex descripto resuelve correctamente el problema de programación lineal.

Completando el método para considerar soluciones degeneradas, podemos demostrar la siguiente propiedad de los problemas de programación lineal.

\vspace{0.5cm}

\begin{theorem}
Consideremos el problema de programación lineal de minimizar una funcional $\inner{\cb}{\xb}$ sobre un poliedro $P$. Si $P$ tiene al menos un punto extremal entonces o bien el costo óptimo es $-\infty$ o bien existe un punto extremal que es óptimo.
\end{theorem}


\end{frame}

%------------------------------------------------------------------

\begin{frame}
\frametitle{Existencia de puntos extremales}

En general los poliedros pueden no tener puntos extremales. Por ejemplo, un semiplano en $\R^2$ no tiene puntos extremales.
Una condición muy simple para determinar si un poliedro tiene puntos extremales es la existencia o no de líneas rectas incluidas en el poliedro.

Decimos que un poliedro $P \subset \R^n$ contiene una recta si existe un vector $\xb \in P$ y una dirección $\db \in \R^n$ tales que $\xb + \lambda \db \in P$ para todo $\lambda \in \R$.


\end{frame}

%------------------------------------------------------------------

\begin{frame}
\frametitle{Existencia de puntos extremales}

\begin{theorem}
Dado un poliedro $P = \{\xb \in \R^n \mid \inner{\ab_i}{\xb} \ge b_i, i = 1, \dots, m\}$, las siguientes condiciones son equivalentes.

\begin{enumerate}
\item El poliedro $P$ contiene al menos un punto extremal.
\item El poliedro $P$ no contiene ninguna recta.
\item Existen $n$ vectores linealmente independientes entre los vectores $\ab_1, \ab_2, \dots, \ab_m$.
\end{enumerate}
\end{theorem}

\textbf{Demostración.} \cite[Teorema 2.6]{Bertsimas1997}.

En particular, cualquier poliedro acotado tiene puntos extremales y cualquier poliedro en forma estándar tiene puntos extremales, debido a que el ortante positivo $\{x_i \ge 0, 1 \le i \le n\}$ no contiene ninguna recta.

\end{frame}

%------------------------------------------------------------------

\begin{frame}
\frametitle{Referencias}

\bibliographystyle{apalike}
\bibliography{../../latex/optimizacion}

\end{frame}

\end{document}

%------------------------------------------------------------------

%------------------------------------------------------------------

\begin{frame}
\frametitle{Costo reducido}

La siguiente propiedad será clave para estudiar la dualidad en programación semidefinida.

\begin{theorem}
Una matriz $A$ es semidefinida positiva si y solo si $\innerTrace{A}{X} \ge 0$ para toda matriz $X \in \Splus$.
\end{theorem}

\begin{proof}
Si $A \succeq 0$, entonces $A = B^T B$ para alguna matriz $B$ y por lo tanto,
$$
\innerTrace{A}{X} = \tr(AX) = \tr(B^T B X) = \tr(B X B^T).
$$

Como $X \succeq 0$, también $B X B^T \succeq 0$ y por lo tanto $\tr(B X B^T) \ge 0$.

Para la otra dirección, si $x \in \R^n$, entonces $x x^T \in \Rnn$ es una matriz semidefinida positiva. Si $\langle A, X\rangle \ge 0$ para toda $X$, entonces $\innerTrace{A}{x x^T}  \ge 0$ para todo $x \in \Rn$ y
$$
\innerTrace{A}{xx^T} = \tr(A x x^T) = \tr(x^T A x) = x^T A x
$$
y por lo tanto $x^T A x \ge 0$ para todo $x$.
\end{proof}

