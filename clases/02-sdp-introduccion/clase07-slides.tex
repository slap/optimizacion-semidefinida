\documentclass[aspectratio=169,12pt,spanish]{beamer}
\usepackage[T1]{fontenc}
\usepackage[spanish]{babel}

\usepackage{wrapfig}

%\usepackage{multicol}
%\usepackage{mathtools}

\usepackage[normalem]{ulem}

\usepackage{pgf,tikz}
\usetikzlibrary{matrix}
\usetikzlibrary{arrows}

%\usepackage{wrapfig}
\mode<presentation>
\usefonttheme{professionalfonts}
\usetheme{Darmstadt}
\usecolortheme{orchid}
\useoutertheme{default}
\setbeamertemplate{headline}{}

\newcounter{savedenum}
\newcommand*{\saveenum}{\setcounter{savedenum}{\theenumi}}
\newcommand*{\resume}{\setcounter{enumi}{\thesavedenum}}

\renewcommand{\baselinestretch}{1.1}

%gets rid of bottom navigation bars
\setbeamertemplate{footline}[page number]

%gets rid of navigation symbols
\setbeamertemplate{navigation symbols}{}

%\frameframe{none} % No default frame

%\setlength{\framewidth}{8.7in} \setlength{\frameheight}{7.2in}

\parindent 0pt
\setlength{\parskip} {1ex plus 0.5ex minus 0.2ex}


\usepackage[bbgreekl]{mathbbol}
\usepackage{amssymb, amsthm, amsmath}
\usepackage{bm}

\newtheorem{ejercicio}{Ejercicio}
\newtheorem{proposition}[theorem]{Proposición}

\DeclareSymbolFontAlphabet{\mathbb}{AMSb}
\DeclareSymbolFontAlphabet{\mathbbl}{bbold}

\usepackage{multicol}
\usepackage{colortbl}
\usepackage{lmodern}
\usepackage{tabularx}
\usepackage{multirow}
\usepackage{stmaryrd}
\usepackage{color}
\usepackage{graphicx}
\usepackage{hyperref}

\graphicspath{ {../../images} }
\usepackage{listings}
\lstset{
  basicstyle=\ttfamily,
  columns=fullflexible,
}

\usepackage{url}
\usepackage{multicol}
\usepackage{dsfont}

% Bold symbols for vectors and matrices
\newcommand{\xstar}{\bm{x}^{\star}}
\newcommand{\alphab}{\bm{\alpha}}
\newcommand{\ab}{\bm{a}}
\newcommand{\bb}{\bm{b}}
\newcommand{\cb}{\bm{c}}
\newcommand{\db}{\bm{d}}
\newcommand{\eb}{\bm{e}}
\newcommand{\gb}{\bm{g}}
\newcommand{\mb}{\bm{m}}
\newcommand{\pb}{\bm{p}}
\newcommand{\qb}{\bm{q}}
\newcommand{\rb}{\bm{r}}
\newcommand{\ssb}{\bm{s}}
\newcommand{\ub}{\bm{u}}
\newcommand{\vb}{\bm{v}}
\newcommand{\wb}{\bm{w}}
\newcommand{\xb}{\bm{x}}
\newcommand{\yb}{\bm{y}}
\newcommand{\zb}{\bm{z}}

\newcommand{\Ab}{\bm{A}}
\newcommand{\Bb}{\bm{B}}
\newcommand{\Cb}{\bm{C}}
\newcommand{\Db}{\bm{D}}
\newcommand{\Eb}{\bm{E}}
\newcommand{\Fb}{\bm{F}}
\newcommand{\Gb}{\bm{G}}
\newcommand{\Hb}{\bm{H}}
\newcommand{\Ib}{\bm{I}}
\newcommand{\Id}{\bm{I}}
\newcommand{\Kb}{\bm{K}}
\newcommand{\Lb}{\bm{L}}
\newcommand{\Mb}{\bm{M}}
\newcommand{\Pb}{\bm{P}}
\newcommand{\Qb}{\bm{Q}}
\newcommand{\Rb}{\bm{R}}
\newcommand{\Sb}{\bm{S}}
\newcommand{\Tb}{\bm{T}}
\newcommand{\Ub}{\bm{U}}
\newcommand{\Vb}{\bm{V}}
\newcommand{\Wb}{\bm{W}}
\newcommand{\Xb}{\bm{X}}
\newcommand{\Yb}{\bm{Y}}
\newcommand{\Zb}{\bm{Z}}
\newcommand{\Lambdab}{\bm{\Lambda}}
\newcommand{\cero}{\bm{0}}

% Rings and fields
\newcommand{\A}{\mathbb{A}}
\newcommand{\Z}{\mathbb{Z}}
\newcommand{\Q}{\mathbb{Q}}
\newcommand{\C}{\mathbb{C}}
\newcommand{\R}{\mathbb{R}}
\newcommand{\K}{\mathbb{K}}
\newcommand{\N}{\mathbb{N}}

\newcommand{\borel}{{\mathcal B}}
\newcommand{\pmom}{{\rho_{\text{mom}}}}
\newcommand{\MX}{{\mathcal{M}(X)}}


% Inner product
\newcommand{\innerl}[2]{\langle #1, #2 \rangle}
\newcommand{\inner}[2]{#1 \boldsymbol{\cdot} #2}
\newcommand{\innerTrace}[2]{#1 \bullet #2}

% Symmetric and positive definite matrices
\newcommand{\Splusplusn}{{\mathcal S_{++}^n}}
\newcommand{\Splusn}{{\mathcal S_+^n}}
\newcommand{\Splus}{{\mathcal S_+}}
\newcommand{\Sym}{{\mathcal S}}
\newcommand{\Symn}{{\mathcal S^n}}

% Cones
\newcommand\CC{\mathcal{C}}
\DeclareMathOperator{\cone}{cono}
\DeclareMathOperator{\conv}{conv}
\DeclareMathOperator{\supp}{supp}


% Spectrahedron
\newcommand{\eLL}{{\mathcal L}}

% Matrices and vectors over R or C
\newcommand{\Rnn}{\R^{n\times n}}
\newcommand{\Cnn}{\C^{n\times n}}
\newcommand{\Rn}{\R^{n}}
\newcommand{\Rm}{\R^{m}}


% Math operators
\DeclareMathOperator{\Tr}{Tr}
\DeclareMathOperator{\tr}{Tr}
\DeclareMathOperator{\interior}{int}
\DeclareMathOperator{\rank}{rank}
\DeclareMathOperator{\diag}{diag}

\newcommand\one{\mathds{1}} 

\pagestyle{empty}

\begin{document}

%------------------------------------------------------------------

\begin{frame}

 \begin{center}

\Large\textbf{Optimización Semidefinida} \\
\large\textbf{Clase 07 - Dualidad}
%\vspace{0.5cm}

% \textit{Santiago Laplagne} \\
%slaplagn@dm.uba.ar \\


%\vspace{0.5cm}
%{\small Trabajo en progreso en conjunto con \emph{Jose Capco} (Universit\"at Innsbruck) y \emph{Claus Scheiderer} %(Universit\"at Konstanz).} \\

\vspace{1cm}
 Segundo Cuatrimestre 2021
 \\
 {\small Facultad de Ciencias Exactas y Naturales, UBA}
 \end{center}

\end{frame}



%------------------------------------------------------------------

\begin{frame}
\frametitle{Dualidad en programación lineal}

Una de las propiedades m\'as interesantes de la programaci\'on lineal es que a cada problema en forma estándar
\begin{alignat*}{2}
  & \text{minimizar: } & & \inner{\cb}{\xb}  \\
   & \text{sujeto a: } & \quad & \Ab \xb = \bb, \\
   &&& \xb \ge 0
\end{alignat*}
que llamamos \emph{problema primal} podemos asociarle un \emph{problema dual} de la siguiente forma:
\begin{alignat*}{2}
  & \text{maximizar: } & & \inner{\yb}{\bb}  \\
   & \text{sujeto a: } & \quad & \Ab^T \yb \le \cb,
\end{alignat*}
que guarda relaciones muy fuertes con el problema primal.

\end{frame}

%------------------------------------------------------------------

\begin{frame}
\frametitle{Ejercicio}

Calcular el problema dual asociado al problema
\begin{alignat*}{2}
  & \text{minimizar: } & & 3x_1 + 2x_2 + 7x_3 + 4x_4  \\
   & \text{sujeto a: } & \quad & 4x_1 - 3x_2 + x_4 = 7, \\
   &  & \quad & x_1 + x_3 - x_4 = 9, \\
   &&& \xb \ge 0.
\end{alignat*}



\end{frame}

%------------------------------------------------------------------

\begin{frame}
\frametitle{Motivación}

Notamos $\xb^\star$ a una solución óptima, que suponemos que existe.

Podemos relajar el problema primal reemplazando la restricción $\Ab\xb = \bb$ por una penalidad
$$\inner{\yb}{(\bb - \Ab\xb)},$$
donde $\yb$ es un vector de precios del mismo tamaño que $\bb$. Obtenemos entonces el siguiente problema:
\begin{alignat*}{2}
  & \text{minimizar: } & & \inner{\cb}{\xb} + \inner{\yb}{(\bb - \Ab\xb)}, \\
   & \text{sujeto a: } & \quad & \xb \ge \cero.
\end{alignat*}



\end{frame}

%------------------------------------------------------------------

\begin{frame}
\frametitle{Motivación}

Llamamos $g(\yb)$ al costo óptimo en función de $\yb$ del problema relajado. Como tenemos más libertad en la elección de variables en el problema relajado, esperamos que $g(\yb)$ sea menor o igual que el costo óptimo $\inner{\cb}{\xb^\star}$ del problema primal. En efecto,
$$
g(\yb) = \min_{\xb \ge 0} \left\{ \inner{\cb}{\xb} + \inner{\yb}{(\bb-\Ab\xb)}\right\} \le \inner{\cb}{\xb^\star} + \inner{\yb}{(\bb - \Ab\xb^\star)} = \inner{\cb}{\xb^\star},
$$
donde usamos para la última desigualdad que $\xb^\star$ es una solución óptima del problema primal y por lo tanto $\Ab\xb^\star = \bb$.

\end{frame}

%------------------------------------------------------------------

\begin{frame}
\frametitle{Motivación}

Por lo tanto, para cualquier vector $\yb$,
$$g(\yb) \le \inner{\cb}{\xb^\star}$$
y el valor de $g(\yb)$ es una cota inferior del costo óptimo $\inner{\cb}{\xb^\star}$ del problema primal. Queremos hallar la mejor cota posible, es decir, queremos hallar el vector $\yb$ que maximice $g(\yb)$. Este problema se conoce como el \emph{problema dual}, que podemos plantear en la forma:
\begin{alignat*}{2}
  & \text{maximizar: } & & \min_{\xb \ge 0} \left\{ \inner{\cb}{\xb} + \inner{\yb}{(\bb-\Ab\xb)}\right\}, \\
   & \text{sujeto a: } & \quad & \text{ninguna restricción}.
\end{alignat*}
donde $\yb$ es la variable de optimización.

\end{frame}

%------------------------------------------------------------------

\begin{frame}
\frametitle{Motivación}

El resultado principal de la teoría de dualidad afirma que el costo óptimo del problema dual es igual al costo óptimo $\inner{\cb}{\xb^\star}$ del problema primal.

En otras palabras, si elegimos el valor de $\yb$ para el cual $g(\yb)$ es máximo, la posibilidad de violar las restricciones $ \Ab\xb = \bb$ no es de ninguna ayuda.


\end{frame}

%------------------------------------------------------------------

\begin{frame}
\frametitle{Motivación}

Por la definición de $g(\yb)$,
$$
\begin{aligned}
g(\yb) = \min_{\xb \ge 0} \left\{ \inner{\cb}{\xb} + \inner{\yb}{(\bb-\Ab\xb)}\right\}
&= \inner{\yb}{\bb} + \min_{\xb \ge 0} \left\{ \inner{\cb}{\xb} - \inner{\yb}{(\Ab\xb)} \right\} \\
&= \inner{\yb}{\bb} + \min_{\xb \ge 0} \left\{ \inner{\cb}{\xb} - \inner{(\Ab^T\yb)}{\xb} \right\} \\
&= \inner{\yb}{\bb} + \min_{\xb \ge 0} \left\{ \inner{(\cb - \Ab^T\yb)}{\xb} \right\},
\end{aligned}
$$
recordando que $\inner{\xb}{(\Ab\yb)} = \xb^T \Ab \yb = (\Ab^T \xb)^T \yb = \inner{(\Ab^T \xb)}{\yb}$.


\end{frame}

%------------------------------------------------------------------

\begin{frame}
\frametitle{Motivación}

Ahora bien, notemos que como nos restringimos a los vectores $\xb \ge \cero$,
$$\min_{\xb \ge 0} \left\{ \inner{(\cb - \Ab^T\yb)}{\xb} \right\} = \begin{cases}
0 & \text{si } \cb - \Ab^T\yb \ge \cero, \\
-\infty, & \text{en otro caso.}\end{cases}
$$

Como queremos maximizar $g(\yb)$, podemos quedarnos solo con los casos para los que $g(\yb) \neq -\infty$. Concluimos que el problema dual es equivalente al problema
\begin{alignat*}{2}
  & \text{maximizar: }  & \quad & \inner{\yb}{\bb},   \\
   & \text{sujeto a: }  &       & \Ab^T\yb \le \cb. \\
\end{alignat*}


\end{frame}

%------------------------------------------------------------------

\begin{frame}
\frametitle{Motivación}

En resumen:
\begin{itemize}
\item Definimos un vector $\yb$ de parámetros o variables duales, y para cada elección de $\yb$ podemos encontrar una cota inferior para el costo óptimo del problema primal.
\item El problema dual consiste en maximizar esa cota, es decir, obtener la cota más precisa posible.
\item Para algunos vectores $\yb$, la cota que obtenemos es $-\infty$, que no tiene utilidad.
\item Por lo tanto, nos restringimos a las elecciones de $\yb$ que nos dan cotas no triviales, y eso nos da las restricciones del problema dual.
\end{itemize}

\end{frame}

%------------------------------------------------------------------

\begin{frame}
\frametitle{Dualidad débil}

A partir del razonamiento anterior, deducimos el siguiente resultado

\begin{theorem}[dualidad débil]
Si $\xb$ es una solución factible del problema primal e $\yb$ es una solución factible del problema dual, entonces
$$
\inner{\yb}{\bb} \le \inner{\cb}{\xb}.
$$
\end{theorem}

Si bien este resultado es muy simple, tiene consecuencias importantes.

\begin{corollary}
\begin{itemize}
\item Si el costo óptimo del problema primal es $-\infty$, entonces el problema dual no es factible.
\item Si el costo óptimo del problema dual es $+\infty$, entonces el problema primal no es factible.
\end{itemize}
\end{corollary}

\end{frame}

%------------------------------------------------------------------

\begin{frame}
\frametitle{Dualidad débil}

Obtenemos también un criterio para determinar la optimalidad de las soluciones.

\begin{corollary}
Sean $\xb^\star \in \R^n$ e $\yb^\star \in \R^m$ soluciones factibles de los problemas primal y dual respectivamente, y supongamos que $$\inner{\yb^\star}{\bb} = \inner{\cb}{\xb^\star}.$$

Entonces $\xb^{\star}$ e $\yb^\star$ son soluciones óptimas de los problemas primal y dual respectivamente.
\end{corollary}

\end{frame}

%------------------------------------------------------------------

\begin{frame}
\frametitle{Dualidad fuerte}
Enunciamos el resultado principal de la teoría dual de programación lineal.

\begin{theorem}[dualidad fuerte]
Si un problema primal de programación lineal tiene solución óptima, entonces también el problema dual tiene solución óptima, y los respectivos costos óptimos son iguales.
\end{theorem}

\textbf{Demostración.}
Para una solución óptima $\xb$, el vector de costos reducidos $\cb^T - \cb_B^T \Bb^{-1} \Ab$ satisface
$$\cb^T - \cb_B^T \Bb^{-1} \Ab \ge 0.$$

%($\pb^T = \cb_B^T \Bb^{-1}$)
Definimos ahora $\yb \in \R^m$ tal que $\yb^T = \cb_B^T \Bb^{-1}$. Vemos que $\yb^T \Ab  \le \cb^T$, y por lo tanto, $\yb$ es una solución factible del problema dual.

Ver demostración completa en el apunte.

\end{frame}

%------------------------------------------------------------------

\begin{frame}
\frametitle{Dualidad en programación semidefinida}

Estudiamos ahora la teoría de dualidad en programación semidefinida.

Definimos un problema en forma primal estándar como
\begin{alignat}{2}
  & \text{minimizar: } & \quad & \innerTrace{\Cb}{\Xb} \nonumber \\
  & \text{sujeto a: }  \quad & & \innerTrace{\Ab_i}{\Xb} = b_i, \quad i = 1, \dots, m, \tag{SDP-P}\label{sdp-p}\\
   && & \Xb \succeq 0, \nonumber
\end{alignat}
y asociamos a un problema de esta forma otro problema, llamado \emph{problema dual}, que puede plantearse en la forma
\begin{alignat}{2}
  & \text{maximizar: } &   \quad & \inner{\yb}{\bb} \tag{SDP-D}\label{sdp-d} \\
  & \text{sujeto a: } & \quad & \sum_{i=1}^m y_i  \Ab_i \preceq \Cb. \nonumber
\end{alignat}


\end{frame}

%------------------------------------------------------------------

\begin{frame}
\frametitle{Ejercicio}

Consideramos el problema primal para $\Xb = \begin{pmatrix} x_{11} & x_{21} \\ x_{21} & x_{22} \end{pmatrix}$,
\begin{alignat*}{2}
  & \text{minimizar: }  & \quad & 2x_{11} + 2x_{12}   \\
   & \text{sujeto a: } & \quad & x_{11} + x_{22} = 1, \\
   & & &  \Xb \succeq 0.
\end{alignat*}

Calcular el problema dual asociado.

\end{frame}

%------------------------------------------------------------------

\begin{frame}
\frametitle{Matrices semidefinidas positivas y producto interno traza}


Para adaptar el desarrollo realizado en programación lineal, utilizamos el siguiente resultado.

\begin{theorem}
Una matriz $\Ab$ es semidefinida positiva si y solo si $\innerTrace{\Ab}{\Xb} \ge 0$ para toda matriz $\Xb \in \Splus$.
\label{teo:traza}
\end{theorem}

\textbf{Demostración.}
Si $\Ab \succeq 0$, entonces $\Ab = \Bb^T \Bb$ para alguna matriz $\Bb$ y por lo tanto,
$$
\innerTrace{\Ab}{\Xb} = \tr(\Ab\Xb) = \tr(\Bb^T \Bb \Xb) = \tr(\Bb \Xb \Bb^T).
$$

Como $\Xb \succeq 0$, también $\Bb \Xb \Bb^T \succeq 0$ y por lo tanto $\tr(\Bb \Xb \Bb^T) \ge 0$.

\end{frame}

%------------------------------------------------------------------

\begin{frame}
\frametitle{Matrices semidefinidas positivas y producto interno traza}

Para la otra dirección, si $\xb \in \R^n$, entonces $\xb \xb^T \in \Rnn$ es una matriz semidefinida positiva. Si $\innerTrace{\Ab}{\Xb} \ge 0$ para toda $\Xb$, entonces $\innerTrace{\Ab}{\xb \xb^T}  \ge 0$ para todo $\xb \in \Rn$ y
$$
\innerTrace{\Ab}{\xb\xb^T} = \tr(\Ab \xb \xb^T) = \tr(\xb^T \Ab \xb) = \xb^T \Ab \xb
$$
y por lo tanto $\xb^T \Ab \xb \ge 0$ para todo $\xb$.



\end{frame}

%------------------------------------------------------------------

\begin{frame}
\frametitle{Motivación}

Podemos relajar el problema primal reemplazando las restricciones
$$\innerTrace{\Ab_i}{\Xb} = b_i, \quad i = 1, \dots, m,$$
por una penalidad $$\sum_{i=1}^m y_i (b_i - \innerTrace{\Ab_i}{\Xb}),$$
donde $\yb \in \R^m$ es un vector de costos del mismo tamaño que $\bb$. Obtenemos el siguiente problema relajado:
\begin{alignat*}{2}
  & \text{minimizar: } & & \innerTrace{\Cb}{\Xb} + \sum_{i=1}^m y_i (b_i - \innerTrace{\Ab_i}{\Xb}), \\
   & \text{sujeto a: } & \quad & \Xb \succeq 0.
\end{alignat*}


\end{frame}

%------------------------------------------------------------------

\begin{frame}
\frametitle{Motivación}

Llamamos $g(\yb)$ al costo óptimo en función de $\yb$ del problema relajado. Como tenemos más libertad en la elección de variables en el problema relajado, esperamos que $g(\yb)$ sea menor o igual que el costo óptimo $\innerTrace{\Cb}{\Xb}$ del problema primal.

En efecto, si $\Xb^\star$ es una solución óptima del problema primal (suponiendo que existe tal matriz),
$$
\begin{aligned}
g(\yb) &= \min_{\Xb \succeq 0} \left\{ \innerTrace{\Cb}{\Xb} + \sum_{i=1}^m y_i (b_i - \innerTrace{\Ab_i}{\Xb}) \right\} \\
&\le \innerTrace{\Cb}{\Xb^\star} + \sum_{i=1}^m y_i (b_i - \innerTrace{\Ab_i}{\Xb^\star}) = \innerTrace{\Cb}{\Xb^\star},
\end{aligned}
$$
donde usamos para la última desigualdad que $\Xb^\star$ es una solución óptima del problema primal y por lo tanto $\innerTrace{\Ab_i}{\Xb^\star} = b_i$ para todo $1 \le i \le m$.

\end{frame}

%------------------------------------------------------------------

\begin{frame}
\frametitle{Motivación}

Por lo tanto, para cualquier vector $\yb$,
$$g(\yb) \le \innerTrace{\Cb}{\Xb^\star}$$
y el valor de $g(\yb)$ es una cota inferior del costo óptimo $\inner{\Cb}{\Xb^\star}$ del problema primal. Queremos hallar la mejor cota posible, es decir, queremos hallar el vector $\yb$ que maximice $g(\yb)$. Este problema se conoce como el \emph{problema dual}, que podemos plantear en la forma:
\begin{alignat*}{2}
  & \text{maximizar: } & & \min_{\Xb \succeq 0} \left\{ \innerTrace{\Cb}{\Xb} + \sum_{i=1}^m y_i (b_i - \innerTrace{\Ab_i}{\Xb}) \right\} \\
   & \text{sujeto a: } & \quad & \text{ninguna restricción},
\end{alignat*}
donde $\yb = (y_1, \dots, y_m)$ es la variable de optimización.
\end{frame}

%------------------------------------------------------------------

\begin{frame}
\frametitle{Motivación}

Por la definición de $g(\yb)$,
$$
\begin{aligned}
g(\yb) &= \min_{\Xb \succeq 0} \left\{ \innerTrace{\Cb}{\Xb} + \sum_{i=1}^m y_i (b_i - \innerTrace{\Ab_i}{\Xb})\right\} \\
&= \sum_{i=1}^m y_i b_i + \min_{\Xb \succeq 0} \left\{ \innerTrace{\Cb}{\Xb} - \sum_{i=1}^m y_i (\innerTrace{\Ab_i}{\Xb}) \right\} \\
&= \inner{\yb}{\bb} + \min_{\Xb \succeq 0} \left\{ \innerTrace{\left(\Cb - \sum_{i=1}^m y_i \Ab_i\right)}{\Xb} \right\}. \\
\end{aligned}
$$

\end{frame}

%------------------------------------------------------------------

\begin{frame}
\frametitle{Motivación}

Ahora bien, por la propiedad del producto interno traza, si $\Cb - \sum_{i=1}^m y_i \Ab_i \succeq 0$,
$$\innerTrace{\left(\Cb - \sum_{i=1}^m y_i \Ab_i\right)}{\Xb} \ge 0$$
para toda $\Xb \succeq 0$ y vale 0 si $\Xb = \cero$.

Recíprocamente, si $\Cb - \sum_{i=1}^m y_i \Ab_i \not\succeq 0$, existe $\Xb \succeq 0$ tal que
$$\innerTrace{(\Cb - \sum_{i=1}^m y_i \Ab_i)}{\Xb} < 0,$$
y tomando múltiplos de $\Xb$ podemos hacer este valor tan chico como querramos. Obtenemos
$$\min_{\Xb \succeq 0} \left\{ \innerTrace{\left(\Cb - \sum_{i=1}^m y_i \Ab_i\right)}{\Xb} \right\} = \begin{cases}
0 & \text{si } \Cb - \sum_{i=1}^m y_i \Ab_i \succeq 0, \\
-\infty, & \text{en otro caso.}\end{cases}
$$

\end{frame}

%------------------------------------------------------------------

\begin{frame}
\frametitle{Motivación}

Como queremos maximizar $g(\yb)$, podemos quedarnos solo con los casos para los que $g(\yb) \neq -\infty$, que corresponden a $\Cb - \sum_{i=1}^m y_i \succeq 0$.

Concluimos que el problema dual es equivalente al problema
\begin{alignat*}{2}
  & \text{maximizar: }  & \quad & \inner{\yb}{\bb},   \\
   & \text{sujeto a: } & \quad & \sum_{i=1}^m y_i \Ab_i  \preceq \Cb,
\end{alignat*}
como queríamos.

\end{frame}

%------------------------------------------------------------------

\begin{frame}
\frametitle{Dualidad débil}

Podemos sintetizar lo que acabamos de ver en el siguiente teorema.

\begin{theorem}[dualidad débil]
  Dadas $\Xb$ e $\yb$ dos soluciones factibles del problema primal y el problema dual respectivamente,
  $$\innerTrace{\Cb}{\Xb} - \inner{\yb}{\bb} \ge 0.$$
\end{theorem}

\textbf{Demostración}
Tenemos
  $$\innerTrace{\Cb}{\Xb} - \inner{\yb}{\bb} = \innerTrace{\Cb}{\Xb} -  \sum_{i=1}^m y_i \left(\innerTrace{\Ab_i}{\Xb}\right) =
   \innerTrace{\left(\Cb - \sum_{i=1}^m y_i \Ab_i\right)}{\Xb} \ge 0,
   $$
  por la propiedad del producto interno traza.


\end{frame}

%------------------------------------------------------------------

\begin{frame}
\frametitle{Condición de holgura complementaria}

Como en el caso de programación lineal, tenemos el siguiente importante corolario inmediato.

\begin{corollary}
Sean $\Xb$ e $\yb$ soluciones factibles de los problemas primal y dual respectivamente, y supongamos que $\inner{\yb}{\bb} = \innerTrace{\Cb}{\Xb}$. Entonces $\Xb$ e $\yb$ son soluciones óptimas de los problemas primal y dual respectivamente.
\end{corollary}

La condición $\inner{\yb}{\bb} = \innerTrace{\Cb}{\Xb}$ es equivalente a la condición
$$
\innerTrace{\left(\Cb - \sum_{i=1}^m y_i \Ab_i\right)}{\Xb} = 0
$$
que llamamos la \emph{condición de holgura complementaria}.

\end{frame}

%------------------------------------------------------------------

\begin{frame}
\frametitle{Ejemplo 1}

Veamos algunos ejemplos de la relación que puede darse entre el problema primal y dual.

\textbf{Ejemplo.} Consideramos el problema primal para $\Xb = \begin{pmatrix} x_{11} & x_{21} \\ x_{21} & x_{22} \end{pmatrix}$,
\begin{alignat*}{2}
  & \text{minimizar: }  & \quad & 2x_{11} + 2x_{12}   \\
   & \text{sujeto a: } & \quad & x_{11} + x_{22} = 1, \\
   & & &  \Xb \succeq 0.
\end{alignat*}

Despejando $x_{22} = 1 - x_{11}$, las restricciones que deben cumplirse son:
$$x_{11} \ge 0, \quad x_{11} \le 1 \quad \text{ y } \quad x_{11}(1-x_{11}) \ge x_{12}^2,
$$
cuyo gráfico es un disco de radio $1/2$ centrado en el punto $(1/2, 0)$.

\end{frame}

%------------------------------------------------------------------

\begin{frame}
\frametitle{Ejemplo 1 (cont.)}

La solución óptima es
$$
\Xb^\star = \begin{pmatrix}
\frac{2 - \sqrt{2}}{4} & -\frac{\sqrt{2}}{4} \\
-\frac{\sqrt{2}}{4} & \frac{2 + \sqrt{2}}{4}
\end{pmatrix}
$$
y el costo óptimo es $1 - \sqrt{2}$.

Para obtener el problema dual definimos
$$\Cb = \begin{pmatrix}
2 & 1 \\ 1 & 0
\end{pmatrix}, \quad \Ab_1 = \begin{pmatrix}
1 & 0 \\ 0 & 1
\end{pmatrix}\quad \text{ y } \quad b_1 = 1 \ \text{($m = 1$)}.$$



\end{frame}

%------------------------------------------------------------------

\begin{frame}
\frametitle{Ejemplo 1 (cont.)}

El problema dual es
\begin{alignat*}{2}
  & \text{maximizar: }  & \quad & y_1   \\
   & \text{sujeto a: } & \quad & y_1 \Ab_1 \preceq \Cb.
\end{alignat*}
o equivalentemente
\begin{alignat*}{2}
  & \text{maximizar: }  & \quad & y   \\
   & \text{sujeto a: } & \quad & \begin{pmatrix}
2 - y & 1 \\ 1 & -y \end{pmatrix} \succeq 0.
\end{alignat*}

\end{frame}

%------------------------------------------------------------------

\begin{frame}
\frametitle{Ejemplo 1 (cont.)}

En este caso tenemos las restricciones
$$y \le 2, \quad y \le 0 \quad \text{ y } \quad -(2-y)y - 1 \ge 0.$$

Completando cuadrados obtenemos
$$
(y-1)^2 \ge 2,
$$
y el máximo valor posible de $y < 0$ es $y^\star = 1 - \sqrt{2}$.

Vemos que los valores óptimos de ambos problemas coinciden y podemos verificar que se cumple la condición de holgura complementaria
$$
\innerTrace{(\Cb - y^\star \Ab_1 )}{\Xb^\star} = \innerTrace{
\begin{pmatrix}
1 + \sqrt{2} & 1 \\ 1 & \sqrt{2} - 1
\end{pmatrix}}
{\begin{pmatrix}
\frac{2 - \sqrt{2}}{4} & -\frac{\sqrt{2}}{4} \\
-\frac{\sqrt{2}}{4} & \frac{2 + \sqrt{2}}{4}
\end{pmatrix}} = 0.
$$

\end{frame}

%------------------------------------------------------------------

\begin{frame}
\frametitle{Ejemplo: salto de dualidad positivo}

Veamos a continuación otro ejemplo en el cual ambos problemas son factibles y sin embargo los valores óptimos son diferentes.

\textbf{Ejemplo.}
Para $\Xb \in \R^{3 \times 3}$ y $\alpha \ge 0$ dado, consideramos el siguiente par de problemas primal-dual

\begin{multicols}{2}\noindent
\begin{alignat*}{2}
  & \text{minimizar: }  & \quad & \alpha x_{11}   \\
   & \text{sujeto a: } & \quad & x_{22} = 0, \\
   &&& x_{11} + 2 x_{23} = 1, \\
   &&& \Xb \succeq 0.
\end{alignat*}
\begin{alignat*}{2}
  & \text{maximizar: }  & \quad & y_2   \\
   & \text{sujeto a: } & \quad & \begin{pmatrix}
\alpha - y_2 & 0 & 0 \\ 0 & -y_1 & -y_2 \\ 0 & -y_2 & 0 \end{pmatrix} \succeq 0.
\end{alignat*}
\end{multicols}


\end{frame}

%------------------------------------------------------------------

\begin{frame}
\frametitle{Ejemplo: salto de dualidad positivo}

Para el problema primal, como $x_{22}=0$ y $\Xb \succeq 0$, todas las coordenadas de $\Xb$ en la segunda fila y segunda columna deben ser 0.

Por lo tanto,
$$0 = x_{23} = \frac{1 - x_{11}}{2}$$ y obtenemos $x_{11} = 1$.

La única solución factible del problema es
$$
\Xb = \begin{pmatrix}
1 & 0 & 0 \\ 0 & 0 & 0 \\ 0 & 0 & 0 \end{pmatrix}$$
y el costo óptimo es $\alpha$.

\end{frame}

%------------------------------------------------------------------

\begin{frame}
\frametitle{Ejemplo: salto de dualidad positivo}

Para el problema dual, debe ser $y_2 = 0$ y tomando cualquier $y_1 \ge 0$ obtenemos una solución factible.

El costo óptimo es por lo tanto 0 y el salto de dualidad es
$$p^\star - d^\star = \alpha - 0 = \alpha.$$


\end{frame}

%------------------------------------------------------------------

\begin{frame}
\frametitle{Ejemplo: problema sin solución óptima}

Analizamos ahora otra situación que puede darse en la que si bien los costos óptimos coinciden, los problemas pueden no tener solución óptima.

\textbf{Ejemplo.}
Para $\Xb \in \R^{2 \times 2}$, consideramos el siguiente par de problemas primal-dual
\begin{multicols}{2}\noindent
\begin{alignat*}{2}
  & \text{minimizar: }  & \quad & x_{11}   \\
   & \text{sujeto a: } & \quad & x_{21} = 1, \\
   &&& \Xb \succeq 0.
\end{alignat*}
\begin{alignat*}{2}
  & \text{maximizar: }  & \quad & y   \\
   & \text{sujeto a: } & \quad & \begin{pmatrix}
 1 & -\frac{y}{2}  \\ -\frac{y}{2} & 0 \end{pmatrix} \succeq 0.
\end{alignat*}
\end{multicols}


\end{frame}

%------------------------------------------------------------------

\begin{frame}
\frametitle{Ejemplo: problema sin solución óptima}

\textbf{SDP-P.} Para el problema primal, la restricción
$\begin{pmatrix} x_{11} & 1 \\ 1 & x_{22}\end{pmatrix} \succeq 0$
se cumple si
$$x_{11} \ge 0, \quad x_{22} \ge 0, \quad x_{11}x_{22} \ge 1.$$
El costo óptimo es $p^\star = 0$ pero no existe ningún valor de $(x_{11}, x_{22})$ para el cuál se alcance el ínfimo.

\textbf{SDP-D.} Para el problema dual, debe ser $\frac{y}{2} = 0$.

Por lo tanto, el costo óptimo es $d^\star = 0$ que se alcanza para la solución factible $y = 0$.

Vemos que si bien el salto de dualidad $p^\star - d^\star = 0$, no existen $\Xb$ e $\yb$ que cumplan la condición de holgura complementaria.

\end{frame}

%------------------------------------------------------------------

\begin{frame}
\frametitle{Problemas estrictamente factibles}

En cierto sentido, estos últimos ejemplos son patológicos. Bajo condiciones generales la dualidad fuerte también se cumple en programación semidefinida.

Una de tales condiciones es requerir que los problemas sean \emph{estrictamente factibles}.

\begin{itemize}
\item En el problema primal, esto significa que existe $\Xb \succ 0$ que verifica las restricciones.
\item En el problema dual, esto significa que existe $\yb \in \R^m$ tal que $\Cb - \sum_{i=1}^m y_i \Ab_i \succ 0$.
\end{itemize}



\end{frame}

%------------------------------------------------------------------

\begin{frame}
\frametitle{Dualidad fuerte}

Si los problemas son estrictamente factibles, la situación es tan buena como en programación lineal.

\begin{theorem}[dualidad fuerte] Suponemos que tanto el problema primal $\eqref{sdp-p}$ como el problema dual $\eqref{sdp-d}$ son estrictamente factibles. Entonces ambos problemas tienen solución óptima, y el costo óptimo coincide. Es decir, no hay salto de dualidad.
\end{theorem}

\textbf{Demostración.} Ver \cite[Teorema 4.1]{Todd2001}.



\end{frame}

%------------------------------------------------------------------

\begin{frame}

\frametitle{Referencias}

\bibliographystyle{apalike}
\bibliography{../../latex/optimizacion}

\end{frame}

%------------------------------------------------------------------

\end{document} 