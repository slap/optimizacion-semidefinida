\documentclass[aspectratio=169,12pt,spanish]{beamer}
\usepackage[T1]{fontenc}
\usepackage[spanish]{babel}

\usepackage{wrapfig}

%\usepackage{multicol}
%\usepackage{mathtools}

\usepackage[normalem]{ulem}

\usepackage{pgf,tikz}
\usetikzlibrary{matrix}
\usetikzlibrary{arrows}

%\usepackage{wrapfig}
\mode<presentation>
\usefonttheme{professionalfonts}
\usetheme{Darmstadt}
\usecolortheme{orchid}
\useoutertheme{default}
\setbeamertemplate{headline}{}

\newcounter{savedenum}
\newcommand*{\saveenum}{\setcounter{savedenum}{\theenumi}}
\newcommand*{\resume}{\setcounter{enumi}{\thesavedenum}}

\renewcommand{\baselinestretch}{1.1}

%gets rid of bottom navigation bars
\setbeamertemplate{footline}[page number]

%gets rid of navigation symbols
\setbeamertemplate{navigation symbols}{}

%\frameframe{none} % No default frame

%\setlength{\framewidth}{8.7in} \setlength{\frameheight}{7.2in}

\parindent 0pt
\setlength{\parskip} {1ex plus 0.5ex minus 0.2ex}


\usepackage[bbgreekl]{mathbbol}
\usepackage{amssymb, amsthm, amsmath}
\usepackage{bm}

\newtheorem{ejercicio}{Ejercicio}
\newtheorem{proposition}[theorem]{Proposición}

\DeclareSymbolFontAlphabet{\mathbb}{AMSb}
\DeclareSymbolFontAlphabet{\mathbbl}{bbold}

\usepackage{multicol}
\usepackage{colortbl}
\usepackage{lmodern}
\usepackage{tabularx}
\usepackage{multirow}
\usepackage{stmaryrd}
\usepackage{color}
\usepackage{graphicx}
\usepackage{hyperref}

\graphicspath{ {../../images} }
\usepackage{listings}
\lstset{
  basicstyle=\ttfamily,
  columns=fullflexible,
}

\usepackage{url}
\usepackage{multicol}
\usepackage{dsfont}

% Bold symbols for vectors and matrices
\newcommand{\xstar}{\bm{x}^{\star}}
\newcommand{\alphab}{\bm{\alpha}}
\newcommand{\ab}{\bm{a}}
\newcommand{\bb}{\bm{b}}
\newcommand{\cb}{\bm{c}}
\newcommand{\db}{\bm{d}}
\newcommand{\eb}{\bm{e}}
\newcommand{\gb}{\bm{g}}
\newcommand{\mb}{\bm{m}}
\newcommand{\pb}{\bm{p}}
\newcommand{\qb}{\bm{q}}
\newcommand{\rb}{\bm{r}}
\newcommand{\ssb}{\bm{s}}
\newcommand{\ub}{\bm{u}}
\newcommand{\vb}{\bm{v}}
\newcommand{\wb}{\bm{w}}
\newcommand{\xb}{\bm{x}}
\newcommand{\yb}{\bm{y}}
\newcommand{\zb}{\bm{z}}

\newcommand{\Ab}{\bm{A}}
\newcommand{\Bb}{\bm{B}}
\newcommand{\Cb}{\bm{C}}
\newcommand{\Db}{\bm{D}}
\newcommand{\Eb}{\bm{E}}
\newcommand{\Fb}{\bm{F}}
\newcommand{\Gb}{\bm{G}}
\newcommand{\Hb}{\bm{H}}
\newcommand{\Ib}{\bm{I}}
\newcommand{\Id}{\bm{I}}
\newcommand{\Kb}{\bm{K}}
\newcommand{\Lb}{\bm{L}}
\newcommand{\Mb}{\bm{M}}
\newcommand{\Pb}{\bm{P}}
\newcommand{\Qb}{\bm{Q}}
\newcommand{\Rb}{\bm{R}}
\newcommand{\Sb}{\bm{S}}
\newcommand{\Tb}{\bm{T}}
\newcommand{\Ub}{\bm{U}}
\newcommand{\Vb}{\bm{V}}
\newcommand{\Wb}{\bm{W}}
\newcommand{\Xb}{\bm{X}}
\newcommand{\Yb}{\bm{Y}}
\newcommand{\Zb}{\bm{Z}}
\newcommand{\Lambdab}{\bm{\Lambda}}
\newcommand{\cero}{\bm{0}}

% Rings and fields
\newcommand{\A}{\mathbb{A}}
\newcommand{\Z}{\mathbb{Z}}
\newcommand{\Q}{\mathbb{Q}}
\newcommand{\C}{\mathbb{C}}
\newcommand{\R}{\mathbb{R}}
\newcommand{\K}{\mathbb{K}}
\newcommand{\N}{\mathbb{N}}

\newcommand{\borel}{{\mathcal B}}
\newcommand{\pmom}{{\rho_{\text{mom}}}}
\newcommand{\MX}{{\mathcal{M}(X)}}


% Inner product
\newcommand{\innerl}[2]{\langle #1, #2 \rangle}
\newcommand{\inner}[2]{#1 \boldsymbol{\cdot} #2}
\newcommand{\innerTrace}[2]{#1 \bullet #2}

% Symmetric and positive definite matrices
\newcommand{\Splusplusn}{{\mathcal S_{++}^n}}
\newcommand{\Splusn}{{\mathcal S_+^n}}
\newcommand{\Splus}{{\mathcal S_+}}
\newcommand{\Sym}{{\mathcal S}}
\newcommand{\Symn}{{\mathcal S^n}}

% Cones
\newcommand\CC{\mathcal{C}}
\DeclareMathOperator{\cone}{cono}
\DeclareMathOperator{\conv}{conv}
\DeclareMathOperator{\supp}{supp}


% Spectrahedron
\newcommand{\eLL}{{\mathcal L}}

% Matrices and vectors over R or C
\newcommand{\Rnn}{\R^{n\times n}}
\newcommand{\Cnn}{\C^{n\times n}}
\newcommand{\Rn}{\R^{n}}
\newcommand{\Rm}{\R^{m}}


% Math operators
\DeclareMathOperator{\Tr}{Tr}
\DeclareMathOperator{\tr}{Tr}
\DeclareMathOperator{\interior}{int}
\DeclareMathOperator{\rank}{rank}
\DeclareMathOperator{\diag}{diag}

\newcommand\one{\mathds{1}} 

\pagestyle{empty}

\begin{document}

%------------------------------------------------------------------

\begin{frame}

 \begin{center}

\Large\textbf{Optimización Semidefinida} \\
\large\textbf{Clase 12 - Polinomios positivos y sumas de cuadrados}
%\vspace{0.5cm}

% \textit{Santiago Laplagne} \\
%slaplagn@dm.uba.ar \\


%\vspace{0.5cm}
%{\small Trabajo en progreso en conjunto con \emph{Jose Capco} (Universit\"at Innsbruck) y \emph{Claus Scheiderer} %(Universit\"at Konstanz).} \\

\vspace{1cm}
 Segundo Cuatrimestre 2021
 \\
 {\small Facultad de Ciencias Exactas y Naturales, UBA}
 \end{center}

\end{frame}


%------------------------------------------------------------------

\begin{frame}

\frametitle{Polinomios positivos y sumas de cuadrados}

Notamos $\R[\xb] = \R[x_1, \dots, x_n]$ al anillo de polinomios sobre $\R$ en $n$ variables.

Dado un polinomio $p(\xb) \in \R[\xb]$, decimos que
\begin{itemize}
\item $p$ es \emph{positivo} ($p \ge 0$) si $p(\xb) \ge 0$ para todo $\xb \in \R^n$ (y \emph{estrictamente positivo} si $p > 0$ para todo $\xb \in \R^n$).
\item $p$ es una \emph{suma de cuadrados} (SOS) si existen $q_1, \dots, q_s \in \R[\xb]$ tales que
$$
p = q_1^2 + \dots + q_s^2.
$$
\end{itemize}

\end{frame}

%------------------------------------------------------------------

\begin{frame}

\frametitle{Certificados de positividad}

\textbf{Propiedad:} Si $p$ es SOS entonces $p \ge 0$.

\textbf{Problema:} Dado un polinomio $p(\xb) \in \R[\xb]$, determinar si se puede escribir como suma de cuadrados (SOS) $p = p_1^2 + \dots + p_s^2$ y construir la descomposici\'on (aproximada o exacta).


{\bf Aplicaciones}

\begin{itemize}
\item Certificado de positividad.
\item La ecuación $p(\xb) + 1 = 0$ no tiene soluciones reales si $p$ es una suma de cuadrados.
\item Dado un polinomio $p$, si queremos hallar el mínimo de $p$ en $S = \R^n$ o en una regi\'on $S = \{ \xb \in \R^n : g(\xb) \ge 0\}$, para polinomios $\{g_1, \dots, g_s\}$, tenemos
\[
\min\{f(x) : \xb \in S\} = \sup\{a \in \R | f - a \ge 0 \text{ en } S\}.
\]

\end{itemize}


\end{frame}

%------------------------------------------------------------------

\begin{frame}

\frametitle{Sumas de cuadrados en una variable}

\begin{proposition}
Si $p \in \R[x]$ (polinomios en una variable) entonces
$$
p \text{ es suma de cuadrados } \iff p \text{ es positivo}
$$
\end{proposition}


\textbf{Demostraci\'on.}

Tomamos $p \ge 0$. Por el teorema fundamental del \'algebra, podemos factorizar
\[
p(x) = \prod_{i=1}^r (x-a_i) \prod_{j=1}^t (x-b_j)(x - \bar b_j),
\]
donde $a_i \in \R$ son las raíces reales y $b_j, \bar b_j \in \C \smallsetminus \R$ son las raíces complejas conjugadas.

\end{frame}

%------------------------------------------------------------------

\begin{frame}

\frametitle{Demostración (cont.)}
Si la multiplicidad de alguna raíz real $a$ es impar, entonces $p$ atraviesa transversalmente al eje $X$ en $a$ y por lo tanto no puede ser $p \ge 0$.

Por lo tanto, todas las raíces reales aparecen con multiplicidad par y podemos factorizar
\[
p(x) = \prod_{i=1}^s (x-a_i)^{2k_i} \prod_{j=1}^t (x-b_j)(x - \bar b_j).
\]

\end{frame}

%------------------------------------------------------------------

\begin{frame}

\frametitle{Demostración (cont.)}
Para las raíces complejas tenemos
\begin{align*}
(x-b_j)(x - \bar b_j) &= (x - (\alpha_i + I \beta_i  ))(x - (\alpha_i - I  \beta_i )) \\
&= ((x - \alpha_i) - I\beta_i  ))((x - \alpha_i) + I \beta_i  )) \\
&= (x - \alpha_i)^2 + \beta_i^2,
\end{align*}
que es una suma de cuadrados.

Concluimos que $p(x)$ es un producto de sumas de cuadrados, y por lo tanto, distribuyendo los productos, $p(x)$ es una suma de cuadrados.

Más aún, utilizando la identidad
\[
(a^2+b^2)(c^2+d^2) = (ac+bd)^2 + (ad-bc)^2
\]
podemos escribir a cualquier polinomio $p(x) \ge 0$ como suma de 2 cuadrados.

\end{frame}

%------------------------------------------------------------------

\begin{frame}

\frametitle{Sumas de cuadrados en varias variables}

\begin{itemize}
\item El polinomio
\[
f(x, y) = x^4y^2 + x^2y^4 - 3x^2y^2+1
 \]
 es no-negativo en $\R^2$ pero no puede escribirse como suma de cuadrados. (Motzkin, 1967)
\end{itemize}

\textbf{Demostración.}

\begin{itemize}
\item \textbf{$f \ge 0$:} Por desigualdad aritmética-geométrica,
$$
\frac{x^4y^2 + x^2y^4 + 1}{3} \ge \sqrt[3]{(x^4y^2)(x^2y^4) 1} = \sqrt[3]{x^6y^6} = x^2y^2,
$$
y despejando obtenemos $x^4y^2 + x^2y^4 - 3x^2y^2+1 \ge 0$
\end{itemize}

\end{frame}

%------------------------------------------------------------------

\begin{frame}

\frametitle{El polinomio de Motzkin no es una suma de cuadrados}
Escribimos $f = p_1^2 + \dots + p_s^2$, $p_i \in \R[x, y, z_1, ...,z_m]$.

\begin{itemize}
\item Podemos evaluar $z_i=0$ para todo $i$ y obtenemos una descomposición en $\R[x, y]$.
\item Si tomamos un monomio $m$ en los $p_i$ con el mayor grado $d$ en $x$, el monomio $m^2$ no se va a cancelar en la suma, y por lo tanto debe ser un monomio de $f$.
\item Por lo tanto, debe ser $d \le 2$.
\item Luego, siguiendo el mismo razonamiento, no puede aparecer $x^2y^2$ en ningún $p_i$.
\item Luego, tampoco pueden aparecer $x^2$ ni $y^2$ en ningún $p_i$.
\item Finalmente, no puede aparecer $x$ ni $y$ en ningún $p_i$.
\end{itemize}

\end{frame}

%------------------------------------------------------------------

\begin{frame}

\frametitle{El polinomio de Motzkin no es una suma de cuadrados}

\begin{itemize}
\item Concluimos que los polinomios $p_i$ son de la forma
$$
a x^2y + b x y^2 + c xy + d,
$$
con $a, b, c, d \in \R$.
\item Al elevar al cuadrado un polinomio de esta forma, el coeficiente de $x^2y^2$ es siempre no-negativo. ¡Absurdo!
\end{itemize}

\end{frame}

%------------------------------------------------------------------


\begin{frame}
\frametitle{Vector de exponentes}

Dado un vector $\ab \in \N_0^n$, $\ab = (a_1, \dots, a_n)$, definimos el monomio $m = \xb^{\ab}$ como
$$
\xb^{\ab} = x_1^{a_1} x_2^{a_2} \cdots x_n^{a_n},
$$
y análogamente, para un monomio $m$ de esa forma, llamamos a $\ab = (a_1, \dots, a_n) \in \R^n$ su vector de exponentes.


\end{frame}

%------------------------------------------------------------------

\begin{frame}
\frametitle{El polítopo de Newton}

Dado un polinomio $p \in \K[\xb]$, definimos el \emph{soporte} de $p$, $\supp(p)$, como el conjunto de todos los vectores de exponentes de los monomios que aparecen en $p$.

Definimos  su \emph{polítopo de Newton} $\mathcal{N}(p)$ como la cápsula convexa de los vectores de exponentes de los monomios que aparecen en $p$,
$$
\mathcal{N}(p) = \conv(\supp(p)).
$$

Por ejemplo, si $p = x_1 x_2^2 + x_2^2 + x_1 x_2 x_3$ entonces 
$$
\mathcal{N}(p) = \conv{(\{(1,2,0), (0,2,0), (1,1,1)\})},
$$
que es un triángulo en $\R^3$.
\end{frame}

%------------------------------------------------------------------

\begin{frame}
\frametitle{El polítopo de Newton y las sumas de cuadrados}


\begin{theorem}
Si $p = \sum_{i = 1}^s q_i^2$ es una suma de cuadrados, entonces
$$
\mathcal{N}(q_i) \subset \frac{1}{2}\mathcal{N}(p).
$$
\end{theorem}

\textbf{Demostración.}
Consideramos la cápsula convexa de la unión de todos los polítopos de Newton de los $q_i$, $1 \le i \le s$,
$$
K = \conv(\cup_{i=1}^s\mathcal{N}(q_i)).
$$

Recordamos que un polítopo está generado por las combinaciones convexas de sus vértices.

\end{frame}

%------------------------------------------------------------------

\begin{frame}
\frametitle{Demostración (continuación)}



Tomamos un \emph{vértice} $\vb$ de $K$ y suponemos por contradicción que $2\vb \not\in \supp(p)$.

Si $\alpha \xb^{\vb}$ aparece en $q_i$, entonces $\alpha^2 \xb^{2\vb}$ aparece en $q_i^2$ con coeficiente $\alpha^2 > 0$. Para que estos términos se cancelen, debe aparecer también $\xb^{2\vb}$ como producto cruzado de términos de algunos $q_i$.
\end{frame}

%------------------------------------------------------------------

\begin{frame}
\frametitle{Demostración (continuación)}


Es decir, existen $\ub, \wb \in K$ tales que $2\vb = \ub+\wb$.

Pero luego, $\vb = \frac{\ub+\wb}{2} \in K$ no es un vértice, lo que contradice la hipótesis.

Concluimos que para cualquier vértice $\vb$ de $K$, $2\vb \in \supp(p) \subset \mathcal{N}(p)$. 

Como $2K$ es un polítopo, es la cápsula convexa del conjunto de todos sus vértices. 

Como todos los vértices de $2K$ están contenidos en el conjunto convexo $\mathcal{N}(p)$, obtenemos $2K \subset \mathcal{N}(p)$. 

Por lo tanto, $2 \mathcal{N}(q_i) \subset \mathcal{N}(p)$ para todo $1 \le i \le n$.


\end{frame}

%------------------------------------------------------------------

\begin{frame}
\frametitle{Motzkin revisitado}

Utilizando este resultado podemos simplificar la demostración de que el polinomio de Motzkin $f(x, y) = x^4y^2 + x^2y^4 - 3x^2y^2+1$ no es una suma de cuadrados.

A la izquierda representamos el polítopo de Newton $\mathcal{N}(f)$ y a la derecha $\frac{1}{2}\mathcal{N}(f)$.

\begin{figure}
    \centering
    \begin{minipage}{0.45\textwidth}
        \centering
        \includegraphics[width=0.5\textwidth]{sos_motzkinHull-2.jpg} % first figure itself
        %\caption{first figure}
    \end{minipage}\hfill
    \begin{minipage}{0.45\textwidth}
        \centering
        \includegraphics[width=0.5\textwidth]{sos_motzkinHull-3.jpg} % second figure itself
        %\caption{second figure}
    \end{minipage}
\end{figure}

Concluimos inmediatamente que $f = \sum_{i=1}^s q_i$, los $q_i$ son de la forma
$$
q_i(x, y) = a x^2y + b x y^2 + c xy + d.
$$


\end{frame}

%------------------------------------------------------------------

\begin{frame}
\frametitle{Grado de un polinomio}

Definimos el grado de un monomio $x_1^{a_1} \cdots x_n^{a_n}$ como
$$d = |\ab| = a_1 + \dots + a_n,$$
y el grado de un polinomio $p \in \R[\xb]$ como el mayor de los grados de sus monomios.


\end{frame}

%------------------------------------------------------------------

\begin{frame}
\frametitle{Polinomios homogéneos}

Decimos que un polinomios es homogéneo si todos sus monomios tienen el mismo grado.

Dado un polinomio no-homogéneo $f(x_1, \dots, x_n)$ de grado $d$, definimos su homogeneización
$$F(x_0, x_1, \dots, x_n) = x_0^d f\left(\frac{x_1}{x_0}, \dots, \frac{x_n}{x_0}\right)$$
que equivale a multiplicar cada monomio por la potencia de $x_0$ apropiada para que todos los términos tengan grado $d$.


Por ejemplo, homegeneizamos $f(x_1, x_2) = x_1^4x_2^2 + x_1^2x_2^4 - 3x_1^2x_2^2+1$ a
$$
F(x_0, x_1, x_2) = x_1^4x_2^2 + x_1^2x_2^4 - 3x_0^2x_1^2x_2^2+x_0^6
$$
multiplicando cada término por la potencia de $x_0$ apropiada.

\end{frame}

%------------------------------------------------------------------


\begin{frame}
\frametitle{Polinomios homogéneos}


Como corolario del teorema anterior sobre los polítopos de Newton, obtenemos el siguiente resultado.

\textbf{Propiedad:} Si $p(\xb)$ es SOS homogéneo, entonces $p(\xb)$ tiene grado par $2d$ y es suma de cuadrados de polinomios homogéneos de grado $d$.

\textbf{Propiedad:} Las condiciones de no-negatividad y suma de cuadrados se mantienen al homogeneizar, por lo tanto no perdemos generalidad al asumir polinomios homogéneos.


Vamos a trabajar a partir de ahora con polinomios homogéneos.

\end{frame}

%------------------------------------------------------------------


\begin{frame}
\frametitle{Los conos de polinomios positivos y sumas de cuadrados}

Llamamos
\begin{itemize}
\item $H_{n,2d}$ al espacio vectorial de polinomios homog\'eneos de $n$ variables y grado $2d$.
\item $P_{n,2d} \subset H_{n,2d}$ al conjunto de polinomios homog\'eneos positivos de $n$ variables y grado $2d$.
\item $\Sigma_{n,2d} \subset P_{n,2d}$ al subconjunto de sumas de cuadrados.
\end{itemize}

\textbf{Propiedad:} $P_{n,2d}$ y $\Sigma_{n,2d}$ son conos convexos cerrados de dimensión máxima.

\textbf{Ejercicio:} Demostrar que $P_{n,2d}$ es cerrado escribiéndolo como intersección de infinitos semi-espacios cerrados.
\end{frame}

%------------------------------------------------------------------


\begin{frame}
\frametitle{Teorema de Hilbert}


\begin{theorem}{Teorema de Hilbert (1888)}
Los conjuntos $P_{n,2d}$ y $\Sigma_{n,2d}$ son iguales solo en los siguientes casos:

\begin{enumerate}
 \item $n = 2$
 \item $2d=2$
 \item $(n,2d) = (3,4)$
\end{enumerate}
\end{theorem}

\end{frame}


%------------------------------------------------------------------


\begin{frame}
\frametitle{Problema de programaci\'on semidefinida}
Dado $p \in \R[x_1, \dots, x_n]$, homog\'eneo de grado $2d$, podemos escribirlo como un producto
\[
p(\xb) = \vb(\xb)^T \Qb \vb(\xb),
\]
con $\vb$ el vector de monomios de grado $d$ y $\Qb \in \R^{M(d) \times M(d)}$ sim\'etrica, con $M(d) =  \binom{n+d-1}{d}$, la cantidad de monomios de grado $d$ en $n$ variables

Esta ecuación nos una ecuación lineal para cada coeficiente de $p$, en total $\binom{n+2d-1}{2d}$ ecuaciones.

\end{frame}


%------------------------------------------------------------------


\begin{frame}


\frametitle{Ejemplo}

Para el polinomio $p(x,y) = 10x^4+2x^3y+27x^2y^2-24xy^3+5y^4$, planteamos la ecuación matricial
$10x^4+2x^3y+27x^2y^2-24xy^3+5y^4 = $
\[
\begin{pmatrix}
x^2 & xy & y^2
\end{pmatrix}
\begin{pmatrix}
q_{00} & q_{10} & q_{20} \\
q_{10} & q_{11} & q_{21} \\
q_{20} & q_{21} & q_{22} \\
\end{pmatrix}
\begin{pmatrix}
x^2 \\
xy \\
y^2 \\
\end{pmatrix}
\]
y obtenemos que se debe cumplir la igualdad
\begin{align*}
& 10x^4+2x^3y+27x^2y^2-24xy^3+5y^4 = \\
= \  & q_{00} x^4 + 2q_{10} x^3y + (2q_{20} + q_{11})x^2y^2 + 2q_{21}xy^3 + q_{22} y^4.
\end{align*}



\end{frame}



%------------------------------------------------------------------


\begin{frame}


\frametitle{Ejemplo}

Igualando coeficiente a coeficiente 
\begin{align*}
& 10x^4+2x^3y+27x^2y^2-24xy^3+5y^4 = \\
= \  & q_{00} x^4 + 2q_{10} x^3y + (2q_{20} + q_{11})x^2y^2 + 2q_{21}xy^3 + q_{22} y^4
\end{align*}
obtenemos
\begin{align*}
q_{00} &= 10 \\
2q_{10} &= 2 \\
2q_{20} + q_{11} &= 27 \\
2q_{21} &= -24 \\
q_{22} &= 5
\end{align*}



\end{frame}

%------------------------------------------------------------------


\begin{frame}


\frametitle{Ejemplo}

Despejando, obtenemos 

$10x^4+2x^3y+27x^2y^2-24xy^3+5y^4 = $
\[
\begin{pmatrix}
x^2 & xy & y^2
\end{pmatrix}
\begin{pmatrix}
10 & 1 & a \\
1 & -2a + 27 & -12 \\
a & -12 & 5 \\
\end{pmatrix}
\begin{pmatrix}
x^2 \\
xy \\
y^2 \\
\end{pmatrix}
\]
para cualquier $a \in \R$, y todas las matrices que cumplen la igualdad son de esta forma.


\end{frame}



%------------------------------------------------------------------

\begin{frame}

\frametitle{Descomposición como combinación lineal de cuadrados}

Tomamos por ejemplo $a = 1$ y diagonalizamos (descomposici\'on $LDL^t$ por eliminaci\'on gaussiana):
{\scriptsize
\[
\begin{pmatrix}
10 & 1 & 1 \\
1 & 25 & -12 \\
1 & -12 & 5 \\
\end{pmatrix}
=
\begin{pmatrix}
1 & 0 & 0 \\
\frac{1}{10} & 1 & 0 \\
\frac{1}{10} & -\frac{121}{249} & 1 \\
\end{pmatrix}
\begin{pmatrix}
10 & 0 & 0 \\
0 & \frac{249}{10} & 0 \\
0 & 0 & -\frac{244}{249} \\
\end{pmatrix}
\begin{pmatrix}
1 & \frac{1}{10}  & \frac{1}{10}  \\
0 & 1 & -\frac{121}{249}  \\
0 & 0 & 1 \\
\end{pmatrix}
\]
}

Obtenemos la descomposici\'on
\[
f = 10\left(x^2+\frac{1}{10}xy+\frac{1}{10}y^2\right)^2 + \frac{249}{10}\left(xy-\frac{121}{249}y^2\right)^2 - \frac{244}{249}\left(y^2\right)^2
\]

No es una suma de cuadrados porque el último coeficiente es negativo.

\end{frame}


%------------------------------------------------------------------

\begin{frame}

\frametitle{Signatura y suma de cuadrados}

\begin{block}{}
La cantidad de valores positivos en la diagonal es igual a la cantidad de autovalores positivos. 

Por lo tanto, debemos hallar $a \in \R$ tal que $\Qb$ sea semidefinida positiva $\longrightarrow$ \emph{Problema de programaci\'on semidefinida}.
\end{block}

Concretamente, tenemos el siguiente resultado.

\textbf{Propiedad:} Si $p(\xb) \in \R[\xb]$ es un polinomio homogéneo de grado $2d$, las siguientes propiedades son equivalentes:
\begin{enumerate}
\item $p(\xb)$ es una suma de cuadrados,
\item existe $\Qb \succeq 0$ que satisface la fórmula $p(\xb) = \vb(\xb)^T \Qb \vb(\xb)$,
\end{enumerate}
para $\vb(\xb)$ el vector de monomios de grado $d$ en $\R[\xb]$.

\end{frame}


%------------------------------------------------------------------

\begin{frame}

\frametitle{Demostración}

Para probar (2) $\Rightarrow$ (1), dada una matriz $\Qb \succeq 0$, podemos factorizarla $\Qb = \Lb^T \Lb$, con $\Lb$ triangular inferior y obtenemos
$$
p(\xb) = \vb(\xb)^T \Qb \vb(\xb) = \vb(\xb)^T \Lb^T \Lb \vb(\xb) = \sum (\inner{\Lb_i}{\vb(\xb)})^2 = \sum q_i(\xb)^2.
$$

Recíprocamente, si $p(\xb)$ es SOS, $p(\xb) = \sum q_i(\xb)^2$, construimos la matriz $\Xb$ tomando en la fila $i$ los coeficientes de $q_i(\xb)$ y tomamos $\Qb = \Xb^T \Xb \succeq 0$.

\end{frame}


%------------------------------------------------------------------

\begin{frame}

\frametitle{Representación núcleo y representación imagen}

Vemos que podemos plantear el problema mediante ecuaciones sobre los coeficientes de $\Qb$, que se obtienen igualando coeficiente a coeficiente la expresión
$$p(\xb) = \vb(\xb)^T \Qb \vb(\xb).$$
Llamamos representación núcleo o implícita a esta representación.

Resolviendo las ecuaciones, podemos plantear el problema mediante una desigualdad lineal matricial (LMI)
$$\Qb = \Ab_0 + \sum_i y_i \Ab_i$$
que llamamos representación imagen o explícita.


Pregunta: ¿para el ejemplo anterior cuál es la representación explícita y cuál es la representación implícita?

\end{frame}

%------------------------------------------------------------------

\begin{frame}

\frametitle{Programas Sumas de Cuadrados - Motivación}

Vimos la siguiente aplicación de polinomios no-negativos:

Dado un polinomio $p$, si queremos hallar el mínimo de $p$ en $S = \R^n$, planteamos
\[
\min\{f(\xb) : \xb \in \R^n\} = \sup\{\gamma \in \R | f(\xb) - \gamma \ge 0 \ \forall \xb \in \R^n\}.
\]

Para $\gamma \in \R$ dado, podemos reemplazar la condición $f(\xb) - \gamma \ge 0$ por $f(\xb)$ es SOS, lo que nos da un problema de factibilidad de un SDP. 

Resolviendo estos problemas para distintos valores de $\gamma$, podemos obtener cotas para el mínimo de $p$.

Pregunta: ¿podemos obtener la mejor cota resolviendo un solo problema SDP?


\end{frame}

%------------------------------------------------------------------

\begin{frame}

\frametitle{Programas Sumas de Cuadrados}

Vimos cómo verificar si un polinomio dado es una suma de cuadrados.

Podemos extender los resultados para definir una clase de problemas de optimización convexa que llamamos \emph{programas sumas de cuadrados} (SOS).

\begin{definition}
Un problema de optimización por sumas de cuadrados o programa SOS es un problema de optimización convexa de la forma
\begin{alignat*}{2}
  & \text{maximizar: } & & b_1 y_1 + \dots + b_m y_m  \\
   & \text{sujeto a: } & \quad & p_i(\xb; \yb) \text{ es SOS }, 1 \le i \le k,
\end{alignat*}
donde $\yb = (y_1, \dots, y_m) \in \R^m$ es la variable de optimización, $b_i \in \R$, $1 \le i \le m$, $p_i(\xb; \yb) = a_{i0}(\xb) + a_{i1}(\xb) y_1 + \dots + a_{im}(\xb) y_m$, $1 \le i \le k$ y $a_{ij}(\xb) \in \R[\xb]$ son polinomios dados.
\end{definition}



\end{frame}

%------------------------------------------------------------------

\begin{frame}

\frametitle{Programas Sumas de Cuadrados}

\textbf{Observaciones:}
\begin{itemize}
\item Los polinomios $p_i(\xb, \yb)$ son polinomios arbitrarios que son combinaciones afines en los parámetros $y_1, \dots, y_m$.
\item Las variables $\xb$ son variables ``dummy'', no optimizamos sobre ellas sino que son las indeterminadas de los polinomios $p_i$.
\end{itemize}

\textbf{Ejemplo.}
Consideramos el problema
\begin{alignat*}{2}
  & \text{maximizar: } & & y_1 + y_2  \\
  & \text{sujeto a: } & \quad & x^4 + y_1 x + (2 + y_2) \quad \text{es SOS,} \\
  &  & \quad & (y_1 - y_2 + 1)x^2 + y_2 x + 1 \quad  \text{es SOS.}
\end{alignat*}

\end{frame}

%------------------------------------------------------------------

\begin{frame}

\frametitle{Programas Sumas de Cuadrados}

Aunque a primera vista, los programas SOS parecen más generales que los problemas SDP, cada restricción del problema podemos plantearla como la existencia de una matriz $\Qb_i \succeq 0$ tal que
$$
p_i(\xb; \yb) = \vb^T \Qb_i \vb, 
$$
donde las coordenadas de $\Qb_i$ dependen linealmente de las variables $y_i$, $1 \le i \le m$, y por lo tanto un programa SOS es un problema SDP.
\end{frame}

%------------------------------------------------------------------

\begin{frame}

\frametitle{Ejemplo}

Para la restricción 
$$p_1(x; y_1, y_2) = x^4 + y_1 x + (2+y_2) \quad \text{ es sos }
$$
planteamos
$$
x^4 + y_1 x + (2+y_2) = \begin{pmatrix} 1 & x & x^2 \end{pmatrix}
\begin{pmatrix}
q_{00} & q_{10} & q_{20} \\
q_{10} & q_{11} & q_{21} \\
q_{20} & q_{21} & q_{22} \\
\end{pmatrix}
\begin{pmatrix} 1 \\ x \\ x^2 \end{pmatrix}
$$
y obtenemos la restricción $\Qb = 
\begin{pmatrix} 
2+y_2 & \frac{y_1}{2} & -\frac{a}{2} \\ 
\frac{y_1}{2} & a & 0 \\
-\frac{a}{2} & 0 & 1
\end{pmatrix} \succeq 0,
$ donde $a, y_1, y_2$ son las variables del problema SDP.


\end{frame}

%------------------------------------------------------------------

\begin{frame}

\frametitle{Aplicación: optimización polinomial sin restricciones}

\textbf{Caso polinomios univariados}

Para encontrar el mínimo de un polinomio en una variable, utilizamos la equivalencia $
p(x) \ge \gamma \ \forall x \in \R \quad \iff \quad p(x) - \gamma \ge 0 \ \forall x \in \R$.

Obtenemos el siguiente problema de optimización:
\begin{alignat*}{2}
  & \text{maximizar: } & & \gamma  \\
  & \text{sujeto a: } & \quad & p(x) - \gamma \ge 0 \ \forall x \in \R.
\end{alignat*}

Como en una variable un polinomio es no-negativo si y solo si es SOS, obtenemos el problema SOS equivalente
\begin{alignat*}{2}
  & \text{maximizar: } & & \gamma  \\
  & \text{sujeto a: } & \quad & p(x) - \gamma \quad \text{es SOS}.
\end{alignat*}

\end{frame}

%------------------------------------------------------------------

\begin{frame}

\frametitle{Aplicación: optimización polinomial sin restricciones}

\textbf{Caso polinomios multivariados}

Análogamente, si queremos encontrar el mínimo de un polinomio multivariado planteamos el problema de optimización:
\begin{alignat*}{2}
  & \text{maximizar: } & & \gamma  \\
  & \text{sujeto a: } & \quad & p(\xb) - \gamma \ge 0 \ \forall \xb \in \R^n.
\end{alignat*}

En el caso general este problema no se puede plantear eficientemente, pero podemos plantear el problema alternativo
\begin{alignat*}{2}
  & \text{maximizar: } & & \gamma  \\
  & \text{sujeto a: } & \quad & p(\xb) - \gamma \quad \text{es SOS}.
\end{alignat*}

\end{frame}

%------------------------------------------------------------------

\begin{frame}

\frametitle{Aplicación: optimización polinomial sin restricciones}

Llamamos $p_{\star}$ al ínfimo de $p$ (que coincide con el óptimo del primer problema) y $p_{SOS}$ al óptimo del segundo problema.

Como el conjunto factible del problema SOS está incluido en el conjunto factible del primero, obtenemos la desigualdad
$$
p_{SOS} \le p_{\star}.
$$

Si bien en muchos casos (especialmente en dimensión baja) ambos óptimos coinciden, el primer problema es NP-hard y por lo tanto no podemos esperar que los óptimos coincidan siempre.

\end{frame}

%------------------------------------------------------------------


\end{document} 