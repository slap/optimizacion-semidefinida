\documentclass[aspectratio=169,12pt,spanish]{beamer}
\usepackage[T1]{fontenc}
\usepackage[spanish]{babel}

\usepackage{wrapfig}

%\usepackage{multicol}
%\usepackage{mathtools}

\usepackage[normalem]{ulem}

\usepackage{pgf,tikz}
\usetikzlibrary{matrix}
\usetikzlibrary{arrows}

%\usepackage{wrapfig}
\mode<presentation>
\usefonttheme{professionalfonts}
\usetheme{Darmstadt}
\usecolortheme{orchid}
\useoutertheme{default}
\setbeamertemplate{headline}{}

\newcounter{savedenum}
\newcommand*{\saveenum}{\setcounter{savedenum}{\theenumi}}
\newcommand*{\resume}{\setcounter{enumi}{\thesavedenum}}

\renewcommand{\baselinestretch}{1.1}

%gets rid of bottom navigation bars
\setbeamertemplate{footline}[page number]

%gets rid of navigation symbols
\setbeamertemplate{navigation symbols}{}

%\frameframe{none} % No default frame

%\setlength{\framewidth}{8.7in} \setlength{\frameheight}{7.2in}

\parindent 0pt
\setlength{\parskip} {1ex plus 0.5ex minus 0.2ex}


\usepackage[bbgreekl]{mathbbol}
\usepackage{amssymb, amsthm, amsmath}
\usepackage{bm}

\newtheorem{ejercicio}{Ejercicio}
\newtheorem{proposition}[theorem]{Proposición}

\DeclareSymbolFontAlphabet{\mathbb}{AMSb}
\DeclareSymbolFontAlphabet{\mathbbl}{bbold}

\usepackage{multicol}
\usepackage{colortbl}
\usepackage{lmodern}
\usepackage{tabularx}
\usepackage{multirow}
\usepackage{stmaryrd}
\usepackage{color}
\usepackage{graphicx}
\usepackage{hyperref}

\graphicspath{ {../../images} }
\usepackage{listings}
\lstset{
  basicstyle=\ttfamily,
  columns=fullflexible,
}

\usepackage{url}
\usepackage{multicol}
\usepackage{dsfont}

% Bold symbols for vectors and matrices
\newcommand{\xstar}{\bm{x}^{\star}}
\newcommand{\alphab}{\bm{\alpha}}
\newcommand{\ab}{\bm{a}}
\newcommand{\bb}{\bm{b}}
\newcommand{\cb}{\bm{c}}
\newcommand{\db}{\bm{d}}
\newcommand{\eb}{\bm{e}}
\newcommand{\gb}{\bm{g}}
\newcommand{\mb}{\bm{m}}
\newcommand{\pb}{\bm{p}}
\newcommand{\qb}{\bm{q}}
\newcommand{\rb}{\bm{r}}
\newcommand{\ssb}{\bm{s}}
\newcommand{\ub}{\bm{u}}
\newcommand{\vb}{\bm{v}}
\newcommand{\wb}{\bm{w}}
\newcommand{\xb}{\bm{x}}
\newcommand{\yb}{\bm{y}}
\newcommand{\zb}{\bm{z}}

\newcommand{\Ab}{\bm{A}}
\newcommand{\Bb}{\bm{B}}
\newcommand{\Cb}{\bm{C}}
\newcommand{\Db}{\bm{D}}
\newcommand{\Eb}{\bm{E}}
\newcommand{\Fb}{\bm{F}}
\newcommand{\Gb}{\bm{G}}
\newcommand{\Hb}{\bm{H}}
\newcommand{\Ib}{\bm{I}}
\newcommand{\Id}{\bm{I}}
\newcommand{\Kb}{\bm{K}}
\newcommand{\Lb}{\bm{L}}
\newcommand{\Mb}{\bm{M}}
\newcommand{\Pb}{\bm{P}}
\newcommand{\Qb}{\bm{Q}}
\newcommand{\Rb}{\bm{R}}
\newcommand{\Sb}{\bm{S}}
\newcommand{\Tb}{\bm{T}}
\newcommand{\Ub}{\bm{U}}
\newcommand{\Vb}{\bm{V}}
\newcommand{\Wb}{\bm{W}}
\newcommand{\Xb}{\bm{X}}
\newcommand{\Yb}{\bm{Y}}
\newcommand{\Zb}{\bm{Z}}
\newcommand{\Lambdab}{\bm{\Lambda}}
\newcommand{\cero}{\bm{0}}

% Rings and fields
\newcommand{\A}{\mathbb{A}}
\newcommand{\Z}{\mathbb{Z}}
\newcommand{\Q}{\mathbb{Q}}
\newcommand{\C}{\mathbb{C}}
\newcommand{\R}{\mathbb{R}}
\newcommand{\K}{\mathbb{K}}
\newcommand{\N}{\mathbb{N}}

\newcommand{\borel}{{\mathcal B}}
\newcommand{\pmom}{{\rho_{\text{mom}}}}
\newcommand{\MX}{{\mathcal{M}(X)}}


% Inner product
\newcommand{\innerl}[2]{\langle #1, #2 \rangle}
\newcommand{\inner}[2]{#1 \boldsymbol{\cdot} #2}
\newcommand{\innerTrace}[2]{#1 \bullet #2}

% Symmetric and positive definite matrices
\newcommand{\Splusplusn}{{\mathcal S_{++}^n}}
\newcommand{\Splusn}{{\mathcal S_+^n}}
\newcommand{\Splus}{{\mathcal S_+}}
\newcommand{\Sym}{{\mathcal S}}
\newcommand{\Symn}{{\mathcal S^n}}

% Cones
\newcommand\CC{\mathcal{C}}
\DeclareMathOperator{\cone}{cono}
\DeclareMathOperator{\conv}{conv}
\DeclareMathOperator{\supp}{supp}


% Spectrahedron
\newcommand{\eLL}{{\mathcal L}}

% Matrices and vectors over R or C
\newcommand{\Rnn}{\R^{n\times n}}
\newcommand{\Cnn}{\C^{n\times n}}
\newcommand{\Rn}{\R^{n}}
\newcommand{\Rm}{\R^{m}}


% Math operators
\DeclareMathOperator{\Tr}{Tr}
\DeclareMathOperator{\tr}{Tr}
\DeclareMathOperator{\interior}{int}
\DeclareMathOperator{\rank}{rank}
\DeclareMathOperator{\diag}{diag}

\newcommand\one{\mathds{1}} 

\pagestyle{empty}

\begin{document}

%------------------------------------------------------------------

\begin{frame}

 \begin{center}

\Large\textbf{Optimización Semidefinida} \\
\large\textbf{Clase 15 - El problema de los momentos}
%\vspace{0.5cm}

% \textit{Santiago Laplagne} \\
%slaplagn@dm.uba.ar \\


%\vspace{0.5cm}
%{\small Trabajo en progreso en conjunto con \emph{Jose Capco} (Universit\"at Innsbruck) y \emph{Claus Scheiderer} %(Universit\"at Konstanz).} \\

\vspace{1cm}
 Segundo Cuatrimestre 2021
 \\
 {\small Facultad de Ciencias Exactas y Naturales, UBA}
 \end{center}

\end{frame}


%------------------------------------------------------------------

\begin{frame}

\frametitle{Motivación: optimización global de polinomios}

Dado $f \in \R[\xb]$, consideramos el problema de optimización
\begin{alignat*}{2}
  f^{\star} = & & \sup \quad & f(\xb)    \\
   & & \text{sujeto a:} \quad & \xb \in X
\end{alignat*}
y el \emph{problema generalizado de momentos} asociado
\begin{alignat*}{2}
  \rho_{\text{mom}} = & & \sup_{\mu \in \mathcal{M}(X)} & \int_X f d\mu   \\
   & & \text{sujeto a: } & \int_X d\mu = 1.
\end{alignat*}

Ambos problemas son equivalentes. Es decir, $f^{\star} = \rho_{\text{mom}}$.


\end{frame}

%------------------------------------------------------------------

\begin{frame}

\frametitle{El problema de los momentos - Introducción}

Sea $X$ un subconjunto cerrado de $\R^n$ y $\borel = \borel(X)$ la $\sigma$-algebra generada por los conjuntos abiertos de $X$.

Consideramos una medida de Borel en $X$
$$
\mu: \borel(X) \rightarrow [0, +\infty]
$$
tal que $\mu(K) < +\infty$ para cualquier subconjunto compacto $K$ de $X$.

Para una medida $\mu$ en $\borel(\R^n)$, definimos el soporte de $\mu$, $\supp(\mu)$, como el complemento en $\R^n$ del mayor conjunto abierto $O \subset \R^n$ tal que $\mu(O) = 0$.

\end{frame}

%------------------------------------------------------------------

\begin{frame}

\frametitle{Medidas de Borel - Ejemplos}

Ejemplos:
\begin{enumerate}
\item Dada una función continua no-negativa $g$ en $X$, definimos para $M \in \borel$
$$\mu(M) = \int_M g \ dx,$$
la integral de Lebesgue de $g$ en $M$.
\item Dado un subconjunto finito $E$ de $X$, y una función $g: E \rightarrow \R_{\ge 0}$, definimos para $M \in \borel$
$$
\mu(M) = \sum_{\xb \in M \cap E} g(\xb).
$$
\item (Medida de Dirac) Como caso particular, dado un punto $\xb \in X$ definimos la medida
$$\delta_{\xb}(M) = \begin{cases}
1 & \text{ si } \xb \in M, \\
0 & \text{ si no. }
\end{cases}
$$
\end{enumerate}

\end{frame}

%------------------------------------------------------------------

\begin{frame}

\frametitle{Integral de Lebesgue}

A partir de una medida $\mu$ podemos definir la integral de Lebesgue con respecto a $\mu$, $f \rightarrow \int f d\mu$.

En los ejemplos anteriores, tenemos
\begin{enumerate}
\item Si $\mu(M) = \int_M g dx$,
$$\int_X f d\mu = \int_X f(x) g(x) dx.$$
\item Si $\mu(M) = \sum_{x \in M \cap E} g(x)$,
$$\int_X f d\mu = \sum_{x \in M \cap E} f(x) g(x).$$
\end{enumerate}

\end{frame}

%------------------------------------------------------------------

\begin{frame}

\frametitle{El problema generalizado de los momentos}

Para $\mathcal{M}(X)$ el espacio de medidas de Borel  en $X$ positivas, definimos el problema generalizado de momentos (GMP)

\begin{alignat*}{2}
  \rho_{\text{mom}} = & & \sup_{\mu \in \mathcal{M}(X)} & \int_X f d\mu   \\
   & & \text{sujeto a: } & \int_X h_j d\mu \le \gamma_j, \quad j \in \Gamma_1 \\
  &  & & \int_X h_j d\mu = \gamma_j, \quad j \in \Gamma_2
\end{alignat*}

\end{frame}

%------------------------------------------------------------------

\begin{frame}

\frametitle{Aplicación: optimización global de polinomios}

Dado $f \in \R[\xb]$, consideramos el problema de optimización
\begin{alignat*}{2}
  f^{\star} = & & \sup \quad & f(\xb)    \\
   & & \text{sujeto a:} \quad & \xb \in X
\end{alignat*}
y el problema generalizado de momentos asociado
\begin{alignat*}{2}
  \rho_{\text{mom}} = & & \sup_{\mu \in \mathcal{M}(X)} & \int_X f d\mu   \\
   & & \text{sujeto a: } & \int_X d\mu = 1.
\end{alignat*}

\begin{theorem}
Ambos problemas son equivalentes. Es decir, $f^{\star} = \rho_{\text{mom}}$.
\end{theorem}


\end{frame}

%------------------------------------------------------------------

\begin{frame}

\frametitle{Demostración}

Consideramos el caso $f^{\star} < + \infty$.

Como $f(\xb) \le f^{\star}$ para todo $\xb \in X$,
$$
\int_X f d\mu \le \int_X f^{\star} d\mu  = f^{\star}
$$
y por lo tanto $\pmom \le f^{\star}$.

Recíprocamente, para $\xb \in X$ tomamos la medida de Dirac
$$\delta_{\xb} \in \MX,$$
que es una solución factible del problema, con valor $f(\xb)$, y por lo tanto $\pmom \ge f^{\star}$.

\end{frame}

%------------------------------------------------------------------

\begin{frame}

\frametitle{Momentos - Motivación}

Dado un polinomio $f \in \R[x_1, \dots, x_n]$,
$$
f = f_{\alphab_1} \xb^{\alphab_1} + f_{\alphab_2} \xb^{\alphab_2} + \dots + f_{\alphab_s} \xb^{\alphab_s},
$$
con $\alphab_i \in \N_0^n$ podemos calcular su integral
$$
\int_X f d\mu = f_{\alphab_1} \int_X \xb^{\alphab_1} d\mu + f_{\alphab_2} \int_X \xb^{\alphab_2} d\mu + \dots + f_{\alphab_s} \int_X \xb^{\alphab_s}d\mu,
$$
conociendo las integrales de los monomios $\int_X \xb^{\alphab}$ para todo $\alphab \in \N_0^n$.

Esto motiva a considerar el problema de los posibles valores que pueden tomar dichas integrales para entender las posibles medidas en $\MX$.

\end{frame}

%------------------------------------------------------------------

\begin{frame}

\frametitle{El problema de los momentos}
Dado un conjunto semi-algebraico básico $$S = \{\xb \in \R^n \mid g_1(\xb) \ge 0, \dots, g_m(\xb) \ge 0\} \subset \R^n$$
definido por polinomios $g_1, \dots, g_m \in \R[\xb]$, consideramos los problemas:

\textbf{El problema de momentos completo.}
Dada una sucesión infinita $\yb = (y_{\alphab}) \subset \R$ de números reales, con $\alphab \in \N_0^n$, ¿existe una medida $\mu$ con soporte en $S$ tal que
$$
y_{\alphab} = \int_S \xb^{\alphab} d\mu \quad \forall \alphab \in \N_0^n?
$$

\textbf{El problema de momentos truncado.}
Dado un conjunto finito $\Delta \subset \N_0^n$ y una sucesión finita $\yb = (y_{\alphab})_{\alphab \in \Delta} \subset \R$ de números reales, ¿existe una medida $\mu$ con soporte en $S$ tal que
$$
y_{\alphab} = \int_S \xb^{\alphab} d\mu \quad \forall \alphab \in \Delta?
$$

\end{frame}

%------------------------------------------------------------------

\begin{frame}

\frametitle{Ejercicio}

En $\R^2$, para un conjunto $S$ tal que $(3,-1) \in S$, calcular la sucesión
$$y_{\alphab} = \int_S \xb^{\alphab} d\mu$$
 para la medida
$$\mu(X) = \delta_{(3,-1)}(X).$$

Utilizar esta sucesión para calcular $p(3,-1)$ para
$$p(x,y) = x^2y^3 - 2x + 1.$$

\end{frame}

%------------------------------------------------------------------

\begin{frame}

\frametitle{La integral como funcional lineal}

Recordemos que para $f(\xb) = \sum_{\alphab \in \N^n} f_{\alphab} \xb^{\alphab}$,
$$
\int_X f d\mu = \sum_{\alphab \in \N^n} f_{\alphab} \int_X \xb^{\alphab} d\mu.
$$

Motivados por esta fórmula, para una secuencia infinita $\yb = (y_{\alphab}) \subset \R$ definimos la funcional lineal $L_{\yb}: \R[\xb] \rightarrow \R$ por
$$
f(\xb) = \sum_{\alphab \in \N^n} f_{\alphab} \xb^{\alphab} \mapsto L_{\yb}(f) = \sum_{\alphab \in \N^n} f_{\alphab} y_{\alphab}.
$$
\end{frame}

%------------------------------------------------------------------

\begin{frame}

\frametitle{El teorema de Riesz-Haviland}

El siguiente teorema (que vemos sin demostración) será clave para resolver algunos casos del problema de momentos.

\begin{theorem}[Riesz-Haviland]
\label{teo:Riesz}
Sea $\yb = (y_{\alphab})_{\alphab \in \N_0^n} \subset \R$ y sea $K \subset \R^n$ un conjunto cerrado.
Existe una medida de Borel finita $\mu$ en $K$ tal que
$$
\int_K \xb^{\alphab} d\mu = y_{\alphab}, \quad \forall \alphab \in \N_0^n,
$$
si y solo si $L_{\yb}(f) \ge 0$ para todos los polinomios $f \in \R[\xb]$ no-negativos en $K$.
\end{theorem}

\textbf{Observación:} La implicación $\Rightarrow$ es inmediata.


\end{frame}

%------------------------------------------------------------------

\begin{frame}

\frametitle{El problema de los momentos en una variable}

\textbf{Matrices de Hankel}

Dada una sucesión $\yb = (y_i)_{i \in \N_0} \subset \R$, definimos las matrices de Hankel $H_n(\yb) \in \R^{(n+1) \times (n+1)}$ por
\begin{align*}
H_n(\yb)(i,j) &:= y_{i+j-2},
\end{align*}
para todo $i, j \in \N$, $1 \le i,j \le n+1$.


\end{frame}

%------------------------------------------------------------------

\begin{frame}

\frametitle{Ejemplo}

Para $\yb = (1,2,3,4,5,6,7,8,0,0,0, \dots)$,
$$
H_3 = \begin{pmatrix}
y_0 & y_1& y_2 & y_3\\
y_1 & y_2 & y_3 & y_4\\
y_2 & y_3 & y_4 & y_5 \\
y_3 & y_4 & y_5 & y_6 
\end{pmatrix} = 
\begin{pmatrix}
1 & 2 & 3 & 4\\
2 & 3 & 4 & 5\\
3 & 4 & 5 & 6\\
4 & 5 & 6 & 7
\end{pmatrix}.
$$


\end{frame}

%------------------------------------------------------------------

\begin{frame}

\frametitle{El problema de los momentos de Hamburger}

\begin{theorem}
\label{it:hankel} 
Sea $\yb = (y_j)_{j \in \N_0} \subset \R$. Existe una medida de Borel $\mu$ que representa a $\yb$ en $\R$ si y solo si la forma cuadrática
\begin{equation}
\label{eq:hankel}
\qb \mapsto s_n(\qb) := \sum_{i,j = 0}^n y_{i+j} q_i q_j
\end{equation}
es positiva semidefinida para todo $n \in \N$. 

Equivalentemente, $H_n(\yb) \succeq 0$ para todo $n \in \N$.
\end{theorem}

\end{frame}

%------------------------------------------------------------------

\begin{frame}

\frametitle{Demostración}

$(\Rightarrow)$ Si $y_n = \int_\R z^n d\mu(z)$, entonces
\begin{align*}
s_n(\qb) &= \sum_{i,j = 0}^n y_{i+j} q_i q_j \\
&= \sum_{i,j = 0}^n q_i q_j \int_{\R} z^{i+j} d\mu(z) \\
& = \int_{\R} \left(\sum_{i,j = 0}^n q_i q_j z^{i+j} \right) d\mu(z) = \int_{\R} \sum_{i,j = 0}^n (q_i z^i) (q_j z^j) d\mu(z)\\
& = \int_\R (\sum_{i=0}^n q_i z^i)^2 d\mu(z) \ge 0.
\end{align*}
\end{frame}

%------------------------------------------------------------------

\begin{frame}

\frametitle{Demostración}

Recíprocamente, si $H_n(\yb) \succeq 0$ para todo $n \in \N$ y
$$f = \sum_{k=0}^{2d} f_k x^k \in \R[x]$$ 
 un polinomio no-negativo en $\R$, veamos que $\sum_{k=0}^{2d} f_k y_k = L_{\yb}(f) \ge 0$.

Recordemos que $f$ se puede escribir como una suma de cuadrados $f = \sum_{j = 1}^r g_j^2$. Luego,
$$
L_{\yb}(f) = L_{\yb}(\sum_{j=1}^r g_j^2) = \sum_{j=1}^r L_{\yb}(g_j^2).
$$

Veamos que $L_{\yb}(g^2) \ge 0$ para cualquier polinomio $g \in \R[x]$.
\end{frame}

%------------------------------------------------------------------

\begin{frame}

\frametitle{Demostración}

Para todo $\qb \in \R^{n+1}$ tenemos $\qb^t H_n(\yb) \qb \ge 0$.

Ahora 
\begin{align*}
L_{\yb}(g^2) &= L_{\yb}\left((\sum_{j=0}^d g_j x^j)^2\right) \\
&= L_{\yb}\left(\sum_{i,j} g_i g_j x^{i+j}\right) \\
&= \sum_{0 \le i,j \le d} g_i g_j y_{i+j} = \gb^t H_d(\yb) \gb \ge 0.
\end{align*}

Como $f \ge 0$ es arbitrario, obtenemos por el Teorema \ref{teo:Riesz} que $y_i = \int_\R x^i d\mu$ para alguna medida $\mu$ en $\R$.


\end{frame}

%------------------------------------------------------------------

\begin{frame}

\frametitle{Ejemplo}

Repasamos la construcción paso a paso en los siguientes ejemplos:

\begin{enumerate}
\item Hallar $\max_{x \in \R} 1 - x^2$.
\item Hallar $\min_{x \in \R} x^2 - 2x + 5$.
\end{enumerate}

En el primer caso, el problema es equivalente a 
\begin{alignat*}{2}
  \rho_{\text{mom}} = & & \sup_{\mu \in \mathcal{M}(\R)} & \int_{\R} 1 - x^2 d\mu   \\
   & & \text{sujeto a: } & \int_\R d\mu = 1.
\end{alignat*}

\end{frame}

%------------------------------------------------------------------

\begin{frame}

\frametitle{Ejemplo}

Por linealidad, 
$$
\int_{\R} 1 - x^2 d\mu = \int_{\R} 1 d\mu - \int_{\R} x^2 d\mu
$$
y definiendo $y_i = \int_{\R} x^i d\mu$, 
$$
\int_{\R} 1 - x^2 d\mu = y_0 - y_2.
$$

La restricción $\int_\R d\mu = 1$ nos da $y_0 = 1$.

Por lo tanto, nuestro problema se convierte en
\begin{alignat*}{2}
  \rho_{\text{mom}} = & & \sup_{\yb = (y_0, y_1, y_2, \dots)} & y_0 - y_2   \\
   & & \text{sujeto a: } & y_0 = 1, \\
   &&& \text{existe una medida $\mu$ que representa a $\yb$ en $\R$.}
\end{alignat*}


\end{frame}

%------------------------------------------------------------------

\begin{frame}

\frametitle{Ejemplo}

A la vez, este último problema es equivalente a 
\begin{alignat*}{2}
  \rho_{\text{mom}} = & & \sup_{\yb = (y_0, y_1, y_2, \dots)} & y_0 - y_2   \\
   & & \text{sujeto a: } & y_0 = 1, \\
   &&& \begin{pmatrix} 
   y_0 & y_1 & y_2 & \dots & y_n\\
   y_1 & y_2 & y_3 & &  \\
   y_2 & y_3 & y_4 & &\vdots \\
   \vdots & &  &  & \\
   y_n & & \cdots &  & y_{2n}
   \end{pmatrix} \succeq 0 \ \text{ para todo } n \in \N
\end{alignat*}



\end{frame}

%------------------------------------------------------------------

\begin{frame}

\frametitle{Ejemplo}

En particular 
$$\begin{pmatrix}
   1 & y_1 \\
   y_1 & y_2 \\
   \end{pmatrix} \succeq 0 
$$ 
y esto nos da las restricciones $y_2 \ge 0$ y $y_2 \ge y_1^2$.

El máximo de $1 - y_2$ en esta región se da para $y_2 = 0$. Por lo tanto, obtenemos $$\max_{x \in \R} 1 - x^2 \le 1.$$

Tomando la sucesión $\yb = (1, 0, 0, 0, \dots)$, vemos que $$\max_{x \in \R} 1 - x^2 = 1.$$

\end{frame}

%------------------------------------------------------------------

\begin{frame}

\frametitle{Ejemplo}

En el caso de $\min_{x \in \R} x^2 - 2x + 5$, obtenemos en forma análoga el problema equivalente
\begin{alignat*}{2}
  \rho_{\text{mom}} = & & \inf_{\yb = (y_0, y_1, y_2, \dots)} & 5 y_0 - 2 y_ 1 + y_2   \\
   & & \text{sujeto a: } & y_0 = 1, \\
   &&& \begin{pmatrix}
   y_0 & y_1 & y_2 & \dots & y_n\\
   y_1 & y_2 & y_3 & &  \\
   y_2 & y_3 & y_4 & &\vdots \\
   \vdots & &  &  & \\
   y_n & & \cdots &  & y_{2n}
   \end{pmatrix} \succeq 0 \ \text{ para todo } n \in \N
\end{alignat*}

Vemos que al modificar el polinomio, solo cambia la función objetivo, pero no la región de factibilidad.
\end{frame}

%------------------------------------------------------------------

\begin{frame}

\frametitle{Ejemplo}

En este caso, el mínimo de $5 - 2 y_ 1 + y_2$ en la región $y_2 \ge 0$ y $y_2 \ge y_1^2$ se alcanza para
$$
y_1 = y_2 = 1
$$
y por lo tanto $$\min_{x \in \R} x^2 - 2x + 5 \ge 5 - 2 + 1 = 4.$$

Tomando la sucesión $\yb = (1, 1, 1, 1, \dots)$, vemos que $$\min_{x \in \R} x^2 - 2x + 5 = 4.$$

Encontrar condiciones para las cuales es posible extender una sucesión finita $(y_0, y_1, \dots, y_n)$ a una sucesión representada por una medida $\mu$ se conoce como el problema truncado de los momentos.

\end{frame}

%------------------------------------------------------------------

\begin{frame}

\frametitle{El problema de los momentos de Stieltjes}

Si en lugar de tomar $X = \R$ tomamos $X = \R_{\ge 0}$, el problema correspondiente se conoce como el problema de los momentos de Stieltjes.

Dada una sucesión $y = (y_i) \subset \R$, definimos la matricez de Hankel $B_n(y) \in \R^{(n+1) \times (n+1)}$ por
$B_n(y)(i,j) := y_{i+j-1}$, 
para todo $i, j \in \N$, $1 \le i,j \le n+1$.

\begin{theorem}
Sea $\yb = (y_j)_{j \in \N_0} \subset \R$. Existe una medida de Borel $\mu$ que representa a $\yb$ en $\R_{\ge 0}$ si y solo si $\Hb_n(\yb) \succeq 0$ y $\Bb_n(\yb) \succeq 0$ para todo $n \in \N$.
\label{it:stieltjes}
\end{theorem}

\textbf{Demostración:} Ejercicio. Utilizar que $p \in \R[x]$ es no-negativo en $[0, +\infty)$ si y solo si 
$p = f_0 + x f_1$ para $f_0, f_1$ polinomios sumas de cuadrados.
\end{frame}

\end{document} 