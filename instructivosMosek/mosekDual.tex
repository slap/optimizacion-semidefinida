\documentclass[11pt]{article}
\usepackage[T1]{fontenc}
\usepackage[spanish]{babel}

%\include{amsfonts}
%\usepackage{amssymb,amsmath,latexsym,epsfig,euscript}
\usepackage{amssymb}
\usepackage{enumitem}
\usepackage{amsmath}
\usepackage{multicol}

\usepackage{bm}


%
\def\d{\displaystyle}
%
\usepackage[heightrounded]{geometry}
\geometry{top=2cm,bottom=2cm,left=2cm,right=2cm}
\parindent=0pt

\usepackage{graphicx}

\graphicspath{ {../../images} }
\usepackage{listings}
\lstset{
  basicstyle=\ttfamily,
  columns=fullflexible,
}

\usepackage{url}
\usepackage{multicol}
\usepackage{dsfont}

% Bold symbols for vectors and matrices
\newcommand{\xstar}{\bm{x}^{\star}}
\newcommand{\alphab}{\bm{\alpha}}
\newcommand{\ab}{\bm{a}}
\newcommand{\bb}{\bm{b}}
\newcommand{\cb}{\bm{c}}
\newcommand{\db}{\bm{d}}
\newcommand{\eb}{\bm{e}}
\newcommand{\gb}{\bm{g}}
\newcommand{\mb}{\bm{m}}
\newcommand{\pb}{\bm{p}}
\newcommand{\qb}{\bm{q}}
\newcommand{\rb}{\bm{r}}
\newcommand{\ssb}{\bm{s}}
\newcommand{\ub}{\bm{u}}
\newcommand{\vb}{\bm{v}}
\newcommand{\wb}{\bm{w}}
\newcommand{\xb}{\bm{x}}
\newcommand{\yb}{\bm{y}}
\newcommand{\zb}{\bm{z}}

\newcommand{\Ab}{\bm{A}}
\newcommand{\Bb}{\bm{B}}
\newcommand{\Cb}{\bm{C}}
\newcommand{\Db}{\bm{D}}
\newcommand{\Eb}{\bm{E}}
\newcommand{\Fb}{\bm{F}}
\newcommand{\Gb}{\bm{G}}
\newcommand{\Hb}{\bm{H}}
\newcommand{\Ib}{\bm{I}}
\newcommand{\Id}{\bm{I}}
\newcommand{\Kb}{\bm{K}}
\newcommand{\Lb}{\bm{L}}
\newcommand{\Mb}{\bm{M}}
\newcommand{\Pb}{\bm{P}}
\newcommand{\Qb}{\bm{Q}}
\newcommand{\Rb}{\bm{R}}
\newcommand{\Sb}{\bm{S}}
\newcommand{\Tb}{\bm{T}}
\newcommand{\Ub}{\bm{U}}
\newcommand{\Vb}{\bm{V}}
\newcommand{\Wb}{\bm{W}}
\newcommand{\Xb}{\bm{X}}
\newcommand{\Yb}{\bm{Y}}
\newcommand{\Zb}{\bm{Z}}
\newcommand{\Lambdab}{\bm{\Lambda}}
\newcommand{\cero}{\bm{0}}

% Rings and fields
\newcommand{\A}{\mathbb{A}}
\newcommand{\Z}{\mathbb{Z}}
\newcommand{\Q}{\mathbb{Q}}
\newcommand{\C}{\mathbb{C}}
\newcommand{\R}{\mathbb{R}}
\newcommand{\K}{\mathbb{K}}
\newcommand{\N}{\mathbb{N}}

\newcommand{\borel}{{\mathcal B}}
\newcommand{\pmom}{{\rho_{\text{mom}}}}
\newcommand{\MX}{{\mathcal{M}(X)}}


% Inner product
\newcommand{\innerl}[2]{\langle #1, #2 \rangle}
\newcommand{\inner}[2]{#1 \boldsymbol{\cdot} #2}
\newcommand{\innerTrace}[2]{#1 \bullet #2}

% Symmetric and positive definite matrices
\newcommand{\Splusplusn}{{\mathcal S_{++}^n}}
\newcommand{\Splusn}{{\mathcal S_+^n}}
\newcommand{\Splus}{{\mathcal S_+}}
\newcommand{\Sym}{{\mathcal S}}
\newcommand{\Symn}{{\mathcal S^n}}

% Cones
\newcommand\CC{\mathcal{C}}
\DeclareMathOperator{\cone}{cono}
\DeclareMathOperator{\conv}{conv}
\DeclareMathOperator{\supp}{supp}


% Spectrahedron
\newcommand{\eLL}{{\mathcal L}}

% Matrices and vectors over R or C
\newcommand{\Rnn}{\R^{n\times n}}
\newcommand{\Cnn}{\C^{n\times n}}
\newcommand{\Rn}{\R^{n}}
\newcommand{\Rm}{\R^{m}}


% Math operators
\DeclareMathOperator{\Tr}{Tr}
\DeclareMathOperator{\tr}{Tr}
\DeclareMathOperator{\interior}{int}
\DeclareMathOperator{\rank}{rank}
\DeclareMathOperator{\diag}{diag}

\newcommand\one{\mathds{1}} 


\begin{document}


\begin{center}

{\small Facultad de Ciencias Exactas
y Naturales -- Universidad de Buenos Aires } \vskip 1cm

\textbf{{\large Optimización Semidefinida} - Segundo Cuatrimestre 2021}

\medskip\textbf{Problema dual SDP en Mosek}
\end{center}

\medskip

Referencia principal MOSEK Optimizer API for Python 9.3.6

https://docs.mosek.com/latest/pythonapi/index.html

Ver Sección 6.6 - Semidefinite optimization.

\section{Introducción}

Mosek puede resolver problemas de optimización semidefinida de la forma:
\begin{alignat*}{2}
  & \text{minimizar: } & & \sum_{j=0}^{n-1} c_j x_j + \sum_{j=0}^{p-1} \innerTrace{\bar \Cb_j}{\bar \Xb_j} \\
  & \text{sujeto a: }  \quad & & l_i^c \le \sum_{j=0}^{n-1} a_{ij} x_j + \sum_{j=0}^{p-1} \innerTrace{\bar \Ab_{ij}}{\bar \Xb_j} \le u_i^c, \quad i = 0, \dots, m-1, \\
  &  \quad & & l_i^x \le x_j \le u_i^x, \quad j = 0, \dots, n-1, \\
  & & & \xb \in \R^n, \bar \Xb_j \succeq 0, j = 0, \dots, p-1
\end{alignat*}


%\section{Ejemplo 1}
Vamos a implementar un problema dual en Mosek, es decir un problema de la forma:
\begin{alignat*}{2}
  & \text{maximizar: }  & \quad & \inner{\yb}{\bb},   \\
   & \text{sujeto a: } & \quad & \sum_{i=0}^{n-1} y_i \Pb_i  \preceq \Db,
\end{alignat*}
para matrices $\Db, \Pb_i (1 \le i \le s) \in \R^{d \times d}$ (utilizamos distinta notación para diferenciar de la notación de Mosek).

Observamos que en la forma de plantear el problema en Mosek no tenemos forma de ingresar una desigualdad lineal matricial como aparece en el problema dual. Lo que podemos hacer es definir una matrix 
$$
\Xb = \Db - \sum_{i=1}^{s} y_i \Pb_i,
$$
que debe cumplir $\Xb \succeq 0$.

Esta igualdad es la que nos da las restricciones que ingresamos en Mosek. Tenemos una restricción para cada casilla de $\Xb$ en la parte  triangular inferior.

Considremos entonces:
\begin{itemize}
\item $p$ es la cantidad de matrices semidefinidas positivas $\Xb$. Por lo tanto, tenemos $p = 1$ y una única matrix $\bar \Xb_0$.
\item El vector $\xb$ es un vector de variables auxiliares, que en este caso corresponde a las $n$ variables $y_0, \dots, y_{n-1}$ del problema.
\item En la función a maximizar no aparecen las coordenadas de $\bar \Xb_0$, por lo tanto tomamos $\bar \Cb_0 = \cero$. 
\item La cantidad de restricciones es $m = \frac{d(d+1)}{2}$.
\item No hay cotas para las variables $x_j$.
\end{itemize}

Por lo tanto, las restricciones corresponden a las coordenadas de la ecuación $\bar \Xb_0 + \sum_{i=1}^{s} y_i \Pb_i = \Db$ el problema dual planteado en el formato de Mosek es
\begin{alignat*}{2}
  & \text{maximizar: } & & \sum_{j=0}^{n-1} b_j y_j  \\
%  & \text{sujeto a: }  \quad & & \Db(i,j) \le \sum_{k=0}^{n-1} \Pb_k(i,j) y_k +  \Eb_{(i,j)} \bar \Xb_0  \le \Db(i,j), \quad 1 \le j \le i \le m, \\
  & \text{sujeto a: }  \quad & & \bar \Xb_0 + \sum_{j=0}^{n-1} y_i \Pb_i = \Db, \\
  & & & \bar \Xb_0 \succeq 0
\end{alignat*}
donde la restricción $\bar \Xb_0 + \sum_{j=0}^{n-1} y_i \Pb_i = \Db $ debemos ingresarla como $\frac{d(d+1)}{2}$ restricciones.

\section{Ejemplo}
Para calcular el mínimo autovalor de una matriz $\Ab \in \Sym^n$, planteamos el problema
\begin{alignat*}{2}
  & \text{maximizar: } & & \alpha \\
  & \text{sujeto a: } & & \Ab - \alpha \Ib  \succeq 0.
\end{alignat*}

Tomamos como ejemplo
$$ \Ab = \begin{pmatrix}
3 & 2 \\
2 & 1
\end{pmatrix}.
$$

Renombrando $\alpha$ como $y_0$, la función a maximizar es $y_0$. La ecuación $\Xb + y_0 \Ib = \Ab$ es
$$
\begin{pmatrix} x_{11} & x_{12} \\ x_{12} & x_{22} \end{pmatrix} + y_0 \begin{pmatrix}
1 & 0 \\
0 & 1
\end{pmatrix} = \begin{pmatrix}
3 & 2 \\
2 & 1
\end{pmatrix},
$$
que nos da las ecuaciones:
\begin{align*}
x_{11} + y_0 &= 3 \\
x_{21} &= 2 \\
x_{22} + y_0 &= 1
\end{align*}

\section{Implementación}

Veamos cómo implementar el problema dual en Mosek. Debemos ingresar:
\begin{enumerate}
\item la dimensión $d$ de la matrix $\bar \Xb_0$,
\item la matrix $\bar \Cb_0 = \cero$,
\item la cantidad $n$ de variables auxiliares $y_j$ (que en la notación de Mosek son las variables $x_j$),
\item los coeficientes $c_j$ de las variables auxiliares $y_j$ en la función objetivo,
\item los coeficientes $a_{ij}$ de las variables $y_j$ en las restricciones, $i = 0, \dots, \frac{d(d+1)}{2}-1$,
\item las matrices $\bar \Ab_{i0}, i = 0, \dots, \frac{d(d+1)}{2}-1$ de las restricciones,
\item las cotas $l_i^c = u_i^c = b_i$, $i = 0, \dots, \frac{d(d+1)}{2}-1$
\item por último si buscamos el máximo o el mínimo de la función objetivo.
\end{enumerate}

\subsection{Dimension $d$ de la matrix $\bar \Xb_0$}

Definimos una variable \texttt{BARVARDIM} como una lista de los tamaños de las matrices $\bar \Xb_i$. Como en nuestro caso, hay una sola matriz, ingresamos
\begin{lstlisting}
BARVARDIM = [d]
\end{lstlisting}
reemplazando $d$ por el valor correspondiente

Agregamos al problema las variables de optimización correspondientes a las coordenadas de las matrices con el comando \texttt{appendbarvars}:

\begin{lstlisting}
task.appendbarvars(BARVARDIM)
\end{lstlisting}


\subsection{Matriz $\bar \Cb_0$ de coeficientes}

Como tenemos $\bar \Cb_0 = \cero$, no necesitamos ingresar nada en este paso.

\subsection{Variables auxiliares $y_j, 1 \le j \le n$ y cotas de estas variables}

Definimos la cantidad de variables auxiliares:
\begin{lstlisting}
numvar = n
\end{lstlisting}
(reemplazando $n$ por la cantidad correcta) y las agregamos al problema:
\begin{lstlisting}
task.appendvars(numvar)
\end{lstlisting}

Podemos fijar cotas para estas variables. Como en el problema dual, las variables $y_j$ no están acotadas, ingresamos
\begin{lstlisting}
for j in range(numvar):
    # Set the bounds on variable j
    # blx[j] <= x_j <= bux[j]
    task.putvarbound(j, mosek.boundkey.fr, -inf, +inf)
\end{lstlisting}

\subsection{Coeficientes $c_j$ de las variables auxiliares $y_j$ en la función objetivo}
Para cada variable auxiliar $y_j$ que aparece en la función objetivo, indicamos el coeficiente $c_j$ mediante el comando \texttt{putcj}:
\begin{lstlisting}
task.putcj(j, c_j)
\end{lstlisting}

En el ejemplo, la única variable auxiliar en la función objetivo es $y_0$ con coeficiente 1. Ingresamos
\begin{lstlisting}
task.putcj(0, 1.0)
\end{lstlisting}

\subsection{Matrices $\bar \Ab_i$ de restricciones}

Ingresamos las matrices $\bar \Ab_i$ también en forma esparsa, ingresando las coordenadas de la parte triangular inferior de la matriz. Si hay $m$ matrices $\bar \Ab_i$, definimos tres listas de longitud $m$, donde el elemento $i$-ésimo de la lista es una lista de valores correspondiente a la definición esparsa de la matriz $\bar \Ab_i$:
\begin{itemize}
\item \texttt{barai}, contiene las coordenadas $i$ de las casillas no nulas de las matrices $\bar \Ab_i$, $0 \le i \le m-1$.
\item \texttt{baraj}, contiene las coordenadas $j$ de las casillas no nulas de las matrices $\bar \Ab_i$, $0 \le i \le m-1$.
\item \texttt{barcval}, contiene los valores de las coordenadas $a_{ij}$ de las matrices $\bar \Ab_i$, $0 \le i \le m-1$, para los índices definidos en \texttt{barai} y \texttt{baraj}.
\end{itemize}

\textbf{Ejemplo.} Para las ecuaciones
\begin{align*}
x_{11} + y_0 &= 3 \\
x_{21} &= 2 \\
x_{22} + y_0 &= 1
\end{align*}
las matrices son 
$$\bar \Ab_0 = \begin{pmatrix} 1 & 0 \\ 0 & 0 \end{pmatrix}, \quad \bar \Ab_1 = \begin{pmatrix} 0 & 0.5 \\ 0.5 & 0 \end{pmatrix} \text{ y } \bar \Ab_2 = \begin{pmatrix} 0 & 0 \\ 0 & 1 \end{pmatrix}.$$

Ingresamos tres listas en cada variable, la $i$-ésima lista en cada variable corresponde a la matriz$\Ab_i$:
\begin{lstlisting}
barai   = [[0],   [1],   [1]  ]
baraj   = [[0],   [0],   [1]  ]
baraval = [[1.0], [0.5], [1.0]]
\end{lstlisting}

Cargamos las matrices $\bar \Ab_i$, las cargamos mediante los comandos
\begin{lstlisting}
numcon = len(barai)
task.appendcons(numcon)

syma = []
for i in range(len(barai)):
    syma.append(task.appendsparsesymmat(BARVARDIM[0],
                                        barai[i],
                                        baraj[i],
                                        baraval[i]))
    task.putbaraij(i, 0, [syma[i]], [1.0])
\end{lstlisting}


\subsection{Coeficientes $a_{ij}$ de la variable $y_j$ en la restricción $i$}
Definimos dos listas \texttt{asub} y \texttt{aval}. Cada lista contiene $m$ listas, la $i$-ésima lista corresponde a las variables de la $i$-ésima restricción:
\begin{itemize}
\item en la $i$-ésima lista de \texttt{asub} ingresamos los índices de las variables auxiliares que aparecen en la $i$-ésima restricción,
\item en la $i$-ésima lista de \texttt{aval} ingresamos los coeficientes de las respectivas variables auxiliares.
\end{itemize}

En el \textbf{ejemplo} aparece la variable $y_0$ en las restricciones $i=0$ y $i=2$, con coeficiente 1 en ambas ecuaciones.
Ingresamos:
\begin{lstlisting}
# Below is the sparse representation of the A
# matrix stored by row.
asub = [[0]  , [ ], [0]  ]
aval = [[1.0], [ ], [1.0]]
\end{lstlisting}

Cargamos estos coeficientes en el problema:
\begin{lstlisting}
for i in range(numcon):
    # Input row i of A
    task.putarow(i,                  # Constraint (row) index.
                 asub[i],            # Column index of non-zeros in constraint i.
                 aval[i])            # Non-zero values of row i.
\end{lstlisting}


\subsection{Cotas  $l_i^c = u_i^c = b_i$, $i = 0, \dots, m-1$.}

Para definir las cotas en las restricciones 
$$l_i^c \le \sum_{j=0}^{n-1} a_{ij} x_j + \sum_{j=0}^{p-1} \innerTrace{\bar \Ab_{ij}}{\bar \Xb_j} \le u_i^c, \quad i = 0, \dots, m-1,$$
definimos tres listas de longitud $m$:
\begin{itemize}
\item \texttt{bkc}: lista de constantes, la $i$-ésima constante indica el tipo de cota correspondiente a la $i$-ésima restricción. Para restricciones de igualdad, utilizamos la constante \texttt{mosek.boundkey.fx}.
\item \texttt{blc}: lista de cotas inferiores, el $i$-ésimo valor de la lista indica la cota inferior de la $i$-ésima restricción. Para restricciones de igualdad, ingresamos el valor $b_i$.
\item \texttt{buc}: lista de cotas superiores, el $i$-ésimo valor de la lista indica la cota superior de la $i$-ésima restricción. Para restricciones de igualdad, ingresamos el valor $b_i$.
\end{itemize}

\textbf{Ejemplo.} Para las restricciones
\begin{align*}
x_{11} + y_0 &= 3 \\
x_{21} &= 2 \\
x_{22} + y_0 &= 1
\end{align*}
ingresamos
\begin{lstlisting}
# Bound keys for constraints
bkc = [mosek.boundkey.fx, mosek.boundkey.fx, mosek.boundkey.fx]

# Bound values for constraints
blc = [3.0, 2.0, 1.0]
buc = [3.0, 2.0, 1.0]
\end{lstlisting}

Y cargamos todas las variables en el problema
\begin{lstlisting}
for i in range(numcon):
    # Set the bounds on constraints.
    # blc[i] <= constraint_i <= buc[i]
    task.putconbound(i, bkc[i], blc[i], buc[i])
\end{lstlisting}

\subsection{Función objetivo}

Finalmente, indicamos el sentido de la función objetivo (\texttt{maximize} o \texttt{minimize}):
\begin{lstlisting}
task.putobjsense(mosek.objsense.minimize)
\end{lstlisting}

Con todos estos datos ingresado, ya podemos correr la optimización.





Vamos a resolver el siguiente ejemplo:
\begin{alignat*}{2}
  & \text{minimizar: } & & x_{11} \\
  & \text{sujeto a: } & & x_{00} = 1 \\
  & & & x_{01} = 3 \\
  & & & \begin{pmatrix} x_{00} & x_{01} \\ x_{01} & x_{11} \end{pmatrix} \succeq 0.
\end{alignat*}
(numeramos filas y columnas a partir de 0 para compatbilidad con Python).

Comparando con la forma general del problema SDP en Mosek descripto en \url{https://docs.mosek.com/latest/pythonapi/tutorial-sdo-shared.html}, tenemos:

\begin{enumerate}
\item $p = 1$, hay una sola matriz $\Xb$ sobre la que tenemos la condición $\Xb \succeq 0$.
\item Por lo tanto, el único valor de $j$ es $j = 0$ y $\bar \Xb_0 = \begin{pmatrix} x_{00} & x_{01} \\ x_{01} & x_{11} \end{pmatrix}$.
\item $r_0 = 2$, es la dimensión de la matrix $\Xb_0$.
\item La función a minimizar es $0 x_{00} + 0 x_{01} + 1 x_{11}$, que con la notación $\langle \Ab, \Bb \rangle = \sum \sum a_{ij} b_{ij}$, corresponde a
    $$
    \begin{pmatrix} 0 & 0 \\ 0 & 1 \end{pmatrix} \begin{pmatrix} x_{00} & x_{01} \\ x_{01} & x_{11} \end{pmatrix},
    $$
    por lo tanto, $\bar \Cb_0 = \begin{pmatrix} 0 & 0 \\ 0 & 1 \end{pmatrix}$.
\item No hay variables adicionales $x_j$ en la función a minimizar, por lo tanto, $n = 0$.
\item Tenemos dos restricciones sobre los coeficientes de $\Xb$, por lo tanto $m = 2$ y  las dos restricciones van indexadas con la variable $i = 0, 1$.
\item La restricción $x_{00} = 1$ la escribimos como
$$
\begin{pmatrix} 1 & 0 \\ 0 & 0 \end{pmatrix}
\begin{pmatrix} x_{00} & x_{01} \\ x_{01} & x_{11} \end{pmatrix} = 1
$$
y la restricción $x_{01} = 3$ la escribimos como
$$
\begin{pmatrix} 0 & 1/2 \\ 1/2 & 0 \end{pmatrix}
\begin{pmatrix} x_{00} & x_{01} \\ x_{01} & x_{11} \end{pmatrix} = 3
$$
(observar que todas las matrices deben ser simétrica).
Obtenemos
$$
\bar \Ab_{00} = \begin{pmatrix} 1 & 0 \\ 0 & 0 \end{pmatrix} \quad \text { y }
\bar \Ab_{10} = \begin{pmatrix} 0 & 1/2 \\ 1/2 & 0 \end{pmatrix}.
$$
\end{enumerate}

Ahora estamos en condiciones de escribir el programa en Mosek, reemplazando el código correspondiente en el primer ejemplo de \url{https://docs.mosek.com/latest/pythonapi/tutorial-sdo-shared.html}.

\begin{enumerate}
\item Construimos la matriz $\bar C_0$ definiéndola en forma esparsa con tres listas barci, barcj y barcval de la misma longitud $k$. Para cada índice $a$, $0 \le a \le k$, se asigna una coordenada de $C$ por la fórmula
    $$c_{(\texttt{barci[a]}, \texttt{barcj[a]})} = \texttt{barcval[a]}.$$
    En nuestro ejemplo hay un solo coeficiente no nulo en $\bar C_0$: $c_{11} = 1$ y por lo tanto definimos
\begin{lstlisting}
barci = [1]
barcj = [1]
barcval = [1.0]
\end{lstlisting}

\item De la misma forma definimos las matrices $\bar A_{i0}$. En este caso como son varias matrices, las definimos mediante listas de listas. Cada lista indica los coeficientes de una matriz. Obtenemos
\begin{lstlisting}
barai = [[0], [1]]
baraj = [[0], [0]]
baraval = [[1.0], [0.5]]
\end{lstlisting}

Notar que solo definimos las matrices para las coordenadas $i \ge j$, y las otras coordenadas quedan definidas por simetr\'ia (si definimos alguna coordenada $j > i$ nos tira error).

\item Ahora definimos las restricciones que corresponden a cada matriz $\Ab_i$. En nuestro caso tenemos dos restricciones de igualdad, por lo tanto definimos
\begin{lstlisting}
bkc = [mosek.boundkey.fx, mosek.boundkey.fx]
\end{lstlisting}

En esta definici\'on podemos modificar fx por otras restricciones, ver la tabla
\url{https://docs.mosek.com/latest/javaapi/tutorial-lo-shared.html#doc-optimizer-tab-boundkeys2}.

\item Y fijamos las restricciones
\begin{lstlisting}
blc = [1.0, 3.0]
buc = [1.0, 3.0]
\end{lstlisting}
que nos indica
$$\texttt{blc[i]} \le \langle \Ab_{i0}, \Xb_{0} \rangle \le \texttt{buc[i]}, i = 0,1.$$

\item Definimos los tamaños:
\begin{lstlisting}
numcon = len(bkc)
BARVARDIM = [2]
\end{lstlisting}
donde $numcon$ es la cantidad $m$ de restricciones y $BARVARDIM$ es una lista con los tamaños de la matrice $\Xb_j$, que en nuestro caso es una sola matriz de tamaño 2.

\item Finalmente, cargamos todas las matrices y restricciones en las variables apropiadas

\begin{lstlisting}
task.appendcons(numcon)

# Append one symmetric variables of dimension 3 (3x3 matrix)
task.appendbarvars(BARVARDIM)

symc = task.appendsparsesymmat(BARVARDIM[0],
                    barci,
                    barcj,
                    barcval)

syma0 = task.appendsparsesymmat(BARVARDIM[0],
                              barai[0],
                              baraj[0],
                              baraval[0])

syma1 = task.appendsparsesymmat(BARVARDIM[0],
                              barai[1],
                              baraj[1],
                              baraval[1])

task.putbarcj(0, [symc], [1.0])
task.putbaraij(0, 0, [syma0], [1.0])
task.putbaraij(1, 0, [syma1], [1.0])

for i in range(numcon):
  # Set the bounds on constraints.
  # blc[i] <= constraint_i <= buc[i]
  task.putconbound(i, bkc[i], blc[i], buc[i])
\end{lstlisting}

En nuestro ejemplo no hay variables $x_j$ por lo tanto fijamos $\texttt{numvar = 0}$ y todo lo que tiene que ver con $\texttt{numvar}$, $\texttt{aval}$, $\texttt{asub}$ lo eliminamos.

\item Y especificamos que queremos minimizar la función objetivo.

\begin{lstlisting}
# Input the objective sense (minimize/maximize)
task.putobjsense(mosek.objsense.minimize)
\end{lstlisting}

\end{enumerate}










\section{Ejemplo 2}

Calcular el menor autovalor de
$$\Ab =
\begin{pmatrix}
6 & 2 & 2 \\
2 & 4 & 2 \\
2 & 2 & 2
\end{pmatrix}
$$
resolviendo el problema SDP:
\begin{alignat*}{2}
  & \text{maximizar: } & & \alpha \\
  & \text{sujeto a: } & & \Ab - \alpha \Ib  \succeq 0.
\end{alignat*}


En este caso tenemos
$$
\Xb =
\begin{pmatrix}
x_{00} & x_{01} & x_{02} \\
x_{01} & x_{11} & x_{12} \\
x_{02} & x_{12} & x_{22}
\end{pmatrix} = \begin{pmatrix}
6 & 2 & 2 \\
2 & 4 & 2 \\
2 & 2 & 2
\end{pmatrix}
- \begin{pmatrix}
\alpha & 0 & 0 \\
0 & \alpha & 0 \\
0 & 0 & \alpha
\end{pmatrix}
$$

Por lo tanto en el problema modelo de \url{https://docs.mosek.com/latest/pythonapi/tutorial-sdo-shared.html}, tomamos $n=1$ y $x_0 = \alpha$, siendo $x_0$ la función a maximizar.



\end{document}


\textbf{Para entregar:} entregar un ejercicio a elección entre los marcados ($\diamondsuit$).

\vspace{0.5cm}

\noindent \textbf{Primeros pasos en Mosek}

\begin{enumerate}

\item (Mosek) Calcular el menor autovalor de la matriz
$$
\Ab = \begin{pmatrix}
3 & 0 & 6 \\
0 & 0 & 1 \\
6 & 1 & 1
\end{pmatrix}
$$
resolviendo en Mosek el problema SDP
\begin{alignat*}{2}
  & \text{maximizar: } & & \eta \\
  & \text{sujeto a: } & \quad & \Ab - \eta I  \succeq 0.
\end{alignat*}

\item (Mosek)
Vimos en la primera clase que para determinar si
$$f(x) = 10x^4+2x^3+27x^2-24x+5 \ge 0$$
 para todo $x \in \R$ tenemos que determinar si existe $\Ab \in \R^{3 \times 3}$ tal que
\begin{align*}
&a_{11} = 5, \quad 2a_{12} = -24, \quad 2a_{13} + a_{22} = 27, \quad 2a_{23} = 2, \quad a_{33} = 10, \\
&\Ab \succeq 0.
\end{align*}

Resolver en Mosek este problema SDP, calculando el máximo de $a_{22}$ sujeto a las restricciones dadas.

\end{enumerate}

\noindent \textbf{Matrices simétricas}
\begin{enumerate}[resume]

\item
Dada una matriz simétrica $\Ab \in \Rnn$, probar
\begin{enumerate}
\item $\Ab^n$ es simétrica para todo $n \in \N$.
\item $\Ab^{-1}$ es simétrica.
\item Si $\Bb \in \Rnn$ es simétrica, $\Ab\Bb$ es simétrica si y solo si $\Ab$ y $\Bb$ conmutan ($\Ab\Bb = \Bb\Ab$).
\end{enumerate}

\item Para el producto interno usual $\inner{\xb}{\yb}$ de $\Rnn$, probar que una matriz $\Ab \in \Rnn$ es simétrica si y solo si
$$
\inner{\Ab \xb}{\yb} = \inner{\xb}{\Ab \yb}
$$
para todo $\xb, \yb \in \R^n$.

\item Dada $\Ab \in \Rnn$ simétrica con $\{\ub_1, \dots, \ub_n\}$ base ortonormal de autovectores correspondiente a los autovalores $\{\lambda_1, \dots, \lambda_n\}$, probar que
$$
\Ab = \sum_{i=1}^n \lambda_i \ub_i \ub_i^T.
$$

\item (Resolver a mano o en Python) Para la matriz
$$
\Ab = \begin{pmatrix}
3 & 2 & 4 \\
2 & 0 & 2 \\
4 & 2 & 0
\end{pmatrix},
$$
\begin{enumerate}
\item Hallar una factorización $\Ab = \Ub \Db \Ub^T$ con $\Ub$ ortogonal y $\Db$ diagonal.
\item Hallar mediante diagonalización simultánea de filas y columnas una matriz $\Bb \in \Q^{3 \times 3}$ diagonal congruente a $\Ab$. Determinar una matriz $\Sb$ tal que $\Sb \Ab \Sb^T = \Bb$.
\item Verificar que las matrices $\Db$ y $\Bb$ halladas tienen la misma signatura.
\item Escribir a la matriz $\Ab$ como suma de matrices simétricas de rango 1.
\end{enumerate}



\end{enumerate}

\noindent \textbf{Matrices definidas positivas}

\begin{enumerate}[resume]
\item
Dadas matrices simétricas $\Ab \in \Rn$, $\Bb \in \Rm$, definimos la matriz $\Ab \oplus \Bb$ por
$$
\Ab \oplus \Bb =
\begin{pmatrix}
\Ab & 0 \\
0 & \Bb
\end{pmatrix}.
$$

Probar que $\Ab \succeq 0$ y $\Bb \succeq 0$ si y solo si $\Ab \oplus \Bb \succeq 0$.

\item Si $\Ab \in \Rn$ es semidefinida positiva y $a_{ii} = 0$, probar que $a_{ij} = 0 = a_{ji}$ para todo $j = 1, \dots, n$.

\item Probar que una matriz semidefinida positiva $\Ab \in \Rnn$ tiene rango 1 si y solo si
$$\Ab = \xb \xb^T$$
para algún vector $\xb \in \Rn$.

\item Sea $\Ab \in \Rn$ una matrix semidefinida positiva y $\xb \in \Rn$. Probar que
$$
\xb^T \Ab \xb = 0 \iff \Ab \xb = 0.
$$

\item (Complemento de Schur.) Sea $\Xb$ una matriz simétrica definida por bloques
$$
\Xb = \begin{pmatrix}
\Ab & \Bb \\
\Bb^T & \Cb
\end{pmatrix}
$$
con $\Ab \in \R^{m \times m}$, $\Bb \in \R^{m \times n}$ y $\Cb \in \Rnn$, con $\Ab$ no-singular. Probar que
$$
\Xb \succeq 0 \iff  \Cb - \Bb^T \Ab^{-1} \Bb \succeq 0.
$$

Sugerencia: probar que
$$
\Xb = \Pb^T \begin{pmatrix}
\Ab & 0 \\
0 & \Cb - \Bb^T \Ab^{-1} \Bb
\end{pmatrix}, \text{ con }
\Pb = \begin{pmatrix}
\Ib & \Ab^{-1} \Bb \\
0 & \Ib
\end{pmatrix}.
$$

\item Para las matrices
$$
\Ab = \begin{pmatrix}
13 & 1 & 6 \\
 1 & 2 & 2 \\
 6 & 2 & 4
 \end{pmatrix},
\quad \quad
\Bb = \begin{pmatrix}
3 & 2 & 4 \\
2 & 0 & 2 \\
4 & 2 & 3
\end{pmatrix},
\quad \quad
\Cb = \begin{pmatrix}
 9 & -3 & 6 \\
 -3 &  2 & -2 \\
 6 & -2 & 5
 \end{pmatrix}
 $$
\begin{enumerate}
\item Determinar cuáles son semidefinidas positivas y encontrar una factorización $\Mb \Mb^t$ para las que lo sean.
\item Determinar cuáles son definidas positivas y calcular la factorización de Cholesky para las que lo sean.
\end{enumerate}

\end{enumerate}
\pagebreak

\noindent \textbf{Conos y espectrahedros}
\begin{enumerate}[resume]
\item Probar que el conjunto
$$
\mathcal{L}^{n+1} = \left\{ (\xb, t) \in \Rn \times \R : \|x\|_2 =\sqrt{x_1^2 + \dots + x_n^2} \le t \right\}
$$
es un cono y determinar si es puntiagudo.

Realizar un gráfico aproximado de $\mathcal{L}^{3}$.

\item Graficar en $\R^2$ el espectrahedro dado por
$$S =
\left\{
(y_1, y_2) \in \R^2 \mid
\begin{pmatrix}
0 & 0 \\
0 & 1
\end{pmatrix}
+
y_1 \begin{pmatrix}
1 & 0 \\
0 & -1
\end{pmatrix}
+
y_2 \begin{pmatrix}
0 & 1 \\
1 & 0
\end{pmatrix}
\succeq 0 \right\}
$$

\item Graficar en $\R^2$ el espectrahedro dado por
$$S =
\left\{
(x, y) \in \R^2 \mid
\begin{pmatrix}
 x & 1 \\ 1 & y
\end{pmatrix} \succeq 0\right\}.
$$

\item ¿A qué objeto geométrico corresponde el espectrahedro en $\R^3$ dado por
$$S =
\left\{
(x, y, z) \in \R^3 \mid
\begin{pmatrix}
 1 + x & y & 0 & 0  \\
 y & 1-x & 0 & 0 \\
 0 & 0 & 1+z & 0 \\
 0 & 0 & 0 & 1-z
\end{pmatrix} \succeq 0\right\}?
$$

\item Ingresar el siguiente código en Python para realizar el gráfico de una curva dada en forma implícita.
\begin{lstlisting}
from sympy import var, plot_implicit
var('x y')
plot_implicit(x**2 + x**3 - y**2)
\end{lstlisting}

\item ($\diamondsuit$) Considerar el espectrahedro en $\R^2$ dado por
$$S =
\left\{
(x,y) \in \R^2 \mid \Ab(x,y) = \begin{pmatrix}
x+1 & 0 & y \\
0 & 2 & -x-1 \\
y & -x-1 2
\end{pmatrix}\succeq 0
\right\}
$$

\begin{enumerate}
\item Calcular el determinante de $\Ab(x,y)$ y el polinomio característico.
\item Graficar en Python las soluciones de $\det(\Ab(x,y)) = 0$.
\item Determinar el gráfico del espectrahedro $S$.
\end{enumerate}

\item ($\diamondsuit$) Graficar (con la ayuda de Python) el espectrahdro en $\R^2$ dado por
$$S =
\left\{
(x,y) \in \R^2 \mid \Ab(x,y) = \begin{pmatrix}
1 & x & x+y \\
x & 1 & y \\
x+y & y & 1
\end{pmatrix} \succeq 0
\right\}
$$

\end{enumerate}

\noindent \textbf{Problemas de programación semidefinida}

\begin{enumerate}[resume]

\item Resolver (a mano) el problema
\begin{alignat*}{2}
  & \text{minimizar: } & & x_{11} \\
  & \text{sujeto a: } & \quad &
  \begin{pmatrix} x_{11} & 1 \\ 1 & x_{22} \end{pmatrix}.
\end{alignat*}

¿Se alcanza el ínfimo hallado?

\pagebreak
\item ($\diamondsuit$) Resolver (a mano o en Mosek) los problemas SDP
\begin{multicols}{2}
\noindent
\begin{alignat*}{2}
  & \text{minimizar: } & & y \\
  & \text{sujeto a: } & \quad &
\begin{pmatrix}
5 & -12 & y \\
-12 & 27 - 2y & 1 \\
y & 1 & 10 \\
\end{pmatrix}
\end{alignat*}
\begin{alignat*}{2}
  & \text{maximizar: } & & y \\
  & \text{sujeto a: } & \quad &
\begin{pmatrix}
5 & -12 & y \\
-12 & 27 - 2y & 1 \\
y & 1 & 10 \\
\end{pmatrix}
\end{alignat*}
\end{multicols}

A partir de los resultados hallados, determinar el espectrahedro del conjunto factible.

\item ($\diamondsuit$) Para el problema SDP primal:
\begin{alignat*}{2}
  & \text{minimizar: } & & 2x_{11} + 2x_{12} \\
  & \text{sujeto a: } & \quad & x_{11} + x_{22} = 1, \\
  & & & \begin{pmatrix} x_{11} & x_{12} \\ x_{12} & x_{22}  \end{pmatrix} \succeq 0.
\end{alignat*}
\begin{enumerate}
\item Resolver el problema.
\item Plantear el problema dual.
\item Resolver el problema dual y calcular el salto de dualidad.
\end{enumerate}

\item Resolver el par de problemas SDP primal/dual y calcular el salto de dualidad.
\begin{multicols}{2}
\noindent
\begin{alignat*}{2}
  & \text{minimizar: } & & \alpha x_{11} \\
  & \text{sujeto a: } & \quad & x_{22} = 0, \\
  & & \quad & x_{11} + 2x_{23} = 1, \\
  & & \quad & \Xb \in \R^{3 \times 3} \succeq 0.
\end{alignat*}
\begin{alignat*}{2}
  & \text{maximizar: } & & y_2 \\
  & \text{sujeto a: } & & \begin{pmatrix} y_2 & 0 & 0 \\ 0 & y_1 & y_2 \\ 0 & y_2 & 0  \end{pmatrix} \preceq \begin{pmatrix} \alpha & 0 & 0 \\ 0 & 0 & 0 \\ 0 & 0 & 0 \end{pmatrix}.
\end{alignat*}
\end{multicols}

\item Considerar el par de problemas SDP primal/dual:
\begin{multicols}{2}
\noindent
\begin{alignat*}{2}
  & \text{minimizar: } & & x_{11} \\
  & \text{sujeto a: } & \quad & 2x_{12} = 1, \\
  & & & \begin{pmatrix} x_{11} & x_{12} \\ x_{12} & x_{22}  \end{pmatrix} \succeq 0.
\end{alignat*}
\begin{alignat*}{2}
  & \text{maximizar: } & & y \\
  & \text{sujeto a: } & & \begin{pmatrix} 0 & y \\ y & 0  \end{pmatrix} \succeq \begin{pmatrix} 1 & 0 \\ 0 & 0  \end{pmatrix}.
\end{alignat*}
\end{multicols}

\begin{enumerate}
\item Resolver ambos problemas y calcular el salto de dualidad.
\item ¿Se alcanza el óptimo en ambos problemas?
\end{enumerate}

\end{enumerate}

\end{document}
