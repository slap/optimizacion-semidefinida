\documentclass[11pt]{article}
%\include{amsfonts}
%\usepackage{amssymb,amsmath,latexsym,epsfig,euscript}
\usepackage{amssymb}
\usepackage{enumitem}
\usepackage{amsmath}
\usepackage{multicol}

\usepackage{bm}

\def\C{\mathbb{C}}
\def \N{\mathbb{N}}
\def \Q{\mathbb{Q}}
\def \R{\mathbb{R}}
\def \Z{\mathbb{Z}}

\newcommand{\inner}[2]{#1 \boldsymbol{\cdot} #2}
\newcommand{\innerTrace}[2]{#1 \bullet #2}

\newcommand{\vb}{\bm{v}}
\newcommand{\ub}{\bm{u}}
\newcommand{\wb}{\bm{w}}
\newcommand{\bb}{\bm{b}}
\newcommand{\xb}{\bm{x}}
\newcommand{\yb}{\bm{y}}
\newcommand{\zb}{\bm{z}}
\newcommand{\eb}{\bm{e}}
\newcommand{\ab}{\bm{a}}
\newcommand{\cb}{\bm{c}}
\newcommand{\db}{\bm{d}}
\newcommand{\pb}{\bm{p}}
\newcommand{\qb}{\bm{q}}
\newcommand{\Ab}{\bm{A}}
\newcommand{\Bb}{\bm{B}}
\newcommand{\Cb}{\bm{C}}
\newcommand{\Eb}{\bm{E}}
\newcommand{\Fb}{\bm{F}}
\newcommand{\Gb}{\bm{G}}
\newcommand{\Lb}{\bm{L}}
\newcommand{\Mb}{\bm{M}}
\newcommand{\Ub}{\bm{U}}
\newcommand{\Qb}{\bm{Q}}
\newcommand{\Rb}{\bm{R}}
\newcommand{\Sb}{\bm{S}}
\newcommand{\Vb}{\bm{V}}
\newcommand{\Db}{\bm{D}}
\newcommand{\Pb}{\bm{P}}
\newcommand{\Ib}{\bm{I}}
\newcommand{\Hb}{\bm{H}}
\newcommand{\Xb}{\bm{X}}
\newcommand{\cero}{\bm{0}}

%
\def\d{\displaystyle}
%
\usepackage[heightrounded]{geometry}
\geometry{top=2cm,bottom=2cm,left=2cm,right=2cm}
\parindent=0pt

\begin{document}

\begin{center}

{\small Facultad de Ciencias Exactas
y Naturales -- Universidad de Buenos Aires } \vskip 1cm

\textbf{{\large Optimización Semidefinida} - Segundo Cuatrimestre 2021}

\medskip\textbf{Pr\'actica 1 - Programación Lineal}
\end{center}

\medskip

\textbf{Para entregar:} entregar un ejercicio a elección entre los marcados ($\diamondsuit$) y un ejercicio a elección entre los marcados ($\clubsuit$).

\begin{enumerate}

\item (Python) Resolver utilizando el comando \texttt{linalg.solve} de \texttt{numpy} el sistema de ecuaciones
$$\left\{
\begin{array}{ccc}
x_1+x_2-2x_3 &=& -2 \\
3x_1-2x_2+x_3 &=& 3 \\
x_1-x_2+x_3 &=& 2 \end{array}\right.
$$

\item (Python) Para el sistema de ecuaciones
$$
\left\{
\begin{array}{ccc}
x_1+x_2-2x_3+x_4 &=& -2 \\ 3x_1-2x_2+x_3+5x_4 &=& 3 \\
x_1-x_2+x_3+2x_4 &=& 2 \end{array}\right.
$$
hallar en Python una solución particular del sistema, y generadores del espacio de soluciones del sistema homogéneo. ¿Cuáles son todas las soluciones del sistema?

\item \label{ej:metodografico} ($\diamondsuit$) Resolver por el método gráfico
\begin{alignat*}{2}
  & \text{maximizar: } & & z = 3x + 2y \\
   & \text{sujeto a: } & \quad & x + 2y \le 4, \\
   & & & x-y \le 1, \\
   & & & x \ge 0, \\
   & & & y \ge 0.
\end{alignat*}

\item  ($\diamondsuit$) Considerar el problema
\begin{alignat*}{2}
  & \text{minimizar: } & & z = 5x + 7y \\
   & \text{sujeto a: } & \quad & 2x + 3y \ge 6, \\
   & & & 3x-y \le 15, \\
   & & & -x + 4y \le 4, \\
   & & & 2x + 5y \le 27, \\
   & & & x \ge 0, \\
   & & & y \ge 0.
\end{alignat*}
\begin{enumerate}
\item Graficar el conjunto factible.
\item Hallar las coordenadas de todos los vértices del conjunto.
\item Evaluar la función objetivo en cada uno de los vértices.
\item ¿Cuál es la solución del problema?
\end{enumerate}

\item  ($\diamondsuit$) La parte líquida de una dieta debe proveer por día al menos 300 calorías, 36 unidades de vitamina A y 90 unidades de vitamina C.
Un vaso de una bebida dietética X provee 60 calorías, 12 unidades de vitamina A y 10 unidades de vitamina C.
Un vaso de una bebida dietética Y provee 60 calories, 6 unidades de vitamin A y 30 unidades de vitamina C.
La bebida X tiene un costo de \$12 por vaso y la bebida Y tiene un costo de \$15 por vaso. ¿Cuántos vasos de cada bebida deben tomarse por día si se quiere minimizar el costo total y cumplir con todos los requerimientos de la dieta?


\item \label{ej:primal} Hallar vectores $\cb$ y $\bb$ y una matriz $\Ab$ tales que el problema del Ejercicio \ref{ej:metodografico} quede planteado de la forma
\begin{alignat*}{2}
  & \text{maximizar: } & \quad & \inner{\cb}{\xb} \\
   & \text{sujeto a: } & & \Ab \xb \le \bb, \\
   & & & \xb \ge 0,
\end{alignat*}
para $\xb = (x, y)$.


\item ($\clubsuit$) Escribir un programa que dada una matriz $\Ab \in \R^{m \times n}$, un vector $\bb \in \R^m$ y un vector $\xb \in \R^n$, determine si se satisfacen las condiciones
\begin{align*}
& \Ab \xb \le \bb \\
& \xb \ge 0
\end{align*}

\item ($\clubsuit$) Escribir un programa que dado un vector $\cb \in \R^n$ y una lista de vectores $[\xb_1, \dots, \xb_s]$ en $\R^n$, devuelva el valor máximo de $\inner{\cb}{\xb_i}$, $1 \le i \le s$.


\item Agregando variables de holgura, llevar el problema del Ejercicio \ref{ej:primal} a la forma
\begin{alignat*}{2}
  & \text{maximizar: } & \quad & \inner{\cb}{\xb} \\
   & \text{sujeto a: } & & \Ab \xb = \bb, \\
   & & & \xb \ge 0. \\
\end{alignat*}


\item Dado el problema
\begin{alignat*}{2}
  & \text{maximizar: } & \quad & -2x + 3y - 5z \\
   & \text{sujeto a: } & & 7x - 5y + 6z \le 10 , \\
   & & & -2x + 8y - 4z \le 3, \\
   & & & 9x -2y - 5z \le 4, \\
   & & & y, z \ge 0,
\end{alignat*}
en el cuál no tenemos la restricción $x \ge 0$, ¿cómo podemos plantearlo como un conjunto de problemas en forma estándar? ¿Cómo podemos plantearlo como un único problema en forma estándar? Resolver ambas posibilidades y comparar.


\item ($\clubsuit$) Escribir un programa que reciba $\cb$, $\Ab$ y $\bb$ correspondientes a un problema de programación lineal en forma estándar
\begin{alignat*}{2}
  & \text{maximizar: } & \quad & \inner{\cb}{\xb} \\
   & \text{sujeto a: } & & \Ab \xb = \bb, \\
   & & & \xb \ge 0,
\end{alignat*}
calcule todos los vértices del conjunto factible y determine el valor máximo de la función objetivo sobre los vértices.

\item ($\clubsuit$) Escribir un programa que reciba $\cb$, $\Ab$ y $\bb$ correspondientes a un problema de programación lineal en forma estándar
\begin{alignat*}{2}
  & \text{maximizar: } & \quad & \inner{\cb}{\xb} \\
   & \text{sujeto a: } & & \Ab \xb = \bb, \\
   & & & \xb \ge 0, \\
\end{alignat*}
resuelva el problema utilizando el método Simplex.




\end{enumerate}
\end{document}
