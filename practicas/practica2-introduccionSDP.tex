\documentclass[11pt]{article}
%\include{amsfonts}
%\usepackage{amssymb,amsmath,latexsym,epsfig,euscript}
\usepackage{amssymb}
\usepackage{enumitem}
\usepackage{amsmath}
\usepackage{multicol}

\usepackage{bm}


%
\def\d{\displaystyle}
%
\usepackage[heightrounded]{geometry}
\geometry{top=2cm,bottom=2cm,left=2cm,right=2cm}
\parindent=0pt

\usepackage{graphicx}

\graphicspath{ {../../images} }
\usepackage{listings}
\lstset{
  basicstyle=\ttfamily,
  columns=fullflexible,
}

\usepackage{url}
\usepackage{multicol}
\usepackage{dsfont}

% Bold symbols for vectors and matrices
\newcommand{\xstar}{\bm{x}^{\star}}
\newcommand{\alphab}{\bm{\alpha}}
\newcommand{\ab}{\bm{a}}
\newcommand{\bb}{\bm{b}}
\newcommand{\cb}{\bm{c}}
\newcommand{\db}{\bm{d}}
\newcommand{\eb}{\bm{e}}
\newcommand{\gb}{\bm{g}}
\newcommand{\mb}{\bm{m}}
\newcommand{\pb}{\bm{p}}
\newcommand{\qb}{\bm{q}}
\newcommand{\rb}{\bm{r}}
\newcommand{\ssb}{\bm{s}}
\newcommand{\ub}{\bm{u}}
\newcommand{\vb}{\bm{v}}
\newcommand{\wb}{\bm{w}}
\newcommand{\xb}{\bm{x}}
\newcommand{\yb}{\bm{y}}
\newcommand{\zb}{\bm{z}}

\newcommand{\Ab}{\bm{A}}
\newcommand{\Bb}{\bm{B}}
\newcommand{\Cb}{\bm{C}}
\newcommand{\Db}{\bm{D}}
\newcommand{\Eb}{\bm{E}}
\newcommand{\Fb}{\bm{F}}
\newcommand{\Gb}{\bm{G}}
\newcommand{\Hb}{\bm{H}}
\newcommand{\Ib}{\bm{I}}
\newcommand{\Id}{\bm{I}}
\newcommand{\Kb}{\bm{K}}
\newcommand{\Lb}{\bm{L}}
\newcommand{\Mb}{\bm{M}}
\newcommand{\Pb}{\bm{P}}
\newcommand{\Qb}{\bm{Q}}
\newcommand{\Rb}{\bm{R}}
\newcommand{\Sb}{\bm{S}}
\newcommand{\Tb}{\bm{T}}
\newcommand{\Ub}{\bm{U}}
\newcommand{\Vb}{\bm{V}}
\newcommand{\Wb}{\bm{W}}
\newcommand{\Xb}{\bm{X}}
\newcommand{\Yb}{\bm{Y}}
\newcommand{\Zb}{\bm{Z}}
\newcommand{\Lambdab}{\bm{\Lambda}}
\newcommand{\cero}{\bm{0}}

% Rings and fields
\newcommand{\A}{\mathbb{A}}
\newcommand{\Z}{\mathbb{Z}}
\newcommand{\Q}{\mathbb{Q}}
\newcommand{\C}{\mathbb{C}}
\newcommand{\R}{\mathbb{R}}
\newcommand{\K}{\mathbb{K}}
\newcommand{\N}{\mathbb{N}}

\newcommand{\borel}{{\mathcal B}}
\newcommand{\pmom}{{\rho_{\text{mom}}}}
\newcommand{\MX}{{\mathcal{M}(X)}}


% Inner product
\newcommand{\innerl}[2]{\langle #1, #2 \rangle}
\newcommand{\inner}[2]{#1 \boldsymbol{\cdot} #2}
\newcommand{\innerTrace}[2]{#1 \bullet #2}

% Symmetric and positive definite matrices
\newcommand{\Splusplusn}{{\mathcal S_{++}^n}}
\newcommand{\Splusn}{{\mathcal S_+^n}}
\newcommand{\Splus}{{\mathcal S_+}}
\newcommand{\Sym}{{\mathcal S}}
\newcommand{\Symn}{{\mathcal S^n}}

% Cones
\newcommand\CC{\mathcal{C}}
\DeclareMathOperator{\cone}{cono}
\DeclareMathOperator{\conv}{conv}
\DeclareMathOperator{\supp}{supp}


% Spectrahedron
\newcommand{\eLL}{{\mathcal L}}

% Matrices and vectors over R or C
\newcommand{\Rnn}{\R^{n\times n}}
\newcommand{\Cnn}{\C^{n\times n}}
\newcommand{\Rn}{\R^{n}}
\newcommand{\Rm}{\R^{m}}


% Math operators
\DeclareMathOperator{\Tr}{Tr}
\DeclareMathOperator{\tr}{Tr}
\DeclareMathOperator{\interior}{int}
\DeclareMathOperator{\rank}{rank}
\DeclareMathOperator{\diag}{diag}

\newcommand\one{\mathds{1}} 


\begin{document}

\begin{center}

{\small Facultad de Ciencias Exactas
y Naturales -- Universidad de Buenos Aires } \vskip 1cm

\textbf{{\large Optimización Semidefinida} - Segundo Cuatrimestre 2021}

\medskip\textbf{Pr\'actica 2 - Introducción SDP}
\end{center}

\medskip

\textbf{Para entregar:} entregar un ejercicio a elección entre los marcados ($\diamondsuit$).

\vspace{0.5cm}

\noindent \textbf{Primeros pasos en Mosek}

\begin{enumerate}

\item (Mosek) Calcular el menor autovalor de la matriz
$$
\Ab = \begin{pmatrix}
3 & 0 & 6 \\
0 & 0 & 1 \\
6 & 1 & 1
\end{pmatrix}
$$
resolviendo en Mosek el problema SDP
\begin{alignat*}{2}
  & \text{maximizar: } & & \eta \\
  & \text{sujeto a: } & \quad & \Ab - \eta I  \succeq 0.
\end{alignat*}

\item (Mosek)
Vimos en la primera clase que para determinar si
$$f(x) = 10x^4+2x^3+27x^2-24x+5 \ge 0$$
 para todo $x \in \R$ tenemos que determinar si existe $\Ab \in \R^{3 \times 3}$ tal que
\begin{align*}
&a_{11} = 5, \quad 2a_{12} = -24, \quad 2a_{13} + a_{22} = 27, \quad 2a_{23} = 2, \quad a_{33} = 10, \\
&\Ab \succeq 0.
\end{align*}

Resolver en Mosek este problema SDP, calculando el máximo de $a_{22}$ sujeto a las restricciones dadas.

\end{enumerate}

\noindent \textbf{Matrices simétricas}
\begin{enumerate}[resume]

\item
Dada una matriz simétrica $\Ab \in \Rnn$, probar
\begin{enumerate}
\item $\Ab^n$ es simétrica para todo $n \in \N$.
\item $\Ab^{-1}$ es simétrica.
\item Si $\Bb \in \Rnn$ es simétrica, $\Ab\Bb$ es simétrica si y solo si $\Ab$ y $\Bb$ conmutan ($\Ab\Bb = \Bb\Ab$).
\end{enumerate}

\item Para el producto interno usual $\inner{\xb}{\yb}$ de $\Rnn$, probar que una matriz $\Ab \in \Rnn$ es simétrica si y solo si
$$
\inner{\Ab \xb}{\yb} = \inner{\xb}{\Ab \yb}
$$
para todo $\xb, \yb \in \R^n$.

\item Dada $\Ab \in \Rnn$ simétrica con $\{\ub_1, \dots, \ub_n\}$ base ortonormal de autovectores correspondiente a los autovalores $\{\lambda_1, \dots, \lambda_n\}$, probar que
$$
\Ab = \sum_{i=1}^n \lambda_i \ub_i \ub_i^T.
$$

\item (Resolver a mano o en Python) Para la matriz
$$
\Ab = \begin{pmatrix}
3 & 2 & 4 \\
2 & 0 & 2 \\
4 & 2 & 0
\end{pmatrix},
$$
\begin{enumerate}
\item Hallar una factorización $\Ab = \Ub \Db \Ub^T$ con $\Ub$ ortogonal y $\Db$ diagonal.
\item Hallar mediante diagonalización simultánea de filas y columnas una matriz $\Bb \in \Q^{3 \times 3}$ diagonal congruente a $\Ab$. Determinar una matriz $\Sb$ tal que $\Sb \Ab \Sb^T = \Bb$.
\item Verificar que las matrices $\Db$ y $\Bb$ halladas tienen la misma signatura.
\item Escribir a la matriz $\Ab$ como suma de matrices simétricas de rango 1.
\end{enumerate}



\end{enumerate}

\noindent \textbf{Matrices definidas positivas}

\begin{enumerate}[resume]
\item
Dadas matrices simétricas $\Ab \in \Rn$, $\Bb \in \Rm$, definimos la matriz $\Ab \oplus \Bb$ por
$$
\Ab \oplus \Bb =
\begin{pmatrix}
\Ab & 0 \\
0 & \Bb
\end{pmatrix}.
$$

Probar que $\Ab \succeq 0$ y $\Bb \succeq 0$ si y solo si $\Ab \oplus \Bb \succeq 0$.

\item Si $\Ab \in \Rn$ es semidefinida positiva y $a_{ii} = 0$, probar que $a_{ij} = 0 = a_{ji}$ para todo $j = 1, \dots, n$.

\item Probar que una matriz semidefinida positiva $\Ab \in \Rnn$ tiene rango 1 si y solo si
$$\Ab = \xb \xb^T$$
para algún vector $\xb \in \Rn$.

\item Sea $\Ab \in \Rn$ una matrix semidefinida positiva y $\xb \in \Rn$. Probar que
$$
\xb^T \Ab \xb = 0 \iff \Ab \xb = 0.
$$

\item (Complemento de Schur.) Sea $\Xb$ una matriz simétrica definida por bloques
$$
\Xb = \begin{pmatrix}
\Ab & \Bb \\
\Bb^T & \Cb
\end{pmatrix}
$$
con $\Ab \in \R^{m \times m}$, $\Bb \in \R^{m \times n}$ y $\Cb \in \Rnn$, con $\Ab$ no-singular. Probar que
$$
\Xb \succeq 0 \iff  \Ab \succeq 0 \text{ y } \Cb - \Bb^T \Ab^{-1} \Bb \succeq 0 .
$$

Sugerencia: probar que
$$
\Xb = \Pb^T \begin{pmatrix}
\Ab & 0 \\
0 & \Cb - \Bb^T \Ab^{-1} \Bb
\end{pmatrix} \Pb, \text{ con }
\Pb = \begin{pmatrix}
\Ib & \Ab^{-1} \Bb \\
0 & \Ib
\end{pmatrix}.
$$

\item Para las matrices
$$
\Ab = \begin{pmatrix}
13 & 1 & 6 \\
 1 & 2 & 2 \\
 6 & 2 & 4
 \end{pmatrix},
\quad \quad
\Bb = \begin{pmatrix}
3 & 2 & 4 \\
2 & 0 & 2 \\
4 & 2 & 3
\end{pmatrix},
\quad \quad
\Cb = \begin{pmatrix}
 9 & -3 & 6 \\
 -3 &  2 & -2 \\
 6 & -2 & 5
 \end{pmatrix}
 $$
\begin{enumerate}
\item Determinar cuáles son semidefinidas positivas y encontrar una factorización $\Mb \Mb^t$ para las que lo sean.
\item Determinar cuáles son definidas positivas y calcular la factorización de Cholesky para las que lo sean.
\end{enumerate}

\end{enumerate}
\pagebreak

\noindent \textbf{Conos y espectrahedros}
\begin{enumerate}[resume]
\item Probar que el conjunto
$$
\mathcal{L}^{n+1} = \left\{ (\xb, t) \in \Rn \times \R : \|x\|_2 =\sqrt{x_1^2 + \dots + x_n^2} \le t \right\}
$$
es un cono y determinar si es puntiagudo.

Realizar un gráfico aproximado de $\mathcal{L}^{3}$.

\item Graficar en $\R^2$ el espectrahedro dado por
$$S =
\left\{
(y_1, y_2) \in \R^2 \mid
\begin{pmatrix}
0 & 0 \\
0 & 1
\end{pmatrix}
+
y_1 \begin{pmatrix}
1 & 0 \\
0 & -1
\end{pmatrix}
+
y_2 \begin{pmatrix}
0 & 1 \\
1 & 0
\end{pmatrix}
\succeq 0 \right\}
$$

\item Graficar en $\R^2$ el espectrahedro dado por
$$S =
\left\{
(x, y) \in \R^2 \mid
\begin{pmatrix}
 x & 1 \\ 1 & y
\end{pmatrix} \succeq 0\right\}.
$$

\item ¿A qué objeto geométrico corresponde el espectrahedro en $\R^3$ dado por
$$S =
\left\{
(x, y, z) \in \R^3 \mid
\begin{pmatrix}
 1 + x & y & 0 & 0  \\
 y & 1-x & 0 & 0 \\
 0 & 0 & 1+z & 0 \\
 0 & 0 & 0 & 1-z
\end{pmatrix} \succeq 0\right\}?
$$

\item Ingresar el siguiente código en Python para realizar el gráfico de una curva dada en forma implícita.
\begin{lstlisting}
from sympy import var, plot_implicit
var('x y')
plot_implicit(x**2 + x**3 - y**2)
\end{lstlisting}

\item ($\diamondsuit$) Considerar el espectrahedro en $\R^2$ dado por
$$S =
\left\{
(x,y) \in \R^2 \mid \Ab(x,y) = \begin{pmatrix}
x+1 & 0 & y \\
0 & 2 & -x-1 \\
y & -x-1 & 2
\end{pmatrix}\succeq 0
\right\}
$$

\begin{enumerate}
\item Calcular el determinante de $\Ab(x,y)$ y el polinomio característico.
\item Graficar en Python las soluciones de $\det(\Ab(x,y)) = 0$.
\item Determinar el gráfico del espectrahedro $S$.
\end{enumerate}

\item ($\diamondsuit$) Graficar (con la ayuda de Python) el espectrahdro en $\R^2$ dado por
$$S =
\left\{
(x,y) \in \R^2 \mid \Ab(x,y) = \begin{pmatrix}
1 & x & x+y \\
x & 1 & y \\
x+y & y & 1
\end{pmatrix} \succeq 0
\right\}
$$

\end{enumerate}

\noindent \textbf{Problemas de programación semidefinida}

\begin{enumerate}[resume]

\item Resolver (a mano) el problema
\begin{alignat*}{2}
  & \text{minimizar: } & & x_{11} \\
  & \text{sujeto a: } & \quad &
  \begin{pmatrix} x_{11} & 1 \\ 1 & x_{22} \end{pmatrix} \succeq 0.
\end{alignat*}

¿Se alcanza el ínfimo hallado?

\pagebreak
\item ($\diamondsuit$) Resolver (a mano o en Mosek) los problemas SDP
\begin{multicols}{2}
\noindent
\begin{alignat*}{2}
  & \text{minimizar: } & & y \\
  & \text{sujeto a: } & \quad &
\begin{pmatrix}
5 & -12 & y \\
-12 & 27 - 2y & 1 \\
y & 1 & 10 \\
\end{pmatrix}  \succeq 0
\end{alignat*}
\begin{alignat*}{2}
  & \text{maximizar: } & & y \\
  & \text{sujeto a: } & \quad &
\begin{pmatrix}
5 & -12 & y \\
-12 & 27 - 2y & 1 \\
y & 1 & 10 \\
\end{pmatrix}  \succeq 0
\end{alignat*}
\end{multicols}

A partir de los resultados hallados, determinar el espectrahedro del conjunto factible.

\item ($\diamondsuit$) Para el problema SDP primal:
\begin{alignat*}{2}
  & \text{minimizar: } & & 2x_{11} + 2x_{12} \\
  & \text{sujeto a: } & \quad & x_{11} + x_{22} = 1, \\
  & & & \begin{pmatrix} x_{11} & x_{12} \\ x_{12} & x_{22}  \end{pmatrix} \succeq 0.
\end{alignat*}
\begin{enumerate}
\item Resolver el problema.
\item Plantear el problema dual.
\item Resolver el problema dual y calcular el salto de dualidad.
\end{enumerate}

\item Resolver el par de problemas SDP primal/dual y calcular el salto de dualidad.
\begin{multicols}{2}
\noindent
\begin{alignat*}{2}
  & \text{minimizar: } & & \alpha x_{11} \\
  & \text{sujeto a: } & \quad & x_{22} = 0, \\
  & & \quad & x_{11} + 2x_{23} = 1, \\
  & & \quad & \Xb \in \R^{3 \times 3} \succeq 0.
\end{alignat*}
\begin{alignat*}{2}
  & \text{maximizar: } & & y_2 \\
  & \text{sujeto a: } & & \begin{pmatrix} y_2 & 0 & 0 \\ 0 & y_1 & y_2 \\ 0 & y_2 & 0  \end{pmatrix} \preceq \begin{pmatrix} \alpha & 0 & 0 \\ 0 & 0 & 0 \\ 0 & 0 & 0 \end{pmatrix}.
\end{alignat*}
\end{multicols}

\item Considerar el par de problemas SDP primal/dual:
\begin{multicols}{2}
\noindent
\begin{alignat*}{2}
  & \text{minimizar: } & & x_{11} \\
  & \text{sujeto a: } & \quad & 2x_{12} = 1, \\
  & & & \begin{pmatrix} x_{11} & x_{12} \\ x_{12} & x_{22}  \end{pmatrix} \succeq 0.
\end{alignat*}
\begin{alignat*}{2}
  & \text{maximizar: } & & y \\
  & \text{sujeto a: } & & \begin{pmatrix} 0 & y \\ y & 0  \end{pmatrix} \preceq \begin{pmatrix} 1 & 0 \\ 0 & 0  \end{pmatrix}.
\end{alignat*}
\end{multicols}

\begin{enumerate}
\item Resolver ambos problemas y calcular el salto de dualidad.
\item ¿Se alcanza el óptimo en ambos problemas?
\end{enumerate}

\end{enumerate}

\end{document}
