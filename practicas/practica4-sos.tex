\documentclass[11pt]{article}
%\include{amsfonts}
%\usepackage{amssymb,amsmath,latexsym,epsfig,euscript}
\usepackage{amssymb}
\usepackage{enumitem}
\usepackage{amsmath}
\usepackage{multicol}

\usepackage{bm}


%
\def\d{\displaystyle}
%
\usepackage[heightrounded]{geometry}
\geometry{top=2cm,bottom=2cm,left=2cm,right=2cm}
\parindent=0pt

\usepackage{graphicx}
\graphicspath{ {../images} }

\usepackage{listings}
\lstset{
  basicstyle=\ttfamily,
  columns=fullflexible,
}

\usepackage{url}
\usepackage{multicol}
\usepackage{dsfont}

% Bold symbols for vectors and matrices
\newcommand{\xstar}{\bm{x}^{\star}}
\newcommand{\alphab}{\bm{\alpha}}
\newcommand{\ab}{\bm{a}}
\newcommand{\bb}{\bm{b}}
\newcommand{\cb}{\bm{c}}
\newcommand{\db}{\bm{d}}
\newcommand{\eb}{\bm{e}}
\newcommand{\gb}{\bm{g}}
\newcommand{\mb}{\bm{m}}
\newcommand{\pb}{\bm{p}}
\newcommand{\qb}{\bm{q}}
\newcommand{\rb}{\bm{r}}
\newcommand{\ssb}{\bm{s}}
\newcommand{\ub}{\bm{u}}
\newcommand{\vb}{\bm{v}}
\newcommand{\wb}{\bm{w}}
\newcommand{\xb}{\bm{x}}
\newcommand{\yb}{\bm{y}}
\newcommand{\zb}{\bm{z}}

\newcommand{\Ab}{\bm{A}}
\newcommand{\Bb}{\bm{B}}
\newcommand{\Cb}{\bm{C}}
\newcommand{\Db}{\bm{D}}
\newcommand{\Eb}{\bm{E}}
\newcommand{\Fb}{\bm{F}}
\newcommand{\Gb}{\bm{G}}
\newcommand{\Hb}{\bm{H}}
\newcommand{\Ib}{\bm{I}}
\newcommand{\Id}{\bm{I}}
\newcommand{\Kb}{\bm{K}}
\newcommand{\Lb}{\bm{L}}
\newcommand{\Mb}{\bm{M}}
\newcommand{\Pb}{\bm{P}}
\newcommand{\Qb}{\bm{Q}}
\newcommand{\Rb}{\bm{R}}
\newcommand{\Sb}{\bm{S}}
\newcommand{\Tb}{\bm{T}}
\newcommand{\Ub}{\bm{U}}
\newcommand{\Vb}{\bm{V}}
\newcommand{\Wb}{\bm{W}}
\newcommand{\Xb}{\bm{X}}
\newcommand{\Yb}{\bm{Y}}
\newcommand{\Zb}{\bm{Z}}
\newcommand{\Lambdab}{\bm{\Lambda}}
\newcommand{\cero}{\bm{0}}

% Rings and fields
\newcommand{\A}{\mathbb{A}}
\newcommand{\Z}{\mathbb{Z}}
\newcommand{\Q}{\mathbb{Q}}
\newcommand{\C}{\mathbb{C}}
\newcommand{\R}{\mathbb{R}}
\newcommand{\K}{\mathbb{K}}
\newcommand{\N}{\mathbb{N}}

\newcommand{\borel}{{\mathcal B}}
\newcommand{\pmom}{{\rho_{\text{mom}}}}
\newcommand{\MX}{{\mathcal{M}(X)}}


% Inner product
\newcommand{\innerl}[2]{\langle #1, #2 \rangle}
\newcommand{\inner}[2]{#1 \boldsymbol{\cdot} #2}
\newcommand{\innerTrace}[2]{#1 \bullet #2}

% Symmetric and positive definite matrices
\newcommand{\Splusplusn}{{\mathcal S_{++}^n}}
\newcommand{\Splusn}{{\mathcal S_+^n}}
\newcommand{\Splus}{{\mathcal S_+}}
\newcommand{\Sym}{{\mathcal S}}
\newcommand{\Symn}{{\mathcal S^n}}

% Cones
\newcommand\CC{\mathcal{C}}
\DeclareMathOperator{\cone}{cono}
\DeclareMathOperator{\conv}{conv}
\DeclareMathOperator{\supp}{supp}


% Spectrahedron
\newcommand{\eLL}{{\mathcal L}}

% Matrices and vectors over R or C
\newcommand{\Rnn}{\R^{n\times n}}
\newcommand{\Cnn}{\C^{n\times n}}
\newcommand{\Rn}{\R^{n}}
\newcommand{\Rm}{\R^{m}}


% Math operators
\DeclareMathOperator{\Tr}{Tr}
\DeclareMathOperator{\tr}{Tr}
\DeclareMathOperator{\interior}{int}
\DeclareMathOperator{\rank}{rank}
\DeclareMathOperator{\diag}{diag}

\newcommand\one{\mathds{1}} 


\begin{document}

\begin{center}

{\small Facultad de Ciencias Exactas
y Naturales -- Universidad de Buenos Aires } \vskip 1cm

\textbf{{\large Optimización Semidefinida} - Segundo Cuatrimestre 2021}

\medskip\textbf{Pr\'actica 4 - Sumas de cuadrados}
\end{center}

\medskip

\textbf{Para entregar:} entregar un ejercicio a elección entre los marcados ($\diamondsuit$).

\vspace{0.5cm}

\begin{enumerate}

\item Calcular las raíces de
$$
p(x) = x^4 + 2x^3 + 6x^2 - 22x + 13
$$
y obtener una descomposición de $p(x)$ como suma de cuadrados.

\item Encontrar una descomposición como suma de cuadrados de
$$
p(x) = x^4 + 4x^3 + 6x^2 + 4x + 5
$$
resolviendo a mano el problema SDP asociado.

\item Un polinomio trigonométrico de grado $d$ es una expresión de la forma
$$
p(\theta) = a_0 + \sum_{k=1}^d (a_k \cos{k \theta} + b_k \sin{k \theta}).
$$
\begin{enumerate}
\item Probar que si $p(\theta)$ es un polinomio trigonométrico de grado $2d$ y $p(\theta) \ge 0$ para todo $\theta \in [-\pi, \pi]$, entonces $p$ admite una descomposición
    $$
    p(\theta) = q_1^2(\theta) + q_2^2(\theta)
    $$
con $q_1$, $q_2$ polinomios trigonométricos.

\item Plantear un problema SDP para determinar un polinomio trigonométrico $p(\theta)$ de grado $2d$ satisface $p(\theta) \ge 0$ para todo $\theta$.

\item Hallar una descomposición como suma de cuadrados de polinomios trigonométricos de la función
$$
p(\theta) = 4 - \sin(\theta) + \sin(2\theta) - 3 \cos(2\theta).
$$

\end{enumerate}

\item (*) ($\diamondsuit$) Demostrar un polinomio de grado 2 en $n$ variables es positivo si y solo si se puede escribir como una suma de cuadrados.

\item (*) Demostrar que el problema de determinar si un polinomio es positivo (o no-negativo) es un problema NP-hard, reduciendo otro problema NP-hard a este problema.

\item Calcular a mano una escritura como suma de cuadrados para el polinomio
$$
p(x_1, x_2) = x_1^2 - x_1 x_2^2 + x_2^4 + 1.
$$

\item Hallar fórmulas combinatorias para la cantidad de monomios de grado $d$ en $n$ variables y para la cantidad de monomios de grado menor o igual que $d$ en $n$ variables.

\item Utilizando Mosek, obtener una descomposición como suma de cuadrados de
$$
p(x,y,z,w) = 2x^4 + x^2y^2 + y^4 - 4x^2 - 4xyz - 2y^2w + y^2 - 2yz + 8z^2 - 2zw + 2w^2.
$$
    
\item ($\diamondsuit$) \textbf{Descomposición de polinomios homogéneos.}

\begin{enumerate}
\item Implementar en Mosek un programa para determinar si un polinomio homogéneo de grado $2d$ en $n$ variables admite una descomposición como suma de cuadrados resolviendo el problema SDP $p = \vb^t \Qb \vb$, $\Qb \succeq 0$, con $\vb$ el vector de monomios de grado $d$ en $n$ variables.
\item Utilizar el programa para determinar si el polinomio 
$$
p(x,y,z) = 2(x^6 + y^6 + z^6) - 2(x^4y^2 + x^4z^2 + y^4x^2 + y^4z^2 + z^4x^2 + z^4y^2) + 6x^2y^2z^2
$$
admite una descomposición como suma de cuadrados.

\item Hallar $\beta_1 > 0$ tal que $p(x,y,z) + \beta(x^6 + y^6 + z^6)$ sea una suma de cuadrados.
\end{enumerate}

\item En el ejercicio anterior, hallar $\beta_2 > 0$ tal que $p(x,y,z) + \beta(x^6 + y^6 + z^6)$ no sea una suma de cuadrados.

\item Considerar el polinomio $f \in \R[x,y,z]_6$,
$$
p=x^6 + y^6 + 7(x^4 + y^4)z^2 + 18x^2y^2z^2 - 23(x^2 + y^2) z^4 + 16 z^6
$$

Verificar en Mosek mediante distintas funciones objetivo que existe una única matriz $\Qb \succeq 0$ tal que $p = \vb^t \Qb \vb$.

\item \textbf{(Descomposición racional)} Dado un polinomio $p(\xb) \in \Q[\xb]_{2d}$, demostrar que $p$ se puede escribir como una suma de cuadrados de polinomios en $\Q[\xb]$ si y solo si existe una matriz $\Qb$ con entradas racionales en el espectrahedro de Gram de $p$ (es decir una matriz racional $\Qb \succeq 0$ tal que $p = \vb^t \Qb \vb$ para $\vb$ el vector de monomios de grado $d$).
    
\item Dado un polinomio $p(\xb) \in \Q[\xb]_{2d}$, probar que si el problema $p = \vb^t \Qb \vb$, $\Qb \succeq 0$, es estrictamente factible, entonces $p$ admite una descomposición como suma de cuadrados con coeficientes racionales.


\item ($\diamondsuit$) Resolver utilizando Mosek el siguiente problema SOS
\begin{alignat*}{2}
  & \text{maximizar: } & & y_1 + y_2  \\
  & \text{sujeto a: } & \quad & x^4 + y_1 x + (2 + y_2) \quad \text{es SOS,} \\
  &  & \quad & (y_1 - y_2 + 1)x^2 + y_2 x + 1 \quad  \text{es SOS.}
\end{alignat*}

\item ($\diamondsuit$) Plantear el problema de hallar el mínimo global de la función
$$
q(x) = \frac{x^3 - 8x + 1}{x^4 + x^2 + 12}
$$
como un programa SOS. Resolver el problema en Mosek.

    
    
    

\end{enumerate}



Los ejercicios marcados (*) son optativos y pueden involucrar temas no vistos en la materia.
\end{document}
